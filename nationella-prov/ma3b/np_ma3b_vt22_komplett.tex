\documentclass{article}
\usepackage[T1]{fontenc}
\usepackage[utf8]{inputenc}
\usepackage{amsmath}
\usepackage{amssymb}
\usepackage{geometry}
\geometry{a4paper, margin=1in}

\title{Nationellt Prov Matematik 3b VT22 \\ \large Komplett med uppgifter och facit}
\author{Extraherat från officiellt NP-material}
\date{Vårterminen 2022}

\begin{document}

\maketitle

\tableofcontents
\newpage

\section{Provstruktur}

\subsection*{Provdelar}
\begin{itemize}
    \item \textbf{Delprov B:} Uppgift 1--11. Endast svar krävs. Inga digitala verktyg.
    \item \textbf{Delprov C:} Uppgift 12--18. Fullständiga lösningar krävs. Inga digitala verktyg.
    \item \textbf{Delprov D:} Uppgift 19--28. Fullständiga lösningar krävs. Digitala verktyg tillåtna.
\end{itemize}

\subsection*{Provtid}
\begin{itemize}
    \item Delprov B och C tillsammans: 120 minuter
    \item Delprov D: 120 minuter
\end{itemize}

\subsection*{Poängfördelning}
Totalt: 59 poäng (21 E-, 22 C-, 16 A-poäng)

\subsection*{Betygsgränser}
\begin{itemize}
    \item \textbf{E:} 15 poäng
    \item \textbf{D:} 24 poäng varav 7 poäng på minst C-nivå
    \item \textbf{C:} 31 poäng varav 13 poäng på minst C-nivå
    \item \textbf{B:} 39 poäng varav 5 poäng på A-nivå
    \item \textbf{A:} 46 poäng varav 8 poäng på A-nivå
\end{itemize}

\newpage

\section{Delprov B - Endast svar krävs}

\subsection*{Uppgift 1 (1/0/0)}
Ett av alternativen A--D är ett exempel på en primitiv funktion till funktionen $f(x) = 3x^2 - 2x$. Vilket?

\begin{itemize}
    \item[A.] $F(x) = 3x - 2$
    \item[B.] $F(x) = x^4 - x^2$
    \item[C.] $F(x) = x^3 - x^2$
    \item[D.] $F(x) = 4x^2 - 2x$
\end{itemize}

\textbf{Facit:} C

\subsection*{Uppgift 2 (1/0/0)}
Den 1 augusti varje sommar inventeras (räknas) antalet gråsälar i Östersjön. Tabellen visar resultatet.

\begin{center}
\begin{tabular}{|c|c|}
\hline
År & Antal gråsälar \\
\hline
2013 & 28\,000 \\
2014 & 32\,000 \\
2015 & 31\,000 \\
2016 & 30\,000 \\
2017 & 30\,000 \\
2018 & 34\,000 \\
2019 & 38\,000 \\
\hline
\end{tabular}
\end{center}

Använd tabellen och bestäm den genomsnittliga förändringshastigheten för antalet gråsälar från den 1 augusti 2015 till den 1 augusti 2018.

\textbf{Facit:} 1000 sälar/år

\subsection*{Uppgift 3 (1/0/0)}
Figuren visar grafen till funktionen $f$.

\textit{[FIGUR: Graf till funktion f]}

Använd grafen och ange vilket av alternativen A--F som är det bästa värdet till $f'(2)$.

\begin{itemize}
    \item[A.] 4
    \item[B.] 2
    \item[C.] 0,5
    \item[D.] $-0,5$
    \item[E.] $-2$
    \item[F.] $-4$
\end{itemize}

\textbf{Facit:} E ($-2$)

\subsection*{Uppgift 4 (1/1/0)}
De 100 första talen i en talföljd bildar den geometriska summan $S_{100} = 2(5^{100} - 1)$

\textbf{a)} Bestäm det första talet i talföljden.

\textbf{Facit:} 2

\textbf{b)} Bestäm det fjärde talet i talföljden.

\textbf{Facit:} 250

\subsection*{Uppgift 5 (1/0/0)}
Figuren visar två räta linjer och sex punkter A--F.

\textit{[FIGUR: Två linjer och sex punkter]}

En av punkterna A--F ligger i området som begränsas av
\[
\begin{cases}
x > 0 \\
y < x - 5 \\
y > -x + 1
\end{cases}
\]

Vilken?

\textbf{Facit:} C

\subsection*{Uppgift 6 (1/1/1)}
Bestäm $f'(x)$ då

\textbf{a)} $f(x) = 4x^3 - 12x$

\textbf{Facit:} $f'(x) = 12x^2 - 12$

\textbf{b)} $f(x) = 2ax^4 - x^{-2}$

\textbf{Facit:} $f'(x) = 8ax^3 + 2x^{-3}$

\textbf{c)} $f(x) = 3 \cdot 2^x - e^{-x}$

\textbf{Facit:} $f'(x) = 3 \cdot 2^x \ln 2 + e^{-x}$

\subsection*{Uppgift 7 (0/1/0)}
Lös ekvationen $3x^4 = 8x^2 - 4$

\textbf{Facit:} $x_1 = \pm\sqrt{2}$, $x_2 = \pm\frac{\sqrt{6}}{3}$ (eller $x \approx \pm 1,41$ och $x \approx \pm 0,82$)

\subsection*{Uppgift 8 (1/1/1)}
Förenkla uttrycken så långt som möjligt.

\textbf{a)} $\frac{x^3 + 6x}{x^3 - 5x}$

\textbf{Facit:} $\frac{x^2 + 6}{x^2 - 5}$

\textbf{b)} $\frac{x^2 + 12x + 18}{2(x^2 - 9)}$

\textbf{Facit:} $\frac{x + 3}{2(x - 3)}$

\textbf{c)} $e^{2x} \cdot e^{-ax} \cdot e^{-x} \cdot e^{0,5ax}$

\textbf{Facit:} $e^{x(1-0,5a)}$ (eller $e^{2x}$)

\subsection*{Uppgift 9 (0/1/0)}
Figuren visar grafen till funktionen $f$.

\textit{[FIGUR: Graf till funktion f]}

Ett av alternativen A--F visar grafen till funktionens derivata $f'$. Vilket?

\textit{[FIGUR: Sex alternativa grafer A--F]}

\textbf{Facit:} B

\subsection*{Uppgift 10 (0/2/0)}
Figuren visar huvuddragen av grafen till funktionen $p$.

\textit{[FIGUR: Graf till funktion p]}

Bestäm för vilka värden på $x$ som

\textbf{a)} $p'(x) < 0$

\textbf{Facit:} $0 < x < 3$

\textbf{b)} uttrycket $\frac{p(x)}{p'(x)}$ inte är definierat.

\textbf{Facit:} 0 och 3

\subsection*{Uppgift 11 (0/0/1)}
Figuren visar grafen till funktionen $f$.

\textit{[FIGUR: Graf till funktion f]}

Bestäm ett värde på $a$ så att $\int_{-1}^{a} f(x) \, dx = 3$

\textbf{Facit:} 1,8 (Svar i intervallet $1,7 \leq a \leq 1,9$ ges poäng)

\newpage

\section{Delprov C - Fullständiga lösningar krävs}

\subsection*{Uppgift 12 (1/0/0)}
Tilde deriverar funktionen $f(x) = e^{2x}$ och ställer upp kvoten $\frac{f'(x)}{f(x)}$

Hon påstår följande: "För alla värden på $x$ kommer kvoten alltid att få värdet 2".

Har Tilde rätt? Motivera ditt svar.

\textbf{Facit:} Ja, Tilde har rätt. $f'(x) = 2e^{2x}$, så $\frac{f'(x)}{f(x)} = \frac{2e^{2x}}{e^{2x}} = 2$ för alla $x$.

\subsection*{Uppgift 13 (2/0/0)}
Beräkna $\int_{1}^{2} 3x^2 \, dx$

\textbf{Facit:} 7

\subsection*{Uppgift 14 (3/1/0)}
Funktionen $f$ ges av $f(x) = x^3 - 3x^2 + 7$

Använd derivata och bestäm koordinaterna för eventuella maximi-, minimi- och terrasspunkter för funktionens graf.

Avgör också, för varje sådan punkt, om det är en maximi-, minimi- eller terrasspunkt.

\textbf{Facit:}
\begin{itemize}
    \item Maximipunkt: $(0, 7)$
    \item Minimipunkt: $(2, 3)$
\end{itemize}

\subsection*{Uppgift 15 (0/2/0)}
Figuren visar ett gråmarkerat område som begränsas av grafen till funktionen $g$, den räta linjen $x = 3$ samt de positiva koordinataxlarna.

\textit{[FIGUR: Gråmarkerat område]}

Funktionen $g$ ges av $g(x) = px^2 + 5x$ där $p$ är en konstant.

Bestäm $p$ så att det gråmarkerade områdets area blir 24 areaenheter.

\textbf{Facit:} $p = 4$

\subsection*{Uppgift 16 (0/2/0)}
Funktionen $f$ ges av $f(x) = x^3 + 3x$

Jaana påstår att funktionen $f$ har två extrempunkter.

Har Jaana rätt? Motivera ditt svar.

\textbf{Facit:} Nej, Jaana har fel. $f'(x) = 3x^2 + 3 = 3(x^2 + 1)$ som alltid är positiv. Funktionen saknar därför extrempunkter.

\subsection*{Uppgift 17 (0/1/3)}
Funktionen $f$ ges av $f(x) = \frac{5}{ax^2}$ där $x \neq 0$ och $a \neq 0$

Bestäm $f'(x)$ med hjälp av derivatans definition.

\textbf{Facit:} $f'(x) = -\frac{10}{ax^3}$

\subsection*{Uppgift 18 (0/0/3)}
Figuren visar grafen till tredjegradsfunktionen $f$ som ges av $f(x) = x^3$ och en tangent till grafen i den punkt där $x = a$. Tangenten, den positiva $x$-axeln och linjen $x = a$ begränsar ett område som har formen av en triangel.

\textit{[FIGUR: Graf med tangent och triangel]}

Bestäm $a$ så att triangeln får arean 1,5 areaenheter.

\textbf{Facit:} $a = \sqrt[4]{9}$ (eller $a \approx 1,73$)

\newpage

\section{Delprov D - Digitala verktyg tillåtna}

\subsection*{Uppgift 19 (1/0/0)}
Funktionen $f$ som ges av $f(x) = (2x - 1)^5$ kan inte deriveras med hjälp av deriveringsreglerna inom denna kurs.

Använd ditt digitala verktyg för att beräkna ett värde på $f'(2)$.

Endast svar krävs.

\textbf{Facit:} 810

\subsection*{Uppgift 20 (2/0/0)}
En geometrisk summa ges av $B + B \cdot 1,4 + B \cdot 1,4^2 + \ldots + B \cdot 1,4^{21}$ där $B$ är en konstant.

Bestäm $B$ så att summan blir 250\,000.

\textbf{Facit:} 61

\subsection*{Uppgift 21 (2/0/0)}
Grafen till funktionen $f(x) = 3x^2 + 4$ har en tangent i den punkt där $x = 2$.

Tangentens ekvation kan skrivas $y = kx - 12$

Bestäm $k$.

\textbf{Facit:} $k = 16$

\subsection*{Uppgift 22 (3/1/0)}
Pojkars längd kan beskrivas med den enkla modellen $f(x) = 78 \cdot e^{0,07x}$ där $f(x)$ är längden i centimeter och $x$ är pojkars ålder i år.

\textbf{a)} Bestäm vid vilken ålder som pojkar är 125 cm långa enligt modellen.

\textbf{Facit:} 6,7 år

\textbf{b)} Använd modellen och bestäm hur snabbt pojkar växer då de är exakt 6 år.

\textbf{Facit:} 8,3 cm/år

\textbf{c)} Undersök om modellen även är giltig för pojkar som går på gymnasiet.

\textbf{Facit:} Nej, modellen är inte giltig. (Resonemang krävs, t.ex. att modellen ger orimligt höga värden för 16--19-åringar)

\subsection*{Uppgift 23 (0/2/0)}
Funktionerna $f$ och $g$ ges av
\[
f(x) = \frac{12}{x} + 8 \quad \text{och} \quad g(x) = x
\]

Lös ekvationen $f'(x) = g'(x)$. Svara med minst två decimaler.

\textbf{Facit:} $x = 1,26$

\subsection*{Uppgift 24 (0/4/0)}
Julius och Sophia planerar att starta en nätbutik för att sälja sittkuddar. De tänker sälja två modeller av sittkuddar, modell A och modell B.

Inköpspriset för en sittkudde av modell A är 600 kr och för modell B 400 kr. De kan köpa in sittkuddar för totalt 60\,000 kr. I deras lagerlokal ryms det som mest 125 sittkuddar.

Julius och Sophia räknar med att sälja alla kuddar som köps in och att vinsten blir 500 kr för varje såld kudde av modell A och 400 kr för varje såld kudde av modell B.

Bestäm hur många sittkuddar av varje modell som de ska köpa in för att vinsten ska bli maximal.

\textbf{Facit:} 50 sittkuddar av modell A och 75 sittkuddar av modell B

\subsection*{Uppgift 25 (0/2/0)}
Funktionen $f$ ges av $f(x) = 2^x$. Figuren visar grafen till funktionen $f$ samt en sekant mellan två punkter på grafen.

\textit{[FIGUR: Graf med sekant]}

Till grafen dras en tangent som är parallell med sekanten. Bestäm $x$-koordinaten för tangeringspunkten. Svara med minst två decimaler.

\textbf{Facit:} $x = 0,75$

\subsection*{Uppgift 26 (0/0/2)}
Funktionen $f$ ges av
\[
f(x) = a(x - a)(x - 2a)(x - 3a) = ax^3 - 6a^2x^2 + 11a^3x - 6a^4
\]
där $a$ är en konstant, $a > 0$.

Grafen till $f$ skär $x$-axeln i punkterna $P$, $Q$ och $R$. Se figur.

\textit{[FIGUR: Graf som skär x-axeln i tre punkter]}

Visa algebraiskt att tangenterna till grafen i punkterna $P$ och $R$ är parallella oavsett värde på konstanten $a$.

\textbf{Facit:} $f'(a) = 2a^3$ och $f'(3a) = 2a^3$, vilket visar att tangenterna är parallella.

\subsection*{Uppgift 27 (0/0/2)}
Wilma har en gammal moped.

Bensinförbrukningen för mopeden kan beskrivas med den förenklade modellen $f(x) = 0,3 + 0,5e^{-0,76x}$ där $f(x)$ är bensinförbrukningen i liter/mil och $x$ är sträckan i mil från start.

Wilma startar med 4,0 liter bensin i tanken. Bestäm hur lång sträcka Wilma kan köra som längst innan bensinen tar slut enligt modellen.

\textbf{Facit:} 11 mil

\subsection*{Uppgift 28 (0/0/3)}
Konstsmeden Suzanna tänker göra smycken av silver och guld. Varje smycke ska bestå av en rektangulär silverplatta och en guldtråd. Guldtråden ska lödas fast 8 mm från silverplattans hörn. Se figur.

\textit{[FIGUR: Rektangulär platta med guldtråd]}

Guldtråd är dyr och hon vill därför använda så lite guld som möjligt till smycket. Smycket får inte heller väga för mycket och därför bestämmer Suzanna att en silverplatta ska ha arean 550 mm$^2$.

Bestäm vilken längd guldtråden får om Suzanna använder så lite guldtråd som möjligt till smycket.

\textbf{Facit:} 23 mm

\end{document}
