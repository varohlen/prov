\documentclass{exam}
\usepackage{graphicx} 


%Format Header and footer
\pagestyle{headandfoot}
\header{\footnotesize Klass:\\Namn:}{\Large\textbf{Hållbarhet och ekologi}}{\footnotesize Nak1a1\\2023}
\headrule
\footrule
\setlength{\columnsep}{0.25cm}
%\setlength{\columnseprule}{1pt}
\footer{}{Sida \thepage}{}
%\extrafootheight{-2cm}

\begin{document}
\vspace{5mm} %5mm vertical space
\begin{center}
\fbox{\fbox{\parbox{6in}{\centering
\textbf{14 kryssfrågor}: markera endast ett alternativ.
}}}
\end{center}
\vspace{5mm} %5mm vertical space

\begin{questions}
\printanswers
\question Vilken av följande är \textbf{\textit{inte}} en \textit{dimension} av \textbf{hållbar utveckling}?
\begin{checkboxes}
   \choice Social hållbarhet
   \choice Ekologisk hållbarhet
   \correctchoice Biologisk hållbarhet
   \choice Ekonomisk hållbarhet
\end{checkboxes} 

\vspace{5mm} 
\hrule 
\vspace{5mm} 
\question Vad av följande är en \textbf{icke-förnyelsebar resurs}:
\begin{checkboxes}
   \choice Vatten
   \correctchoice Litium
   \choice Bomull
   \choice Solenergi
\end{checkboxes}

\vspace{5mm} 
\hrule 
\vspace{5mm} 
\question \textbf{Globalhektar} används ofta för att mäta ekologiska fotavtryck. Det används eftersom:
\begin{checkboxes}
   \choice det representerar alla länder
   \correctchoice det är ett hektar med jordens genomsnittliga produktion och visar tydligt hur mycket en person/företag/land förbrukar
   \choice det mäter hur mycket resurser vi har kvar på jorden
   \choice det låter bra och viktigt
\end{checkboxes}

\vspace{5mm} 
\hrule 
\vspace{5mm} 
\question \textbf{Globala målen} antogs 2015 i form av Agenda 2030. Syftet med dessa är att:
\begin{checkboxes}
   \correctchoice främja en hållbar utveckling på globlalnivå
   \choice utvärdera vilka länder som är mest hållbara
   \choice jämna ut skillnaden mellan rika- och fattiga länder
   \choice arrangera miljömässor för världens alla länder
\end{checkboxes}

\vspace{5mm} 
\hrule 
\vspace{5mm} 
\question Varför är det viktigt att \textbf{återvinna utjänta elektronikprodukter} (ex. mobiler och datorer)? 
\begin{checkboxes}
   \choice för att de riskerar att avge farlig strålning om de ej tas hand om
   \choice eftersom de sparar samhället mycket pengar om de kan återvinns
   \correctchoice för att återvinna de icke-förnyelsebara resurser som krävs för att tillverka elektroniska produkter
   \choice eftersom de kan innehålla privat information som riskerar att spridas
\end{checkboxes}
\break


\question Vad består ett \textbf{ekosystem} av? 
\begin{checkboxes}
   \choice alla organismer på en plats
   \choice arter, population och individer
   \correctchoice organismer och omgivande miljö
   \choice en specifik miljö med och fotosyntetiserande arter 
\end{checkboxes}

\vspace{5mm} 
\hrule 
\vspace{5mm} 
\question Vad innebär de \textbf{abiotiska faktorerna}
\begin{checkboxes}
   \choice olika faktorer i ett ekosystem
   \correctchoice de icke-levande faktorerna i ett ekosystem
   \choice de levande faktorerna i ett ekosystem
   \choice de faktorer som har minst påverkan i ett ekosystem
\end{checkboxes}

\vspace{5mm} 
\hrule 
\vspace{5mm} 
\question En \textbf{population} är:
\begin{checkboxes}
   \choice en grupp av olika arter i ett ekosystem
   \choice en specifik grupp arter
   \choice alla arter i ett ekosystem
   \correctchoice alla individer av en art i ett ekosystem 
\end{checkboxes}

\vspace{5mm} 
\hrule 
\vspace{5mm} 
\question En \textbf{ekologisk nisch} är:
\begin{checkboxes}
   \choice samma sak som ett habitat
   \choice en typ av ekosystem
   \correctchoice habitat och beteende
   \choice en arts miljö 
\end{checkboxes}

\vspace{5mm} 
\hrule 
\vspace{5mm} 
\question En \textbf{producent} är:
\begin{checkboxes}
   \choice arter som kan skapa koldioxid
   \choice en encellig art som är anfader till övriga arter i ett ekosystem
   \choice arter som försöjer ett ekosystem med viktiga näringsämnen
   \correctchoice en art som kan utvinna energi ur enkla ämnen genom ex. fotosyntes
\end{checkboxes}

\vspace{5mm} 
\hrule 
\vspace{5mm} 
\question Varför är \textbf{nedbrytare} viktiga i ett ekosystem?
\begin{checkboxes}
   \choice de håller ekosystemen rena och fria från gifter
   \correctchoice de utvinner energi ur dött material och återför viktiga näringsämnen till näringsväven
   \choice de förser toppkonsumenterna med energi
   \choice de är viktiga i de ekosystem som saknar producenter för att kunna tillföra energi genom fotosyntes
\end{checkboxes}
\break

\question En \textbf{näringsväv} är:
\begin{checkboxes}
   \correctchoice en modell över hur näringsämnen och energi överförs mellan olika organismer
   \choice hur näring tas upp av olika organismer
   \choice utvinningen av närsalter ur jorden av producenter
   \choice ett annat ord för näringskedja
\end{checkboxes}

\vspace{5mm} 
\hrule 
\vspace{5mm} 
\question Mellan varje steg \textit{uppåt} i en \textbf{näringskedja} så:
\begin{checkboxes}
   \choice ökar den totala mängden energi
   \choice är mängden energi konstant
   \correctchoice minskar den totala biomassan
   \choice är alltid biomassan konstant
\end{checkboxes}

\vspace{5mm} 
\hrule 
\vspace{5mm} 
\question \textbf{Ekosystemets bärförmåga} syftar på:
\begin{checkboxes}
   \choice hur många arter som kan befinna sig i samma ekosystem
   \choice hur många olika populationer som kan överleva i ett ekosystem
   \correctchoice hur stor en population i ett ekosystem kan bli
   \choice hur många näringskedjor som kan samsas i ett ekosystem
\end{checkboxes}
\break
\vspace{5mm} %5mm vertical space
\begin{center}
\fbox{\fbox{\parbox{6in}{\centering
\textbf{3 frisvarsfrågor}: Svara på utrymmet under frågan. Använd relevanta begrepp och figurer. (C-A)
}}}
\end{center}
\vspace{5mm} %5mm vertical space
\includegraphics[width=0.2\textwidth]{image.png}
\centering
\begin{questions}

\question 
Det \textbf{ekologiska fotavtrycket} mellan olika länder skiljer sig. Ovan ser ni ett antal länders olika fotavtryck illustrerade. Varför skiljer sig fotavtrycket så mycket mellan ett land som \textit{Sverige} och \textit{Indien}?

\vspace{80mm} %5mm vertical space

\question 
Vad är problemet med \textbf{invaisva arter}? Använd relevanta begrepp och förklara också vad en invasiv art är.

\break

\question
Hur förhåller sig \textbf{människan} som art till ett ekosystems alla olika regler? Vad gäller människan när det kommer till \textit{ekologisk nisch} och \textit{ekosystemets bärförmåga}? På vilket sätt skiljer vi oss och vad har det för konsekvenser?







\end{questions}

\end{document}

