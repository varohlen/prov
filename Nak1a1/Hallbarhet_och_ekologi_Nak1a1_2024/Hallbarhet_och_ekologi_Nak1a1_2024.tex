\documentclass{exam}
\usepackage{graphicx} 

\begin{document}
%Format Header and footer
\pagestyle{headandfoot}
\header{\footnotesize Klass:\\Namn:}{\Large\textbf{Hållbarhet och ekologi}}{\footnotesize Nak1a1\\2023}
\headrule
\footrule
\setlength{\columnsep}{0.25cm}
%\setlength{\columnseprule}{1pt}
\footer{}{Sida \thepage}{}
%\extrafootheight{-2cm}

\vspace{5mm} %5mm vertical space
\begin{center}
\fbox{\fbox{\parbox{6in}{\centering
\textbf{13 kryssfrågor}: markera endast ett alternativ.
}}}
\end{center}
\vspace{5mm} %5mm vertical space

\begin{questions}
    
\question Vilken av följande är en \textit{dimension} av \textbf{hållbar utveckling}? (Flera alternativ kan vara korrekta)
\begin{checkboxes}
   \choice Social hållbarhet
   \choice Ekologisk hållbarhet
   \choice Biologisk hållbarhet
   \choice Ekonomisk hållbarhet
\end{checkboxes} 

\vspace{5mm} 
\hrule 
\vspace{5mm} 
\question Vad av följande är en \textbf{förnyelsebar resurs}:
\begin{checkboxes}
   \choice Olja
   \choice Litium
   \choice Uran
   \choice Bomull
\end{checkboxes}

\vspace{5mm} 
\hrule 
\vspace{5mm} 
\question Vilket av följande är ett exempel på en \textbf{producent} i ett ekosystem?
\begin{checkboxes}
   \choice En art som konsumerar andra organismer
   \choice En art som bryter ner dött organiskt material
   \choice En växt som producerar sin egen energi genom fotosyntes
   \choice En art som lever i symbios med andra arter
\end{checkboxes}

\vspace{5mm} 
\hrule 
\vspace{5mm} 
\question Varför är det viktigt att \textbf{återvinna utjänta elektronikprodukter} (ex. mobiler och datorer)? 
\begin{checkboxes}
   \choice för att de riskerar att avge farlig strålning om de ej tas hand om
   \choice eftersom de sparar samhället mycket pengar om de kan återvinns
   \choice för att återvinna de icke-förnyelsebara resurser som krävs för att tillverka elektroniska produkter
   \choice eftersom de kan innehålla privat information som riskerar att spridas
\end{checkboxes}
\vspace{5mm} 
\hrule 
\vspace{5mm} 

\question Vad består ett \textbf{ekosystem} av? 
\begin{checkboxes}
   \choice alla organismer på en plats
   \choice arter, population och individer
   \choice organismer och omgivande miljö
   \choice en specifik miljö med och fotosyntetiserande arter 
\end{checkboxes}

\break
\question Vilket av följande är en \textbf{biotisk faktor} i ett ekosystem?
\begin{checkboxes}
   \choice Solljus
   \choice Näringsämnen i jorden
   \choice Växter och djur
   \choice Temperatur
\end{checkboxes}
\vspace{5mm} 
\hrule 
\vspace{5mm} 
\question En \textbf{population} är:
\begin{checkboxes}
   \choice en grupp av olika arter i ett ekosystem
   \choice en specifik grupp arter
   \choice alla arter i ett ekosystem
   \choice alla individer av en art i ett ekosystem 
\end{checkboxes}

\vspace{5mm} 
\hrule 
\vspace{5mm} 
\question Vilket av följande exempel beskriver en ekologisk nisch?
\begin{checkboxes}
   \choice Platsen där en art lever
   \choice De specifika förhållanden och resurser som en art behöver för att överleva
   \choice En samling arter som lever tillsammans
   \choice Samma sak som ett habitat
\end{checkboxes}

\vspace{5mm} 
\hrule 
\vspace{5mm} 
\question En \textbf{producent} är:
\begin{checkboxes}
   \choice arter som kan skapa koldioxid
   \choice en encellig art som är anfader till övriga arter i ett ekosystem
   \choice arter som försöjer ett ekosystem med viktiga näringsämnen
   \choice en art som kan utvinna energi ur enkla ämnen genom ex. fotosyntes
\end{checkboxes}

\vspace{5mm} 
\hrule 
\vspace{5mm} 
\question Varför är \textbf{nedbrytare} viktiga i ett ekosystem?
\begin{checkboxes}
   \choice de håller ekosystemen rena och fria från gifter
   \choice de utvinner energi ur dött material och återför viktiga näringsämnen till näringsväven
   \choice de förser toppkonsumenterna med energi
   \choice de är viktiga i de ekosystem som saknar producenter för att kunna tillföra energi genom fotosyntes
\end{checkboxes}
\break

\question En \textbf{näringsväv} är:
\begin{checkboxes}
   \choice en modell över hur näringsämnen och energi överförs mellan olika organismer
   \choice hur näring tas upp av olika organismer
   \choice utvinningen av närsalter ur jorden av producenter
   \choice ett annat ord för näringskedja
\end{checkboxes}

\vspace{5mm} 
\hrule 
\vspace{5mm} 
\question Mellan varje steg \textit{uppåt} i en \textbf{näringskedja} så:
\begin{checkboxes}
   \choice ökar den totala mängden energi
   \choice är mängden energi konstant
   \choice minskar den totala biomassan
   \choice är alltid biomassan konstant
\end{checkboxes}

\vspace{5mm} 
\hrule 
\vspace{5mm} 
\question \textbf{Ekosystemets bärförmåga} syftar på:
\begin{checkboxes}
   \choice hur många arter som kan befinna sig i samma ekosystem
   \choice hur många olika populationer som kan överleva i ett ekosystem
   \choice hur stor en population i ett ekosystem kan bli
   \choice hur många näringskedjor som kan samsas i ett ekosystem
\end{checkboxes}
\break
\vspace{5mm} %5mm vertical space
\begin{center}
\fbox{\fbox{\parbox{6in}{\centering
\textbf{2 frisvarsfrågor}: Svara på utrymmet under frågan. Använd relevanta begrepp och figurer.
}}}
\end{center}
\vspace{5mm} %5mm vertical space

\question 
Vad är problemet med \textbf{invasiva arter}? Använd relevanta begrepp och förklara också vad en invasiv art är.

\vspace{80mm} %5mm vertical space

\question
Hur förhåller sig \textbf{människan} som art till ett ekosystems alla olika regler? Vad gäller människan när det kommer till \textit{ekologisk nisch} och \textit{ekosystemets bärförmåga}? På vilket sätt skiljer vi oss och vad har det för konsekvenser?

\end{questions}
\end{document}
