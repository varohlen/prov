\documentclass[a4paper,11pt]{article}
\usepackage[utf8]{inputenc}
\usepackage[T1]{fontenc}
\usepackage[swedish]{babel}
\usepackage{amsmath,amssymb,amsfonts}
\usepackage{graphicx}
\usepackage{enumitem}
\usepackage{geometry}
\geometry{margin=2.5cm}

\title{Logaritmer}
\author{}
\date{\today}

\begin{document}

\maketitle

%\section{Logaritmer}

%\begin{enumerate}[label=\textbf{\arabic*.}]
%    \item Beräkna: $\lg(100 \cdot 1000)$
    
%    \item Beräkna: $\lg\left(\frac{100}{0,1}\right)$
    
%    \item Beräkna: $\lg 5 + \lg 20$
    
%    \item Beräkna: $\lg 50 - \lg 2$
    
%    \item Lös ekvationen: $\lg x = 2,5$
    
%    \item Lös ekvationen: $\lg x + \lg (x-9) = 1$
    
%    \item Lös ekvationen: $2^x = 32$
    
%    \item Lös ekvationen: $3^{x-1} = 27$
%\end{enumerate}

\section{Enkla exponentialekvationer (lös med logaritmer)}

\begin{enumerate}[label=\textbf{\arabic*.}]
    \item Lös ekvationen: $2^x = 7$
    
    \item Lös ekvationen: $5^x = 20$
    
    \item Lös ekvationen: $10^{2x} = 50$
    
    \item Lös ekvationen: $3^{x+1} = 15$
    
    \item Lös ekvationen: $4 \cdot 2^x = 32$
\end{enumerate}

\end{document}
