\documentclass[a4paper,11pt]{article}
\usepackage[utf8]{inputenc}
\usepackage[T1]{fontenc}
\usepackage[swedish]{babel}
\usepackage{amsmath,amssymb,amsfonts}
\usepackage{graphicx}
\usepackage{enumitem}
\usepackage{geometry}
\geometry{margin=2.5cm}

\title{Repetitionsuppgifter -- Matematik 2b}
\author{Adam Damaj}
\date{\today}

\begin{document}

\maketitle

\section{Algebra och parentesmultiplikation}

\begin{enumerate}[label=\textbf{\arabic*.}]
    \item Förenkla uttrycket: $(3x + 2)(x - 4)$
    
    \item Utveckla och förenkla: $(2a - 5)(3a + 1)$
    
    \item Beräkna: $(x + 3)(x + 5) - (x - 2)(x + 1)$
    
    \item Förenkla: $2(3x - 4) + 5(2x + 1)$
    
    \item Utveckla och förenkla: $(5 - 2y)(5 + 2y)$
\end{enumerate}

\section{Konjugat och kvadreringsregler}

\begin{enumerate}[label=\textbf{\arabic*.}]
    \item Beräkna med hjälp av konjugatregeln: $(4 + \sqrt{3})(4 - \sqrt{3})$
    
    \item Använd första kvadreringsregeln för att utveckla: $(x + 5)^2$
    
    \item Använd andra kvadreringsregeln för att utveckla: $(2a - 3)^2$
    
    \item Förenkla med hjälp av konjugatregeln: $(3x + 2y)(3x - 2y)$
    
    \item Beräkna med hjälp av lämplig kvadreringsregel: $(x - \frac{1}{2})^2$
\end{enumerate}

\section{Enkla ekvationssystem}

\begin{enumerate}[label=\textbf{\arabic*.}]  
    \item Lös ekvationssystemet:
    \begin{align*}
    3x + 2y &= 7\\
    x - y &= 4
    \end{align*}
    
    \item Lös ekvationssystemet:
    \begin{align*}
    4x - 3y &= 10\\
    2x + y &= 8
    \end{align*}
    
    \item Lös ekvationssystemet:
    \begin{align*}
    x + 2y &= 5\\
    3x - y &= 4
    \end{align*}
\end{enumerate}

\section{Problemlösning med andragradsfunktioner}

\begin{enumerate}[label=\textbf{\arabic*.}]
    \item En boll kastas rakt uppåt från marken med en utgångshastighet på $20$ m/s. Bollens höjd $h$ (i meter) efter $t$ sekunder ges av funktionen $h(t) = 20t - 5t^2$. 
    \begin{enumerate}[label=\alph*)]
        \item När når bollen sin högsta höjd?
        \item Hur hög når bollen som högst?
        \item När träffar bollen marken igen?
    \end{enumerate}
    
    \item En rektangel har omkretsen $24$ cm. Låt $x$ vara rektangelns bredd.
    \begin{enumerate}[label=\alph*)]
        \item Uttryck rektangelns längd som en funktion av $x$.
        \item Uttryck rektangelns area $A$ som en funktion av $x$.
        \item Vilka värden kan $x$ anta?
        \item För vilket värde på $x$ blir arean maximal?
        \item Vad är den maximala arean?
    \end{enumerate}
\end{enumerate}


\section{Blandade uppgifter}

\begin{enumerate}[label=\textbf{\arabic*.}]
    \item Utveckla och förenkla: $(3x - 2)^2 - (x + 4)(x - 4)$
    
    \item Beräkna med hjälp av konjugatregeln: $(2\sqrt{5} + 3)(2\sqrt{5} - 3)$
    
    \item Lös ekvationssystemet:
    \begin{align*}
    2x - 3y &= -4\\
    5x + 2y &= 16
    \end{align*}
    
    \item En affär säljer två olika sorters kaffe. Det dyrare kaffet kostar 120 kr/kg och det billigare kostar 80 kr/kg. Affären vill blanda de två sorterna för att få 5 kg blandkaffe som ska säljas för 95 kr/kg.
    \begin{enumerate}[label=\alph*)]
        \item Ställ upp ett ekvationssystem där $x$ är antalet kg av det dyrare kaffet och $y$ är antalet kg av det billigare kaffet.
        \item Lös ekvationssystemet för att bestämma hur många kg av varje sort som behövs.
    \end{enumerate}
    
    \item En rektangel har arean 48 cm². Om längden ökas med 2 cm och bredden minskas med 1 cm, förblir arean oförändrad.
    \begin{enumerate}[label=\alph*)]
        \item Ställ upp ett ekvationssystem för att bestämma rektangelns ursprungliga dimensioner.
        \item Bestäm rektangelns ursprungliga längd och bredd.
    \end{enumerate}
    
    \item Utveckla och förenkla: $(a + b)^3 - (a - b)^3$
    
    \item Två personer, Alex och Billie, arbetar tillsammans för att färdigställa ett projekt. Alex kan göra hela projektet på 12 timmar, medan Billie behöver 8 timmar för att göra samma projekt ensam.
    \begin{enumerate}[label=\alph*)]
        \item Hur stor del av projektet hinner Alex göra på en timme?
        \item Hur stor del av projektet hinner Billie göra på en timme?
        \item Hur lång tid tar det för dem att göra projektet tillsammans?
    \end{enumerate}
\end{enumerate}

\end{document}