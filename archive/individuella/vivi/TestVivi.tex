\documentclass[a4paper,11pt]{article}
\usepackage[utf8]{inputenc}
\usepackage[T1]{fontenc}
\usepackage[swedish]{babel}
\usepackage{amsmath,amssymb,amsfonts}
\usepackage{enumitem}
\usepackage{geometry}
\geometry{margin=2.5cm}

\title{Test: Algebra och andragradsekvationer}
\author{Viktor Arohlén}
\date{\today}

\begin{document}

\maketitle

\section*{Uppgifter}
\begin{enumerate}[label=\textbf{\arabic*.}]
    % 1. Förenkling (addition av kvadreringsuttryck)
    \item Förenkla uttrycket: $(x+4)^2 + (x-2)^2$

    % 2. Andragradsekvation (blandad metod)
    \item Lös ekvationen: $x^2 - 4x = 0$

    % 3. Problemlösning (geometri)
    \item En rektangel har arean $35\text{ cm}^2$. Längden är $4\text{ cm}$ längre än bredden. Bestäm rektangelns dimensioner.

    % 4. Andragradsekvation (heltal pq-formel)
    \item Lös ekvationen: $3x^2 + 18 = 15x$

    % 5. Förenkling (lite svårare)
    \item Förenkla uttrycket: $(x+2)^2 - (x-2)^2$

    % 6. Problemlösning (andragradsekvation)
    \item Summan av två positiva tal är 13 och produkten är 40. Vilka är talen?

\end{enumerate}

\newpage
\section*{Facit}
\begin{enumerate}[label=\textbf{\arabic*.}]
    \item $(x+4)^2 + (x-2)^2 = (x^2+8x+16) + (x^2-4x+4) = 2x^2 + 4x + 20$
    \item $x^2 - 4x = 0 \Rightarrow x^2 - 4x = 0$\newline pq-formeln: $x^2 - 4x + 0 = 0$\newline $x = \frac{4}{2} \pm \sqrt{\left(\frac{4}{2}\right)^2 - 0} = 2 \pm 2$\newline Svar: $x=0$ eller $x=4$
    \item Låt bredden vara $x$. Då är längden $x+4$. $x(x+4)=35 \Rightarrow x^2+4x-35=0$\newline pq-formeln: $x^2 + 4x - 35 = 0 \Rightarrow x^2 + 4x = 35$\newline $x = -2 \pm \sqrt{(-2)^2 + 35} = -2 \pm \sqrt{4+35} = -2 \pm 6.08$\newline Eftersom det ska bli heltal, kontrollräkna: $x^2 + 4x - 35 = 0$\newline pq-formeln: $x = -\frac{4}{2} \pm \sqrt{\left(\frac{4}{2}\right)^2 + 35} = -2 \pm \sqrt{4+35} = -2 \pm 7$\newline Svar: $x=5$ (bredd), $x+4=9$ (längd)
    \item $3x^2 + 18 = 15x \Rightarrow 3x^2 - 15x + 18 = 0 \Rightarrow x^2 - 5x + 6 = 0$\newline pq-formeln: $x^2 - 5x + 6 = 0$\newline $x = \frac{5}{2} \pm \sqrt{\left(\frac{5}{2}\right)^2 - 6} = 2.5 \pm \sqrt{6.25 - 6} = 2.5 \pm 0.5$\newline Svar: $x=3$ eller $x=2$
    \item $(x+2)^2 - (x-2)^2 = (x^2+4x+4)-(x^2-4x+4)=8x$
    \item Låt talen vara $x$ och $13-x$. $x(13-x)=40 \Rightarrow x^2-13x+40=0$\newline pq-formeln: $x^2 - 13x + 40 = 0$\newline $x = \frac{13}{2} \pm \sqrt{\left(\frac{13}{2}\right)^2 - 40} = 6.5 \pm \sqrt{42.25 - 40} = 6.5 \pm 1.5$\newline Svar: $x=8$ och $x=5$ (talen är 8 och 5)
\end{enumerate}

\end{document}
