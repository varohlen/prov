\documentclass[a4paper,12pt]{article}
\usepackage[utf8]{inputenc}
\usepackage[T1]{fontenc}
\usepackage{geometry}
\geometry{margin=1in}
\usepackage{lmodern}

\title{Sammanfattning av Elevers Kommentarer}
\author{}
\date{\today}

\begin{document}

\maketitle

\section*{Inledning}
Denna sammanfattning är baserad på elevers kommentarer inför ett klassråd. All information har avanonymiserats och sammanställts med hjälp av en LLM (Large Language Model).

\section*{Trygghet}
\begin{itemize}
    \item De flesta elever känner sig trygga i skolan och har inga större klagomål.
    \item Några uttrycker en viss oro kring säkerheten, särskilt kopplat till skolans öppna miljö.
    \item \textit{"Stäng dörren mellan komvux och gymnasiet med tanke på händelsen i Örebro."}
    \item Önskemål om tydligare säkerhetsåtgärder nämns av enstaka elever.
\end{itemize}

\section*{Studiero}
\begin{itemize}
    \item Upplevelsen av studiero varierar beroende på var i skolan man befinner sig.
    \item Klassrumsmiljön beskrivs generellt som bra, men vissa upplever problem i gemensamma utrymmen.
    \item \textit{"Lite flummigt på gymnasietorget, annars är det bra."}
    \item Några efterfrågar mer regler för att säkerställa tystare studiemiljöer under vissa tider.
\end{itemize}

\section*{Bemötande}
\begin{itemize}
    \item Majoriteten av eleverna tycker att lärarna har ett bra bemötande.
    \item \textit{"Bra bemötande av alla lärare."}
    \item Bemötandet mellan elever kan vara varierande, där vissa situationer känns mindre inkluderande.
\end{itemize}

\section*{Delaktighet}
\begin{itemize}
    \item Vissa elever känner sig delaktiga i beslut och diskussioner i skolan, medan andra inte gör det.
    \item \textit{"Lite."}
    \item Det nämns att elevrådet gör ett bra jobb, men att fler borde engagera sig för att få större inflytande.
\end{itemize}

\section*{Undervisning}
\begin{itemize}
    \item Undervisningen anses generellt vara bra och lärarna får beröm för sin kompetens.
    \item \textit{"Nästan alltid bra undervisning eftersom vi har bra lärare."}
    \item Viss kritik mot otydliga instruktioner och upplägg i vissa ämnen.
    \item \textit{"Bra, ibland lite oklart. Bort med Motivationslyft."}
\end{itemize}

\section*{Rastmiljöer}
\begin{itemize}
    \item De flesta tycker att rastmiljöerna fungerar bra.
    \item \textit{"Bra men ibland finns det inte någon sittplats."}
    \item Önskemål om fler avskilda studieplatser för de som vill arbeta under rasterna.
\end{itemize}

\section*{Skolrestaurang}
\begin{itemize}
    \item Många är nöjda med skolmaten och uppskattar salladsbaren.
    \item \textit{"Bra med stor salladsbar, kan höja budgeten lite."}
    \item Flera anser att kvaliteten på maten har försämrats och att variationen kunde vara bättre.
    \item \textit{"Maten har blivit sämre."}
\end{itemize}

\section*{Övrigt}
\begin{itemize}
    \item Några kommentarer om skolans fysiska miljö, exempelvis trånga eller dåligt ventilerade lokaler.
    \item Önskemål om fler elevaktiviteter under rasterna.
\end{itemize}

\end{document}
