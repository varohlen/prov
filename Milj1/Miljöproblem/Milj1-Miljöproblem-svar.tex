\documentclass[a4paper]{article}
\usepackage[utf8]{inputenc}
\usepackage[swedish]{babel}
\usepackage{enumitem}
\usepackage{amsmath}
\usepackage[margin=2.5cm]{geometry}

\title{Facit - Miljöproblem och dess effekter}
\author{Viktor Arohlén}
\date{2024}

\begin{document}
\maketitle

\section*{Del 1: Flervalsfrågor (12 poäng)}

\begin{enumerate}
\item Växthuseffekten\\
\textbf{Rätt svar: C} - Den naturliga växthuseffekten håller jordens\\
medeltemperatur omkring 15°C\\
\textit{Förklaring:} Den naturliga växthuseffekten är nödvändig för liv på jorden\\
och håller planetens medeltemperatur på en beboelig nivå.

\item Jordens andetag\\
\textbf{Rätt svar: B} - Årstidsvariationer i koldioxidhalten på grund av\\
växternas fotosyntes\\
\textit{Förklaring:} Under växtsäsongen tar växter upp mer koldioxid genom\\
fotosyntes, vilket skapar tydliga årstidsvariationer i atmosfärens halt.

\item UV-strålning\\
\textbf{Rätt svar: B} - UV-B behövs för D-vitaminproduktion men kan orsaka\\
hudcancer\\
\textit{Förklaring:} UV-B har både positiva och negativa effekter; det är\\
nödvändigt för D-vitaminproduktion men kan också skada DNA.

\item Ozonskiktets funktion\\
\textbf{Rätt svar: B} - Det absorberar skadlig UV-strålning\\
\textit{Förklaring:} Ozonskiktet i stratosfären fungerar som ett filter som\\
absorberar skadlig UV-strålning, särskilt UV-B och UV-C.

\item Marknära ozon\\
\textbf{Rätt svar: C} - Det är en luftförorening som kan skada lungor och\\
växtlighet\\
\textit{Förklaring:} Marknära ozon är en sekundär luftförorening som bildas\\
genom fotokemiska reaktioner.

\item Partikelstorlek\\
\textbf{Rätt svar: B} - Ultrafina partiklar är farligast eftersom de kan ta sig\\
in i blodomloppet\\
\textit{Förklaring:} Ju mindre partiklarna är, desto djupare kan de tränga in\\
i kroppen.

\item Kväveoxider i stadsmiljö\\
\textbf{Rätt svar: B} - Vägtrafik\\
\textit{Förklaring:} I städer är fordonstrafiken den dominerande källan till\\
kväveoxidutsläpp.

\item Försurning\\
\textbf{Rätt svar: C} - Försurning kan leda till utlakning av metaller i\\
marken\\
\textit{Förklaring:} När pH-värdet sjunker ökar metallernas löslighet i\\
marken.

\break

\item Försurningsreaktion\\
\textbf{Rätt svar: B} - $\mathrm{2SO_2 + 2H_2O + O_2 \rightarrow 2H_2SO_4}$\\
\textit{Förklaring:} Denna reaktion visar bildningen av svavelsyra, en stark\\
syra som bidrar till försurning.

\item Åtgärder mot övergödning\\
\textbf{Rätt svar: B} - Minska näringsläckage från jordbruk\\
\textit{Förklaring:} Jordbruket är en huvudkälla till övergödning genom\\
läckage av kväve och fosfor.

\item Övergödda vattendrag\\
\textbf{Rätt svar: C} - Syrebrist i bottenvattnet\\
\textit{Förklaring:} När stora mängder alger dör och bryts ned förbrukas\\
syre.

\item Miljögifter\\
\textbf{Rätt svar: C} - Fettlösliga gifter kan ansamlas i näringskedjan\\
\textit{Förklaring:} Fettlösliga gifter lagras i fettvävnad och ansamlas i\\
högre koncentrationer uppåt i näringskedjan.

\break

\section*{Del 2: Fördjupande frågor (8 poäng)}

\subsection*{Fråga 13 (4 poäng)}
\textbf{Del 1 (2p) - Naturlig vs förstärkt växthuseffekt:}
\begin{itemize}
\item Naturlig växthuseffekt:
  \begin{itemize}
  \item Naturlig process där växthusgaser fångar värmestrålning
  \item Nödvändig för liv på jorden
  \item Håller medeltemperaturen omkring 15°C
  \end{itemize}
\item Förstärkt växthuseffekt:
  \begin{itemize}
  \item Orsakad av mänskliga aktiviteter
  \item Ökade halter av växthusgaser
  \item Leder till global uppvärmning
  \end{itemize}
\end{itemize}

\textbf{Del 2 (2p) - Etiska aspekter:}
\begin{itemize}
\item Historiskt ansvar:
  \begin{itemize}
  \item Industriländer har historiskt stått för största delen av utsläppen
  \item Utvecklingsländer har bidragit minimalt till problemet
  \end{itemize}
\item Utvecklingsrätt:
  \begin{itemize}
  \item Utvecklingsländers behov av ekonomisk tillväxt
  \item Behov av tekniköverföring och stöd
  \item Rättvis fördelning av utsläppsminskningar
  \end{itemize}
\end{itemize}

\subsection*{Fråga 14 (4 poäng)}
\textbf{Del 1 (2p) - Ozonets dubbla roll:}
\begin{itemize}
\item Stratosfäriskt ozon:
  \begin{itemize}
  \item Skyddar mot skadlig UV-strålning
  \item Bildas och bryts ned naturligt
  \item Nödvändigt för liv på land
  \end{itemize}
\item Marknära ozon:
  \begin{itemize}
  \item Bildas genom fotokemiska reaktioner
  \item Skadligt för hälsa och vegetation
  \item Förvärras av värme och solljus
  \end{itemize}
\end{itemize}

\textbf{Del 2 (2p) - Lärdomar från ozonarbetet:}
\begin{itemize}
\item Framgångsfaktorer:
  \begin{itemize}
  \item Internationellt samarbete möjligt
  \item Vetenskaplig konsensus ledde till handling
  \item Tekniska lösningar utvecklades
  \end{itemize}
\item Tillämpning på klimatfrågan:
  \begin{itemize}
  \item Betydelsen av globala avtal
  \item Vikten av teknisk utveckling
  \item Förebyggande åtgärder centrala
  \end{itemize}
\end{itemize}

\end{enumerate}

\end{document}
