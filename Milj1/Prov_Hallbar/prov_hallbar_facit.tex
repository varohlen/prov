\documentclass{exam}
\usepackage{graphicx}
\usepackage[utf8]{inputenc}
\usepackage[swedish]{babel}

%Format Header and footer
\pagestyle{headandfoot}
\header{}{\Large\textbf{FACIT: Prov: Hållbar utveckling: begrepp och reglering}}{}
\headrule
\footrule
\setlength{\columnsep}{0.25cm}
\footer{}{Sida \thepage}{}

\printanswers

\begin{document}

\section*{Frågor}

\begin{questions}

\question \textbf{Koldioxidekvivalent} är ett ... 
\begin{checkboxes}
    \CorrectChoice mått på växthusgaser angett i motsvarande mängd koldioxid
    \choice mått på hur mycket en växthusgas bidrar till global uppvärmning
    \choice mått på koldioxids effekt gällande växthuseffekten
    \choice mått som används för att mäta bilars miljöpåverkan
\end{checkboxes}

\vspace{5mm} \hrule \vspace{5mm}

\question Att vi \textbf{exploaterar naturresurser} innebär att ... 
\begin{checkboxes}
    \choice vi förstör naturen
    \CorrectChoice vi utnyttjar resurser från naturen
    \choice vi endast använder oss av icke-förnyelsebara resurser
    \choice vi endast använder oss av förnyelsebara resurser
\end{checkboxes}

\vspace{5mm} \hrule \vspace{5mm}

\question Att utnyttja resurser från biosfären kallas för /\textbf{ekosystemtjänster}. Dessa delas in i olika kategorier. \textbf{Fotosyntesen} är ett exempel på en ...
\begin{checkboxes}
    \CorrectChoice stödjande tjänst
    \choice försörjningstjänst
    \choice reglerande tjänst
    \choice kulturell tjänst
\end{checkboxes}

\vspace{5mm} \hrule \vspace{5mm}

\question Att utnyttja resurser från biosfären kallas för /\textbf{ekosystemtjänster}. Dessa delas in i olika kategorier. Att beundra ett vackert landskap är ett exempel på en ... 
\begin{checkboxes}
    \choice stödjande tjänst
    \choice försörjningstjänst
    \choice reglerande tjänst
    \CorrectChoice kulturell tjänst
\end{checkboxes}

\vspace{5mm} \hrule \vspace{5mm}

\question \textbf{Greenwashing} kan företag eller organisationer anklagas för, det innebär att ...
\begin{checkboxes}
    \choice att ha ett överdrivet fokus på hållbar utveckling
    \choice att utnyttja statliga subventioner för investeringar i grön teknik
    \CorrectChoice att använda falsk marknadsföring gällande organisationens arbete för hållbar utveckling
    \choice att använda färgen grönt överdrivet mycket
\end{checkboxes}

\vspace{5mm} \hrule \vspace{5mm}

\question I \textbf{avfallstrappan} ...
\begin{checkboxes}
    \choice Att vi måste prioritera alla steg lika mycket
    \CorrectChoice Att det översta alternativet är det mest resurssnåla och energieffektiva
    \choice Att vi i samhällskretsloppet främst vill utvinna så mycket energi som möjligt
    \choice Att vi bör jobba nerifrån och upp
\end{checkboxes}

\vspace{5mm} \hrule \vspace{5mm}

\question Varför används \textbf{globalhektar} för att mäta \textbf{ekologiska fotavtryck}?
\begin{checkboxes}
    \choice Det representerar alla länder
    \CorrectChoice Det är ett hektar med jordens genomsnittliga produktion och visar tydligt hur mycket en person/företag/land förbrukar
    \choice Det mäter hur mycket resurser vi har kvar på jorden
    \choice Det låter bra och viktigt
\end{checkboxes}

\vspace{5mm} \hrule \vspace{5mm}

\question Vad menas med virtuellt vatten?
\begin{checkboxes}
    \choice Allt vatten i Minecraft
    \CorrectChoice Den vattenförbrukning som inte syns i en färdig produkt gällande dess vattenavtryck
    \choice Det vatten som alltid finns i alla rörsystem och bidrar till vattenavtrycket
    \choice Vatten som inte kan förbrukas och måste tas hänsyn till när vi mäter vattenavtryck
\end{checkboxes}

\vspace{5mm} \hrule \vspace{5mm}

\question Vad innebär juridiska styrmedel? (1 poäng)
\begin{checkboxes}
    \choice Skatter, avgifter och subventioner
    \CorrectChoice Lagar och regler i ett samhälle
    \choice Straffskalan för olika miljöbrott
    \choice Det som står i Miljöbalken
\end{checkboxes}

\vspace{5mm} \hrule \vspace{5mm}

\question Samhällets styrmedel kan delas in i två kategorier: \textbf{juridiska} och \textbf{ekonomiska}. Nämn ett exempel från varje kategori och förklara hur det kan användas för att styra samhället mot en hållbar utveckling. (2 poäng)

\textbf{Svar:} \\
\textbf{Juridiskt styrmedel:} Ett exempel är \textit{Miljöbalken}, som innehåller lagar och regler för att skydda miljön. Den kan t.ex. sätta gränsvärden för utsläpp från industrier, vilket tvingar företag att investera i renare teknik för att inte bryta mot lagen. \\
\textbf{Ekonomiskt styrmedel:} Ett exempel är \textit{koldioxidskatten}. Genom att lägga en skatt på utsläpp av koldioxid blir det dyrare för företag och privatpersoner att använda fossila bränslen. Det skapar ett ekonomiskt incitament att minska sina utsläpp och välja mer klimatvänliga alternativ.

\vspace{5mm} \hrule \vspace{5mm}

\question Bilden är ett exempel på en kampanj som anklagats för att vara \textbf{"greenwashing"}. Vad är problemet med denna typ av reklam? (2 poäng)

\textbf{Svar:} Problemet med greenwashing är att det är vilseledande marknadsföring. Företag framställer sig som mer miljövänliga än de faktiskt är för att locka kunder. Det kan leda till att konsumenter tror att de gör ett hållbart val, när produkten eller företaget i själva verket har en stor negativ miljöpåverkan. Detta urholkar förtroendet för miljömärkningar och försvårar för konsumenter att göra genuint hållbara val.

\vspace{5mm} \hrule \vspace{5mm}

\question Vid exploatering av \textbf{icke-förnyelsebara resurser} används begreppet "peak". Utifrån grafen har vi nått \textbf{"peak-fosfor"}? Vad innebär det och vad medför det för problem? (3 poäng)

\textbf{Svar:} Att nå "peak-fosfor" innebär att vi har nått den maximala globala produktionstakten av fosfor från gruvor. Efter denna punkt kommer produktionen oundvikligen att minska, eftersom resursen blir allt svårare och dyrare att utvinna. \\
Problemet är att fosfor är en livsnödvändig och icke-förnybar resurs som är avgörande för jordbruket (används i konstgödsel). När produktionen minskar och priserna stiger kan det leda till matbrist, ökade matpriser och geopolitiska konflikter om de kvarvarande fosforreserverna. Det hotar den globala matsäkerheten.

\end{questions}

\end{document}
