\documentclass{exam}
\usepackage{graphicx}

%Format Header and footer
\pagestyle{headandfoot}
\header{\footnotesize Klass:\\Namn:}{\Large\textbf{Prov: Hållbar utveckling}}{\footnotesize HÅLMIJ0\\2025}
\headrule
\footrule
\setlength{\columnsep}{0.25cm}
\footer{}{Sida \thepage}{}

\begin{document}

\section*{Instruktioner}
Endast ett alternativ är korrekt på flervalsfrågor. Svara kortfattat på övriga frågor.

\subsection*{Poäng}
Varje flervalsfråga ger \textbf{1 poäng}. Frisvarsfrågorna har poäng angivna i frågan.

\section*{Del 1: Flervalsfrågor}

\begin{questions}

\question Vad är den korrekta definitionen av \textbf{hållbar utveckling}?
\vspace{2mm}
\begin{checkboxes}
    \choice En utveckling som maximerar ekonomisk tillväxt.
    \choice En utveckling som tillfredsställer dagens behov utan att äventyra kommande generationers möjligheter att tillfredsställa sina behov.
    \choice En utveckling som fokuserar enbart på att bevara den biologiska mångfalden.
    \choice En utveckling där alla länder har samma ekologiska fotavtryck.
\end{checkboxes}
\vspace{5mm} \hrule \vspace{5mm}

\question Vilken av följande är ett exempel på en \textbf{stödjande ekosystemtjänst}?
\vspace{2mm}
\begin{checkboxes}
    \choice Produktion av mat.
    \choice Rekreation i naturen.
    \choice Fotosyntesen.
    \choice Vattenrening.
\end{checkboxes}
\vspace{5mm} \hrule \vspace{5mm}

\question Vad mäter enheten \textbf{koldioxidekvivalent ($CO_2e$)}?
\vspace{2mm}
\begin{checkboxes}
    \choice Mängden koldioxid i atmosfären.
    \choice Ett mått på utsläpp av olika växthusgaser omräknat till koldioxidens uppvärmningseffekt.
    \choice Hur mycket en enskild växthusgas bidrar till den globala uppvärmningen.
    \choice Endast utsläpp från förbränning av fossila bränslen.
\end{checkboxes}
\vspace{5mm} \hrule \vspace{5mm}

\question Att ett företag använder vilseledande marknadsföring om sina miljöinsatser kallas för...
\vspace{2mm}
\begin{checkboxes}
    \choice Miljöcertifiering.
    \choice Ekologisk nisch.
    \choice Hållbarhetsredovisning.
    \choice Greenwashing.
\end{checkboxes}
\vspace{5mm} \hrule \vspace{5mm}
\break
\question Vad innebär begreppet \textbf{peak oil}?
\vspace{2mm}
\begin{checkboxes}
    \choice När priset på olja är som högst.
    \choice När vi har hittat all olja som finns på jorden.
    \choice Den tidpunkt då den maximala globala produktionshastigheten av olja är nådd.
    \choice När oljan helt har tagit slut.
\end{checkboxes}
\vspace{5mm} \hrule \vspace{5mm}

\question Vilket steg är det mest prioriterade (mest effektiva) i \textbf{avfallstrappan}?
\vspace{2mm}
\begin{checkboxes}
    \choice Återvinna material.
    \choice Utvinna energi.
    \choice Deponera.
    \choice Minimera avfall.
\end{checkboxes}
\vspace{5mm} \hrule \vspace{5mm}

\question Vad är ett exempel på ett \textbf{juridiskt styrmedel} för att främja hållbar utveckling?
\vspace{2mm}
\begin{checkboxes}
    \choice Subventioner för solceller.
    \choice En informationskampanj om sopsortering.
    \choice Miljöbalken.
    \choice Handel med utsläppsrätter.
\end{checkboxes}
\vspace{5mm} \hrule \vspace{5mm}

\question Vad menas med \textbf{virtuellt vatten}?
\vspace{2mm}
\begin{checkboxes}
    \choice Vatten som avdunstat från en produkt.
    \choice Den totala mängd vatten som förbrukats under en produkts hela livscykel.
    \choice Rent dricksvatten som säljs på flaska.
    \choice Vatten som används i virtuella simuleringar.
\end{checkboxes}
\vspace{5mm} \hrule \vspace{5mm}

\question De tre dimensionerna av hållbar utveckling är...
\vspace{2mm}
\begin{checkboxes}
    \choice Individ, Företag, Samhälle.
    \choice Global, Regional, Lokal.
    \choice Ekologisk, Social och Ekonomisk hållbarhet.
    \choice Produktion, Konsumtion, Återvinning.
\end{checkboxes}
\vspace{5mm} \hrule \vspace{5mm}
\break
\question Vad mäter \textbf{globalhektar (gha)}?
\vspace{2mm}
\begin{checkboxes}
    \choice Ett lands totala yta.
    \choice Ett hektar med jordens genomsnittliga biologiska produktivitet.
    \choice Den yta som krävs för att odla en specifik gröda.
    \choice Mängden skog som avverkas per år.
\end{checkboxes}
\vspace{5mm} \hrule \vspace{5mm}

\question Vilket av följande är ett exempel på en \textbf{kulturell ekosystemtjänst}?
\vspace{2mm}
\begin{checkboxes}
    \choice pollinering av grödor.
    \choice att beundra ett vackert landskap.
    \choice rening av luft och vatten.
    \choice tillgång till timmer.
\end{checkboxes}
\vspace{5mm} \hrule \vspace{5mm}

\question Vad innebär det konsumtionsbaserade perspektivet på utsläpp?
\vspace{2mm}
\begin{checkboxes}
    \choice Utsläpp som sker inom ett lands gränser.
    \choice Utsläpp kopplade till de varor och tjänster som konsumeras i ett land, oavsett var de producerats.
    \choice Endast utsläpp från privatpersoners konsumtion.
    \choice Utsläpp från den offentliga sektorn.
\end{checkboxes}
\vspace{5mm} \hrule \vspace{5mm}

\question Vad är syftet med \textbf{Agenda 2030} och de globala målen?
\vspace{2mm}
\begin{checkboxes}
    \choice Att skapa en global regering.
    \choice Att avskaffa all fattigdom, bekämpa ojämlikheter och stoppa klimatförändringarna.
    \choice Att enbart fokusera på ekonomisk utveckling i utvecklingsländer.
    \choice Att reglera internationell handel.
\end{checkboxes}
\vspace{5mm} \hrule \vspace{5mm}

\question Vad är ett exempel på \textbf{historiskt avfall}?
\vspace{2mm}
\begin{checkboxes}
    \choice Matavfall.
    \choice Plastförpackningar.
    \choice Radioaktivt avfall från kärnkraftverk.
    \choice Gamla tidningar.
\end{checkboxes}
\vspace{5mm} \hrule \vspace{5mm}
\break
\question Vad är ett exempel på ett \textbf{ekonomiskt styrmedel}?
\vspace{2mm}
\begin{checkboxes}
    \choice Förbud mot farliga kemikalier.
    \choice Skatt på koldioxidutsläpp.
    \choice Inrättande av nationalparker.
    \choice Krav på miljökonsekvensbeskrivning.
\end{checkboxes}
\vspace{5mm} \hrule \vspace{5mm}

\question Vad innebär \textbf{överexploatering} av en resurs?
\vspace{2mm}
\begin{checkboxes}
    \choice Att resursen används på ett effektivt sätt.
    \choice Att uttaget av resursen är större än dess återväxt.
    \choice Att resursen exporteras till andra länder.
    \choice Att resursen är jämnt fördelad globalt.
\end{checkboxes}
\vspace{5mm} \hrule \vspace{5mm}

\break

\section*{Del 2: Frisvarsfrågor}

\question Samhället använder både \textbf{juridiska} och \textbf{ekonomiska styrmedel} för att uppnå en hållbar utveckling. Ge ett konkret exempel på varje typ av styrmedel och förklara hur de fungerar för att styra samhället i en mer hållbar riktning. (3 poäng)

\vspace{80mm} 
\hrule 
\vspace{5mm}

\question \textbf{Avfallstrappan} är en modell som visar hur vi bör prioritera hanteringen av avfall. Rita och förklara de olika stegen i avfallstrappan. Resonera kring varför det översta steget är det mest eftersträvansvärda för ett hållbart samhälle. (3 poäng)



\end{questions}

\end{document}
