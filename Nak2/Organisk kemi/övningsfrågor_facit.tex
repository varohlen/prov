\documentclass{article}
\usepackage[utf8]{inputenc}
\usepackage[swedish]{babel}
\usepackage{graphicx}
\usepackage{enumitem}

\title{Övningsuppgifter: Organisk kemi (Facit)}
\author{Viktor Arohlén}
\date{\today}

\begin{document}

\maketitle

\section*{Sant eller falskt}

Markera om följande påståenden är \textbf{sanna} eller \textbf{falska}:

\begin{enumerate}[label=\arabic*.]
    \item Metan är uppbyggt av kol-, väte- och syreatomer.\\
    \textbf{Falskt}
    \fbox{\parbox{0.97\linewidth}{\textit{Kommentar: Metan (CH$_4$) består endast av kol och väte, inte syre.}}}

    \item Pentanmolekylen innehåller 3 kolatomer.\\
    \textbf{Falskt}
    \fbox{\parbox{0.97\linewidth}{\textit{Kommentar: Pentan har 5 kolatomer ("pent" = 5).}}}

    \item Metanol är en alkohol.\\
    \textbf{Sant}
    \fbox{\parbox{0.97\linewidth}{\textit{Kommentar: Metanol har en OH-grupp och räknas därför som en alkohol.}}}

    \item Kokpunkten för kolväten stiger ju kortare molekylerna blir.\\
    \textbf{Falskt}
    \fbox{\parbox{0.97\linewidth}{\textit{Kommentar: Längre kolkedjor ger högre kokpunkt.}}}

    \item Kolväten är opolära föreningar.\\
    \textbf{Sant}
    \fbox{\parbox{0.97\linewidth}{\textit{Kommentar: Kolväten består bara av kol och väte och är opolära.}}}

    \item Hydroxylgrupp (-OH) är en funktionell grupp.\\
    \textbf{Sant}
    \fbox{\parbox{0.97\linewidth}{\textit{Kommentar: Funktionella grupper ger ämnen deras speciella egenskaper.}}}

    \item Etanol är endast uppbyggd av kol-, och väteatomer.\\
    \textbf{Falskt}
    \fbox{\parbox{0.97\linewidth}{\textit{Kommentar: Etanol (C$_2$H$_5$OH) innehåller även syre.}}}

    \item Kolväten räknas inte till "fossila bränslen".\\
    \textbf{Falskt}
    \fbox{\parbox{0.97\linewidth}{\textit{Kommentar: Kolväten är huvudbeståndsdelen i fossila bränslen.}}}

    \item Etan är uppbyggt av fler kolatomer än väteatomer.\\
    \textbf{Falskt}
    \fbox{\parbox{0.97\linewidth}{\textit{Kommentar: Etan har 2 kol och 6 väte (C$_2$H$_6$).}}}

    \item Jäsning är en process som svampar använder för att utvinna energi ur glukos.\\
    \textbf{Sant}
    \fbox{\parbox{0.97\linewidth}{\textit{Kommentar: Jäsning är en anaerob process där svampar bryter ner glukos.}}}

    \item Biomolekyler måste inte innehålla kolatomer.\\
    \textbf{Falskt}
    \fbox{\parbox{0.97\linewidth}{\textit{Kommentar: Alla biomolekyler innehåller kol.}}}

    \item Lipider är opolära.\\
    \textbf{Sant}
    \fbox{\parbox{0.97\linewidth}{\textit{Kommentar: Lipider löser sig inte i vatten, vilket visar att de är opolära.}}}

    \item Polyeten är mänskligt framställd kemisk förening.\\
    \textbf{Sant}
    \fbox{\parbox{0.97\linewidth}{\textit{Kommentar: Polyeten är en syntetisk plast.}}}

    \item Plast har funnits i tusentals år.\\
    \textbf{Falskt}
    \fbox{\parbox{0.97\linewidth}{\textit{Kommentar: Plast har bara funnits i drygt 100 år.}}}

    \item Biobaserade plaster är alltid biologiskt nedbrytbara.\\
    \textbf{Falskt}
    \fbox{\parbox{0.97\linewidth}{\textit{Kommentar: Biobaserad betyder att plasten kommer från växter, men den kan ändå vara svår att bryta ner.}}}

    \item Metallar och organiska ämnen kan vara miljögiftiga.\\
    \textbf{Sant}
    \fbox{\parbox{0.97\linewidth}{\textit{Kommentar: Exempel är bly, kvicksilver och vissa bekämpningsmedel.}}}

    \item Förstahandskonsumenter har högre koncentration av giftiga ämnen än toppkonsumenter.\\
    \textbf{Falskt}
    \fbox{\parbox{0.97\linewidth}{\textit{Kommentar: Toppkonsumenter får i sig mest miljögifter genom biomagnifikation.}}}

    \item Cocktaileffekten försvårar forskning på miljögifter.\\
    \textbf{Sant}
    \fbox{\parbox{0.97\linewidth}{\textit{Kommentar: Cocktaileffekten innebär att flera ämnen tillsammans kan ge oväntade effekter.}}}
\end{enumerate}

\break

\section*{Kryssfrågor}

Svara på följande kryssfrågor genom att välja rätt/rätta alternativ (\textbf{facit markerat med fetstil}):

\begin{enumerate}[label=\arabic*.]
    \item Vad av följande är kolhydrater?
    \begin{itemize}
        \item \textbf{Sackaros} \\ \fbox{\parbox{0.9\linewidth}{\textit{Sackaros är en sockerart (disackarid), alltså en kolhydrat.}}}
        \item \textbf{Glukos} \\ \fbox{\parbox{0.9\linewidth}{\textit{Glukos är druvsocker, en enkel kolhydrat (monosackarid).}}}
        \item Omega-3
        \item Glukagon
    \end{itemize}

    \item Vad innebär ett omättat fett?
    \begin{itemize}
        \item Ett fett som inte innehåller dubbelbindningar.
        \item Ett fett som är polär.
        \item Ett fett som bara har enkelbindningar.
        \item \textbf{Ett fett som innehåller dubbel- eller trippelbindningar.} \\ \fbox{\parbox{0.9\linewidth}{\textit{Omättade fetter har minst en dubbel- eller trippelbindning i fettsyrakedjan.}}}
    \end{itemize}

    \item Hur är sackarider uppbyggda?
    \begin{itemize}
        \item Som långa kedjor
        \item Som fyrkanter
        \item \textbf{Som ringar} \\ \fbox{\parbox{0.9\linewidth}{\textit{Sackarider har ofta en ringstruktur i vattenlösning.}}}
        \item Som slingrande kedjor
    \end{itemize}

    \item Vad består proteiner av?
    \begin{itemize}
        \item Fettsyror
        \item Disackarider
        \item Vitaminer och mineraler
        \item \textbf{Aminosyror} \\ \fbox{\parbox{0.9\linewidth}{\textit{Proteiner byggs upp av aminosyror som länkas ihop i långa kedjor.}}}
    \end{itemize}
    \break

    \item Vad är kolhydraters funktion för en organism?
    \begin{itemize}
        \item \textbf{Lagra energi} \\ \fbox{\parbox{0.9\linewidth}{\textit{Kolhydrater lagras som glykogen (djur) eller stärkelse (växter) och används som energireserv.}}}
        \item Bygga upp viktiga organ
        \item Transportera andra ämnen
        \item Hjälpa immunförsvaret
    \end{itemize}



    \item Vad av följande är fördelar med plast?
    \begin{itemize}
        \item \textbf{Billigt} \\ \fbox{\parbox{0.9\linewidth}{\textit{Plast är billigt att tillverka jämfört med många andra material.}}}
        \item \textbf{Hållbart} \\ \fbox{\parbox{0.9\linewidth}{\textit{Plast håller länge och bryts inte ner lätt i naturen.}}}
        \item Miljövänligt
        \item Naturligt förekommande
    \end{itemize}

    \item Vad av följande är exempel på plast?
    \begin{itemize}
        \item \textbf{PVC} \\ \fbox{\parbox{0.9\linewidth}{\textit{PVC är en vanlig plast, används till exempel i rör och golv.}}}
        \item \textbf{PET} \\ \fbox{\parbox{0.9\linewidth}{\textit{PET används bland annat i läskflaskor och förpackningar.}}}
        \item \textbf{Akrylfibrer} \\ \fbox{\parbox{0.9\linewidth}{\textit{Akrylfibrer är syntetiska plastfibrer som används i textilier.}}}
        \item \textbf{Polyester} \\ \fbox{\parbox{0.9\linewidth}{\textit{Polyester är en vanlig plast i kläder och textilier.}}}
    \end{itemize}

    \item Vilka av följande är miljögiftiga metaller?
    \begin{itemize}
        \item \textbf{Kadmium} \\ \fbox{\parbox{0.9\linewidth}{\textit{Kadmium är en tungmetall som är giftig för både miljö och hälsa.}}}
        \item Kryll
        \item Guld
        \item \textbf{Bly} \\ \fbox{\parbox{0.9\linewidth}{\textit{Bly är också en tungmetall som är mycket giftig.}}}
    \end{itemize}
\end{enumerate}

\break

\section*{Frisvarsfrågor}

Svara på följande frågor:

\vspace{5mm} 
\begin{enumerate}[label=\arabic*.]

    \item Vad betyder \textbf{poly}? \\ 
    \fbox{\parbox{0.97\linewidth}{\textbf{Exempelsvar:}\textit{"Poly" betyder "många". I kemi används det för att beskriva molekyler som består av många upprepade delar, t.ex. polyeten.}}}

    \item Vad innebär \textbf{mikroplast} och \textbf{nanoplast}? \\ 
    \fbox{\parbox{0.97\linewidth}{\textbf{Exempelsvar:}\textit{Mikroplast är små plastbitar mindre än 5 mm. Nanoplast är ännu mindre, på nanometerskala. Båda kan tas upp av djur och spridas i naturen.}}}

    \item Förklara orden \textbf{bioackumulation} och \textbf{biomagnifikation}. \\ 
    \fbox{\parbox{0.97\linewidth}{\textbf{Exempelsvar:}\textit{Bioackumulation betyder att ett ämne samlas i en organism över tid. Biomagnifikation innebär att mängden av ämnet ökar ju högre upp i näringskedjan man kommer.}}}

    \item På vilket sätt har miljögifter minskat i naturen? \\ 
    \fbox{\parbox{0.97\linewidth}{\textbf{Exempelsvar:}\textit{Genom lagar och regler har användningen av miljögifter minskat, och reningsverk har blivit bättre på att ta bort dem.}}}
\end{enumerate}

\end{document}
