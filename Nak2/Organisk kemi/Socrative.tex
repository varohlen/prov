\documentclass{article}
\usepackage[utf8]{inputenc}
\usepackage[swedish]{babel}
\usepackage{graphicx}
\usepackage{enumitem}

\title{Övningsuppgifter: Organisk kemi}
\author{Viktor Arohlén}
\date{\today}

\begin{document}

\maketitle

\section*{Sant eller falskt}

Markera om följande påståenden är \textbf{sanna} eller \textbf{falska}:

\begin{enumerate}[label=\arabic*.]
    \item Metan är uppbyggt av kol-, väte- och syreatomer.
    \item Pentanmolekylen innehåller 3 kolatomer.
    \item Metanol är en alkohol.
    \item Kokpunkten för kolväten stiger ju kortare molekylerna blir.
    \item Kolväten är opolära föreningar.
    \item Hydroxylgrupp (-OH) är en funktionell grupp.
    \item Etanol är endast uppbyggd av kol-, och väteatomer.
    \item Kolväten räknas inte till "fossila bränslen".
    \item Etan är uppbyggt av fler kolatomer än väteatomer.
    \item Jäsning är en process som svampar använder för att utvinna energi ur glukos.
    \item Biomolekyler måste inte innehålla kolatomer.
    \item Lipider är opolära.
    \item Polyeten är mänskligt framställd kemisk förening.
    \item Plast har funnits i tusentals år.
    \item Biobaserade plaster är alltid biologiskt nedbrytbara.
    \item Metallar och organiska ämnen kan vara miljögiftiga.
    \item Förstahandskonsumenter har högre koncentration av giftiga ämnen än toppkonsumenter.
    \item Cocktaileffekten försvårar forskning på miljögifter.
\end{enumerate}

\break

\section*{Kryssfrågor}

Svara på följande kryssfrågor genom att välja rätt/rätta alternativ:

\begin{enumerate}[label=\arabic*.]
    \item Vad av följande är kolhydrater?
    \begin{itemize}
        \item Sackaros
        \item Glukos
        \item Omega-3
        \item Glukagon
    \end{itemize}

    \item Vad innebär ett omättat fett?
    \begin{itemize}
        \item Ett fett som inte innehåller dubbelbindningar.
        \item Ett fett som är polär.
        \item Ett fett som bara har enkelbindningar.
        \item Ett fett som innehåller dubbel- eller trippelbindningar.
    \end{itemize}

    \item Hur är sackarider uppbyggda?
    \begin{itemize}
        \item Som långa kedjor
        \item Som fyrkanter
        \item Som ringar
        \item Som slingrande kedjor
    \end{itemize}

    \item Vad består proteiner av?
    \begin{itemize}
        \item Fettsyror
        \item Disackarider
        \item Vitaminer och mineraler
        \item Aminosyror
    \end{itemize}

    \item Vad är kolhydraters funktion för en organism?
    \begin{itemize}
        \item Lagra energi
        \item Bygga upp viktiga organ
        \item Transportera andra ämnen
        \item Hjälpa immunförsvaret
    \end{itemize}

\break

    \item Vad av följande är fördelar med plast?
    \begin{itemize}
        \item Billigt
        \item Hållbart
        \item Miljövänligt
        \item Naturligt förekommande
    \end{itemize}

    \item Vad av följande är exempel på plast?
    \begin{itemize}
        \item PVC
        \item PET
        \item Akrylfibrer
        \item Polyester
    \end{itemize}

    \item Vilka av följande är miljögiftiga metaller?
    \begin{itemize}
        \item Kadmium
        \item Kryll
        \item Guld
        \item Bly
    \end{itemize}
\end{enumerate}

\break

\section*{Frisvarsfrågor}

Svara på följande frågor:

\vspace{5mm} 
\begin{enumerate}[label=\arabic*.]

    \item Vad betyder \textbf{poly}?
    \item Vad innebär \textbf{mikroplast} och \textbf{nanoplast}?
    \item Förklara orden \textbf{bioackumulation} och \textbf{biomagnifikation}.
    \item På vilket sätt har miljögifter minskat i naturen?

\end{enumerate}

\end{document}