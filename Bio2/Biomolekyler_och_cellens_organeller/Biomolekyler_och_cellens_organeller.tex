\documentclass{exam}
\usepackage{graphicx} 

% Format Header and footer
\pagestyle{headandfoot}
\header{\footnotesize Klass:\\Namn:}{\Large\textbf{Biomolekyler och cellens organeller}}{\footnotesize BIOBIO02 - 2024\\Viktor Arohlén}
\headrule
\footrule
\setlength{\columnsep}{0.25cm}
\footer{}{Sida \thepage}{}

\begin{document}
\section*{Instruktioner}
Provet består av två delar \\
    - Grundläggande frågor, svara kortfattat (\textit{10 poäng})\\
    - Fördjupande frågor, svara mer omfattande (\textit{10 poäng})

\subsection*{Poäng}
Antalet poäng är markerat för varje fråga. Totalt \textbf{12 frågor} och \textbf{20 poäng}.\\ \textit{För godkänt resultat krävs 7 poäng.}

\vspace{5mm} %5mm vertical space
\begin{center}
\fbox{\fbox{\parbox{6in}{\centering
\textbf{Grundläggande frågor}: svara kortfattat (\textbf{10 poäng})
}}}
\end{center}
\begin{questions}

\question Förklara följande begrepp (\textbf{2 poäng})

\begin{itemize}
  \item Peptidbindning
  \vspace{10mm}
  \item Polysackarid
  \vspace{10mm}
  \item Lipid
\end{itemize}

\vspace{10mm} %5mm vertical space

\question
Beskriv kortfattat varför människor inte kan bryta ner cellulosa. (\textbf{1 poäng})
\vspace{20mm}

\question Vilket av följande påståenden om kolesterol är sant? \textbf{1 poäng}
\vspace{5mm}
\begin{checkboxes}
    \choice Kolesterol är en fosfolipid som bygger upp cellmembranet
    \choice Kolesterol är en steroid som fungerar som förstadium till vissa hormoner
    \choice Kolesterol fungerar endast som energireserv i kroppen
    \choice Kolesterol är inte närvarande i djurceller
\end{checkboxes}


\vspace{10mm}
\question
Vilken molekyl fungerar som energilager hos djur? (\textbf{1 poäng})
\vspace{5mm}
\begin{checkboxes}
    \choice Cellulosa
    \choice Stärkelse
    \choice Glykogen
    \choice Laktos
\end{checkboxes}
\break


\question Vilket påstående beskriver bäst fosfolipiders roll i cellmembranet? (\textbf{1 poäng})
\vspace{5mm}
\begin{checkboxes}
    \choice De fungerar som enzymer som bryter ned näringsämnen.
    \choice De skapar en hydrofob barriär som separerar cellens inre från dess omgivning.
    \choice De hjälper till att transportera joner in och ut ur cellen.
    \choice De fungerar som energireserv för cellen.
\end{checkboxes}


\vspace{10mm}

\question Vad är den huvudsakliga rollen för det endoplasmatiska retikulumet (ER) i endomembransystemet? \\(\textbf{1 poäng})
\vspace{5mm}
\begin{checkboxes}
    \choice Energiomvandling
    \choice Produktion och transport av proteiner och lipider
    \choice Nedbrytning av avfallsprodukter
    \choice Lagring av genetisk information
\end{checkboxes}

\vspace{10mm}

\question Vilken cytoskelettstruktur hjälper till med muskelkontraktion och cellrörelse? (\textbf{1 poäng})
\vspace{5mm}
\begin{checkboxes}
    \choice Mikrotubuli
    \choice Aktinfilament
    \choice Intermediära filament
    \choice Centrosomer
\end{checkboxes}

\vspace{10mm}

\question
Beskriv kortfattat skillnaden mellan cilier och flageller. (\textbf{1 poäng})
\vspace{20mm}

\vspace{10mm}

\question
Vad har oraneller som ingår i endomembransystemet gemensamt? (\textbf{1 poäng})
\vspace{20mm}

\break



\vspace{5mm} %5mm vertical space
\begin{center}
\fbox{\fbox{\parbox{6in}{\centering
\textbf{Fördjupande frågor}: svara mer utförligt (\textbf{10 poäng})
}}}
\end{center}


\question
En person har svårt att äta mjölkprodukter utan att få magbesvär. Förklara vad som händer i kroppen hos en laktosintolerant person när de konsumerar laktos, och varför detta leder till symptom som uppblåsthet. (\textbf{3 poäng})
\vspace{60mm}

\question
Välj \textbf{två} av organellerna nedan och redogör för hur en cells och en organism funktion hade påverkats om de hade slutat fungera (\textbf{4 poäng}):
\begin{itemize}
  \item Peroxisom
  \item Lysosom
  \item Mikrotubuli
  \item Golgiapparaten
  \item Endoplasmatiska nätverket
  \item Intermediära filament
  \item Mikrofilament
\end{itemize}

\break
\question
Förklara hur en aminosyras primär-, sekundär-, tertiär- och kvartiärstruktur bidrar till proteinets funktion. Ge exempel på vad som kan hända om strukturen förändras. (\textbf{3 poäng})
\vspace{60mm}

\end{questions}

\end{document}
