\documentclass{exam}
\usepackage{graphicx}

% Format Header and footer
\pagestyle{headandfoot}
\header{\footnotesize Klass:\\Namn:}{\Large\textbf{Fysiologi III}\\\medskip\small Nervsystemet, rörelseapparaten och endokrina organsystemet}{\footnotesize BIOBIO02 - 2025}
\headrule
\footrule
\setlength{\columnsep}{0.25cm}
\footer{}{Sida \thepage}{}

\begin{document}

\section*{Instruktioner}
Provet består av två delar \\
    - Grundläggande frågor, svara kortfattat (\textit{14 poäng})\\
    - Fördjupande frågor, svara mer omfattande (\textit{10 poäng} + 2 bonuspoäng)

\subsection*{Poäng}
Antalet poäng är markerat för varje fråga. Totalt \textbf{12 frågor} och \textbf{24 poäng}.\\ \textit{För godkänt resultat krävs 10 poäng.}

\vspace{5mm} %5mm vertical space
\begin{center}
\fbox{\fbox{\parbox{6in}{\centering
\textbf{Grundläggande frågor}: svara kortfattat (\textbf{14 poäng})
}}}
\end{center}
\begin{questions}

% Omformulerad/fräsch variant på synapsfrågan
\question Rangordna stegen i signalöverföringen i en synaps från första till sista. (\textbf{2 poäng})

\begin{itemize}
  \item Aktionspotential når axonterminalen
  \item Vesiklar med neurotransmittorer fuserar med cellmembranet
  \item Neurotransmittorer frisätts
  \item Neurotransmittorer binder till receptorer
  \item Jonkanaler öppnas i postsynaptiska membranet
  \item Spänningsförändring i postsynaptiska cellen
\end{itemize}

\vspace{5mm}

% Omformulerad/fräsch variant på led-frågan
\question Beskriv två olika sorters leder i kroppen och ge exempel på var de finns. (\textbf{2 poäng})
\vspace{40mm}

% Omformulerad design på hormonfråga
\question Vilket av följande är \textbf{inte} ett hormon? (\textbf{1 poäng})
\begin{checkboxes}
    \choice Insulin
    \choice Acetylkolin
    \choice Kortisol
    \choice Tyroxin
\end{checkboxes}

\vspace{5mm}

% Omformulerad/fräsch variant på muskel-fråga
\question Vad menas med antagonistiska muskler? Ge exempel från människokroppen. (\textbf{2 poäng})
\vspace{25mm}

% Omformulerad/fräsch variant på nervsystemet
\question Kryssa för de delar som tillhör det perifera nervsystemet (PNS): (\textbf{1 poäng})
\begin{checkboxes}
    \choice Hjärnan
    \choice Ryggmärgen
    \choice Sensoriska nerver
    \choice Motoriska nerver
\end{checkboxes}

\vspace{5mm}

% Omformulerad/fräsch variant på reflexfråga
\question Vad är en reflexbåge? Beskriv kortfattat. (\textbf{1 poäng})
\vspace{20mm}

% Omformulerad/fräsch variant på signalsubstanser
\question Ge exempel på två olika signalsubstanser (neurotransmittorer) och deras funktion. (\textbf{2 poäng})
\vspace{20mm}

% Omformulerad/fräsch variant på hormonfråga
\question Vilken körtel producerar tillväxthormon? (\textbf{1 poäng})
\begin{checkboxes}
    \choice Sköldkörteln
    \choice Hypofysen
    \choice Binjuren
    \choice Bukspottkörteln
\end{checkboxes}

\vspace{5mm}

% Omformulerad/fräsch variant på homeostasfråga
\question Vad menas med homeostas? Ge ett exempel från människokroppen. (\textbf{2 poäng})
\vspace{20mm}

% Omformulerad/fräsch variant på rörelseapparaten
\question Vilken av följande ben hör till underarmen? (\textbf{1 poäng})
\begin{checkboxes}
    \choice Lårbenet
    \choice Strålbenet
    \choice Nyckelbenet
    \choice Skulderbladet
\end{checkboxes}

\vspace{5mm}

% Omformulerad/fräsch variant på endokrina systemet
\question Nämn två effekter av adrenalin på kroppen. (\textbf{2 poäng})
\vspace{20mm}

% Omformulerad/fräsch variant på fördjupande frågor
\break
\fbox{\fbox{\parbox{6in}{\centering
\textbf{Fördjupande frågor}: svara mer omfattande (\textbf{10 poäng} + 2 bonuspoäng)
}}}

\question Beskriv hur nervimpulser leds genom nervceller och hur signalen överförs mellan celler. (\textbf{3 poäng})
\vspace{40mm}

\question Resonera kring varför det är viktigt med ett fungerande endokrint system för kroppens hälsa. (\textbf{2 poäng})
\vspace{30mm}

\question Diskutera sambandet mellan fysisk aktivitet och nervsystemets/musklernas funktion. (\textbf{2 poäng})
\vspace{30mm}

\question BONUS: Beskriv en sjukdom eller skada som påverkar nervsystemet eller rörelseapparaten och hur den yttrar sig. (\textbf{2 bonuspoäng})
\vspace{30mm}

\end{questions}

\end{document}
