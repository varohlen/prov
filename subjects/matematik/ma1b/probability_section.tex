
\newpage
\section*{Sannolikhet}

\subsection*{Enkla slumpförsök}

\begin{enumerate}[label=\textbf{\arabic*.}]
    \item En vanlig tärning kastas en gång. Beräkna sannolikheten för att:
    \begin{enumerate}[label=\alph*)]
        \item Få en 6:a
        \item Få ett jämnt tal
        \item Få ett tal som är större än 4
    \end{enumerate}
    
    \item En kortlek med 52 kort innehåller 13 kort av varje färg (hjärter, ruter, klöver, spader). Beräkna sannolikheten för att dra:
    \begin{enumerate}[label=\alph*)]
        \item Ett hjärter
        \item Ett ess (det finns ett ess i varje färg)
        \item Ett svart kort (klöver och spader är svarta)
    \end{enumerate}
    
    \item I en urna finns 5 röda, 3 blå och 2 gröna kulor. En kula dras slumpmässigt. Beräkna sannolikheten för att:
    \begin{enumerate}[label=\alph*)]
        \item Kulan är röd
        \item Kulan är blå eller grön
        \item Kulan är varken röd eller blå
    \end{enumerate}
    
    \item I en klass med 30 elever är 18 flickor och 12 pojkar. Av flickorna har 6 glasögon och av pojkarna har 4 glasögon. En elev väljs slumpmässigt. Beräkna sannolikheten för att:
    \begin{enumerate}[label=\alph*)]
        \item Eleven är en flicka
        \item Eleven har glasögon
        \item Eleven är en pojke med glasögon
    \end{enumerate}
\end{enumerate}

\subsection*{Slumpförsök i flera steg}

\begin{enumerate}[label=\textbf{\arabic*.}]
    \item En vanlig tärning kastas två gånger. Beräkna sannolikheten för att:
    \begin{enumerate}[label=\alph*)]
        \item Få två 6:or
        \item Få summan 7
        \item Få minst en 6:a
    \end{enumerate}
    
    \item Två vanliga tärningar kastas samtidigt. Beräkna sannolikheten för att:
    \begin{enumerate}[label=\alph*)]
        \item Få samma tal på båda tärningarna
        \item Få summan 8
        \item Få en summa som är högst 4
    \end{enumerate}
    
    \item Från en kortlek med 52 kort dras två kort i följd utan återläggning. Beräkna sannolikheten för att:
    \begin{enumerate}[label=\alph*)]
        \item Båda korten är ess
        \item Första kortet är ett ess och andra kortet är en kung
        \item Båda korten är röda (hjärter och ruter är röda)
    \end{enumerate}
    
    \item I en urna finns 4 vita och 6 svarta kulor. Två kulor dras slumpmässigt utan återläggning. Beräkna sannolikheten för att:
    \begin{enumerate}[label=\alph*)]
        \item Båda kulorna är vita
        \item Båda kulorna är svarta
        \item En kula är vit och en kula är svart
    \end{enumerate}
\end{enumerate}
\end{enumerate}
