\documentclass[12pt]{article}
\usepackage[utf8]{inputenc}
\usepackage[swedish]{babel}
\usepackage{amsmath}
\usepackage{amssymb}
\usepackage{geometry}
\usepackage{graphicx}
\usepackage{enumitem}
\geometry{a4paper, margin=1in}

\begin{document}

\begin{center}
\Large\textbf{Övningsprov: Algebra och Funktioner} \\
\large Matematik 3b \\[0.5cm]
\normalsize
Namn: \underline{\hspace{6cm}} \quad Datum: \underline{\hspace{3cm}}
\end{center}

\vspace{0.5cm}

% E-NIVÅ UPPGIFTER (1 poäng vardera)

\noindent
\textbf{1. (1p)} \\
Bestäm graden och koefficienten för den högsta gradtermen i polynomet
\[
p(x) = 3x^4 - 2x^3 + 5x - 7
\]

\vspace{0.5cm}

\noindent
\textbf{2. (1p)} \\
Vilka av följande uttryck är polynom? Markera alla korrekta alternativ.
\begin{enumerate}[label=$\square$ \alph*)]
    \item $f(x) = 2x^3 - 5x + 7$
    \item $g(x) = \frac{1}{x} + x^2$
    \item $h(x) = \sqrt{x} + 3$
    \item $k(x) = 4x^5 - 2x^3 + x - 9$
    \item $m(x) = x^{-2} + 5x$
\end{enumerate}

\vspace{0.5cm}

\noindent
\textbf{3. (1p)} \\
Hur många nollställen kan ett femtegradspolynom maximalt ha? Motivera ditt svar.

\vspace{0.5cm}

\noindent
\textbf{4. (2p)} \\
Givet polynomet $p(x) = x^3 - 4x^2 + x + 6$.
\begin{enumerate}[label=\alph*)]
    \item Beräkna $p(2)$.
    \item Är $x = 2$ ett nollställe till $p(x)$? Motivera ditt svar.
\end{enumerate}

\vspace{0.5cm}

\noindent
\textbf{5. (1p)} \\
Polynomet $p(x) = (x - 2)(x + 3)(x - 5)$ är skrivet på faktoriserad form. Bestäm polynomets nollställen.

\vspace{0.5cm}

\noindent
\textbf{6. (1p)} \\
Förenkla uttrycket
\[
\frac{x^2 - 16}{x + 4}
\]

\vspace{0.5cm}

\noindent
\textbf{7. (1p)} \\
För vilket värde på $x$ är uttrycket $\frac{2x + 5}{x - 3}$ inte definierat?

\vspace{0.5cm}
\newpage
\noindent
\textbf{8. (1p)} \\
Förenkla uttrycket
\[
\frac{3x + 6}{x^2 + 2x}
\]

\vspace{0.5cm}

\noindent
\textbf{9. (1p)} \\
Bestäm gränsvärdet
\[
\lim_{x \to 3} (2x + 5)
\]

\vspace{0.5cm}

\noindent
\textbf{10. (1p)} \\
Bestäm gränsvärdet
\[
\lim_{x \to 2} \frac{x^2 - 4}{x - 2}
\]


\noindent
\textbf{11. (2p)} \\
Lös ekvationen $x^4 - 5x^2 + 4 = 0$

\vspace{0.5cm}

\noindent
\textbf{12. (2p)} \\
Beräkna
\[
\frac{2}{x + 1} + \frac{3}{x - 2}
\]
och skriv svaret som ett ende rationellt uttryck.

\vspace{0.5cm}

\noindent
\textbf{13. (2p)} \\
Beräkna
\[
\frac{x^2 - 9}{x + 3} : \frac{x - 3}{x + 1}
\]
och förenkla svaret.

\vspace{0.5cm}

\noindent
\textbf{14. (3p)} \\
Ett polynom $p(x)$ av tredje graden har nollställena $x = -2$, $x = 1$ och $x = 3$. Dessutom gäller att $p(0) = 12$.
\begin{enumerate}[label=\alph*)]
    \item Skriv polynomet på faktoriserad form.
    \item Bestäm polynomet på standardform.
\end{enumerate}

\vspace{0.5cm}

\noindent
\textbf{15. (2p)} \\
Lös ekvationen $2x^4 - 8x^2 = 0$ fullständigt.

\vspace{0.5cm}

\noindent
\textbf{16. (2p)} \\
Förenkla uttrycket
\[
\frac{x^2 - 4}{x^2 + 4x + 4} \cdot \frac{x + 2}{x - 2}
\]

\vspace{0.5cm}

\noindent
\textbf{17. (2p)} \\
Lös ekvationen
\[
\frac{3}{x - 1} = \frac{2}{x + 2}
\]

\vspace{0.5cm}

\noindent
\textbf{18. (2p)} \\
Bestäm gränsvärdet
\[
\lim_{x \to -1} \frac{x^2 + 3x + 2}{x + 1}
\]

\vspace{0.5cm}

\noindent
\textbf{19. (2p)} \\
Bestäm gränsvärdet
\[
\lim_{x \to \infty} \frac{3x^2 - 5x + 1}{x^2 + 2}
\]

\vspace{0.5cm}

\noindent
\textbf{20. (3p)} \\
Nedan visas grafen till ett fjärdegradspolynom $p(x)$.

\begin{center}
\includegraphics[width=0.7\textwidth]{../../../../resources/images/polynom_graf.png}
\end{center}

\begin{enumerate}[label=\alph*)]
    \item Bestäm polynomets nollställen grafiskt.
    \item Vad kan du säga om nollställenas karaktär (enkla eller dubbla)?
\end{enumerate}

\vspace{0.5cm}

\noindent
\textbf{21. (3p)} \\
Funktionen $f(x)$ är definierad som
\[
f(x) = \begin{cases}
x^2 + 1 & \text{om } x < 2 \\
5 & \text{om } x = 2 \\
3x - 1 & \text{om } x > 2
\end{cases}
\]

\begin{enumerate}[label=\alph*)]
    \item Bestäm vänstergränsvärdet $\lim_{x \to 2^-} f(x)$ och högergränsvärdet $\lim_{x \to 2^+} f(x)$.
    \item Existerar $\lim_{x \to 2} f(x)$? Motivera ditt svar.
    \item Är funktionen kontinuerlig i $x = 2$? Motivera ditt svar.
\end{enumerate}

\vspace{0.5cm}

\noindent
\textbf{22. (3p)} \\
Ett företag tillverkar muggar. Produktionskostnaden är 5000 kr i fasta kostnader plus 15 kr per mugg. Låt $x$ vara antalet tillverkade muggar.
\begin{enumerate}[label=\alph*)]
    \item Skriv ett uttryck för den totala kostnaden $K(x)$ för att tillverka $x$ muggar.
    \item Skriv ett uttryck för kostnaden per mugg som en funktion av $x$.
    \item Bestäm gränsvärdet när antalet muggar blir mycket stort. Vad betyder detta i praktiken?
\end{enumerate}

\vspace{0.5cm}


\noindent
\textbf{23. (3p)} \\
Bestäm konstanterna $A$ och $B$ så att likheten
\[
\frac{5x - 1}{x^2 - 4} = \frac{A}{x - 2} + \frac{B}{x + 2}
\]
gäller för alla $x$ där uttrycken är definierade.

\vspace{0.5cm}

\noindent
\textbf{24. (3p)} \\
Lös ekvationen
\[
\frac{x + 1}{x - 1} - \frac{x - 1}{x + 1} = \frac{8}{x^2 - 1}
\]

\vspace{0.5cm}

\noindent
\textbf{25. (3p)} \\
Bestäm gränsvärdet
\[
\lim_{h \to 0} \frac{(x + h)^2 - x^2}{h}
\]

\vspace{0.5cm}

\noindent
\textbf{26. (4p)} \\
Ett polynom $p(x)$ av fjärde graden har nollställena $x = -1$ (dubbel rot), $x = 2$ och $x = 4$. Polynomet går genom punkten $(0, 8)$.

\begin{center}
\includegraphics[width=0.6\textwidth]{../../../../resources/images/polynom_dubbel_rot.png}
\end{center}

\begin{enumerate}[label=\alph*)]
    \item Skriv polynomet på faktoriserad form.
    \item Bestäm polynomet på standardform.
\end{enumerate}

\vspace{0.5cm}

\newpage
\noindent
\textbf{27. (4p)} \\
En rektangulär trädgård ska inhägnas med 100 meter staket. Låt $x$ vara bredden på trädgården (i meter).

\begin{enumerate}[label=\alph*)]
    \item Uttryck trädgårdens längd som en funktion av $x$.
    \item Skriv ett uttryck för trädgårdens area $A(x)$ som en funktion av $x$.
    \item Bestäm $\lim_{x \to 0^+} A(x)$ och $\lim_{x \to 50^-} A(x)$. Vad betyder dessa gränsvärden i praktiken?
    \item Använd gränsvärdesanalys för att förklara varför arean inte kan bli oändligt stor.
\end{enumerate}

\vspace{1cm}

\begin{center}
\textbf{Totalt: 65 poäng}
\end{center}

\end{document}
