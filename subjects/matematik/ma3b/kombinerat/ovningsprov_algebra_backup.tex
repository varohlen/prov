\documentclass[12pt]{article}
\usepackage[utf8]{inputenc}
\usepackage[swedish]{babel}
\usepackage{amsmath}
\usepackage{amssymb}
\usepackage{geometry}
\usepackage{enumitem}
\usepackage{xcolor}
\usepackage{mdframed}
\geometry{a4paper, margin=1in}

\definecolor{facitbg}{RGB}{240,248,255}

\newmdenv[
  backgroundcolor=facitbg,
  linewidth=0pt,
  innertopmargin=10pt,
  innerbottommargin=10pt,
  innerrightmargin=10pt,
  innerleftmargin=10pt,
  skipabove=10pt,
  skipbelow=10pt
]{facitbox}

\title{Facit: Övningsprov Algebra och Funktioner \\ \large Matematik 3b}
\author{}
\date{}

\begin{document}

\maketitle

\section{Polynom och Polynomekvationer}

\subsection*{Uppgift 1 (E)}
\begin{facitbox}
Graden är 4 och koefficienten för högsta gradtermen är 3.
\end{facitbox}

\subsection*{Uppgift 2 (E)}
\begin{facitbox}
\begin{enumerate}[label=\alph*)]
    \item $p(2) = 2^3 - 4 \cdot 2^2 + 2 + 6 = 8 - 16 + 2 + 6 = 0$
    \item Ja, $x = 2$ är ett nollställe eftersom $p(2) = 0$.
\end{enumerate}
\end{facitbox}

\subsection*{Uppgift 3 (C)}
\begin{facitbox}
\begin{enumerate}[label=\alph*)]
    \item $p(x) = a(x + 2)(x - 1)(x - 3)$ där $a$ är en konstant.
    
    Använd villkoret $p(0) = 12$:
    \begin{align*}
    p(0) &= a(0 + 2)(0 - 1)(0 - 3) = a \cdot 2 \cdot (-1) \cdot (-3) = 6a \\
    6a &= 12 \\
    a &= 2
    \end{align*}
    
    Svar: $p(x) = 2(x + 2)(x - 1)(x - 3)$
    
    \item Multiplicera ut:
    \begin{align*}
    p(x) &= 2(x + 2)(x - 1)(x - 3) \\
    &= 2(x + 2)(x^2 - 4x + 3) \\
    &= 2(x^3 - 4x^2 + 3x + 2x^2 - 8x + 6) \\
    &= 2(x^3 - 2x^2 - 5x + 6) \\
    &= 2x^3 - 4x^2 - 10x + 12
    \end{align*}
\end{enumerate}
\end{facitbox}

\subsection*{Uppgift 4 (C)}
\begin{facitbox}
\begin{align*}
2x^4 - 8x^2 &= 0 \\
2x^2(x^2 - 4) &= 0 \\
2x^2(x - 2)(x + 2) &= 0
\end{align*}

Svar: $x = 0$ (dubbelrot), $x = 2$, $x = -2$
\end{facitbox}

\subsection*{Uppgift 5 (A)}
\begin{facitbox}
\begin{enumerate}[label=\alph*)]
    \item $p(x) = a(x + 1)^2(x - 2)(x - 4)$
    
    Använd $p(0) = 8$:
    \begin{align*}
    p(0) &= a(1)^2(-2)(-4) = 8a \\
    8a &= 8 \\
    a &= 1
    \end{align*}
    
    Svar: $p(x) = (x + 1)^2(x - 2)(x - 4)$
    
    \item Vid $x = -1$ (dubbel rot) tangerar grafen x-axeln. Vid $x = 2$ och $x = 4$ skär grafen x-axeln.
\end{enumerate}
\end{facitbox}

\section{Rationella Uttryck}

\subsection*{Uppgift 6 (E)}
\begin{facitbox}
\begin{align*}
\frac{x^2 - 16}{x + 4} &= \frac{(x - 4)(x + 4)}{x + 4} = x - 4
\end{align*}
\end{facitbox}

\subsection*{Uppgift 7 (E)}
\begin{facitbox}
Uttrycket är inte definierat när nämnaren är noll: $x - 3 = 0 \Rightarrow x = 3$
\end{facitbox}

\subsection*{Uppgift 8 (E)}
\begin{facitbox}
\begin{align*}
\frac{3x + 6}{x^2 + 2x} &= \frac{3(x + 2)}{x(x + 2)} = \frac{3}{x}
\end{align*}
\end{facitbox}

\subsection*{Uppgift 9 (C)}
\begin{facitbox}
\begin{align*}
\frac{2}{x + 1} + \frac{3}{x - 2} &= \frac{2(x - 2) + 3(x + 1)}{(x + 1)(x - 2)} \\
&= \frac{2x - 4 + 3x + 3}{(x + 1)(x - 2)} \\
&= \frac{5x - 1}{(x + 1)(x - 2)}
\end{align*}
\end{facitbox}

\subsection*{Uppgift 10 (C)}
\begin{facitbox}
\begin{align*}
\frac{x^2 - 4}{x^2 + 4x + 4} \cdot \frac{x + 2}{x - 2} &= \frac{(x - 2)(x + 2)}{(x + 2)^2} \cdot \frac{x + 2}{x - 2} \\
&= \frac{(x - 2)(x + 2)(x + 2)}{(x + 2)^2(x - 2)} \\
&= 1
\end{align*}
\end{facitbox}

\subsection*{Uppgift 11 (C)}
\begin{facitbox}
\begin{align*}
\frac{3}{x - 1} &= \frac{2}{x + 2} \\
3(x + 2) &= 2(x - 1) \\
3x + 6 &= 2x - 2 \\
x &= -8
\end{align*}

Kontroll: $\frac{3}{-8 - 1} = \frac{3}{-9} = -\frac{1}{3}$ och $\frac{2}{-8 + 2} = \frac{2}{-6} = -\frac{1}{3}$ \checkmark
\end{facitbox}

\subsection*{Uppgift 12 (A)}
\begin{facitbox}
\begin{align*}
\frac{5x - 1}{x^2 - 4} &= \frac{A}{x - 2} + \frac{B}{x + 2} \\
\frac{5x - 1}{(x - 2)(x + 2)} &= \frac{A(x + 2) + B(x - 2)}{(x - 2)(x + 2)}
\end{align*}

Därmed: $5x - 1 = A(x + 2) + B(x - 2)$

Sätt $x = 2$: $10 - 1 = A(4) \Rightarrow A = \frac{9}{4}$

Sätt $x = -2$: $-10 - 1 = B(-4) \Rightarrow B = \frac{11}{4}$

Svar: $A = \frac{9}{4}$, $B = \frac{11}{4}$
\end{facitbox}

\subsection*{Uppgift 13 (A)}
\begin{facitbox}
\begin{align*}
\frac{x + 1}{x - 1} - \frac{x - 1}{x + 1} &= \frac{8}{x^2 - 1} \\
\frac{(x + 1)^2 - (x - 1)^2}{(x - 1)(x + 1)} &= \frac{8}{(x - 1)(x + 1)} \\
(x + 1)^2 - (x - 1)^2 &= 8 \\
x^2 + 2x + 1 - (x^2 - 2x + 1) &= 8 \\
4x &= 8 \\
x &= 2
\end{align*}
\end{facitbox}

\section{Gränsvärden}

\subsection*{Uppgift 14 (E)}
\begin{facitbox}
\[
\lim_{x \to 3} (2x + 5) = 2 \cdot 3 + 5 = 11
\]
\end{facitbox}

\subsection*{Uppgift 15 (E)}
\begin{facitbox}
\begin{align*}
\lim_{x \to 2} \frac{x^2 - 4}{x - 2} &= \lim_{x \to 2} \frac{(x - 2)(x + 2)}{x - 2} \\
&= \lim_{x \to 2} (x + 2) = 4
\end{align*}
\end{facitbox}

\subsection*{Uppgift 16 (C)}
\begin{facitbox}
\begin{align*}
\lim_{x \to -1} \frac{x^2 + 3x + 2}{x + 1} &= \lim_{x \to -1} \frac{(x + 1)(x + 2)}{x + 1} \\
&= \lim_{x \to -1} (x + 2) = 1
\end{align*}
\end{facitbox}

\subsection*{Uppgift 17 (C)}
\begin{facitbox}
\begin{align*}
\lim_{x \to \infty} \frac{3x^2 - 5x + 1}{x^2 + 2} &= \lim_{x \to \infty} \frac{x^2(3 - \frac{5}{x} + \frac{1}{x^2})}{x^2(1 + \frac{2}{x^2})} \\
&= \lim_{x \to \infty} \frac{3 - \frac{5}{x} + \frac{1}{x^2}}{1 + \frac{2}{x^2}} = \frac{3}{1} = 3
\end{align*}
\end{facitbox}

\subsection*{Uppgift 18 (C)}
\begin{facitbox}
\begin{enumerate}[label=\alph*)]
    \item Vänstergränsvärdet: $\lim_{x \to 2^-} f(x) = 2^2 + 1 = 5$
    
    Högergränsvärdet: $\lim_{x \to 2^+} f(x) = 3 \cdot 2 - 1 = 5$
    
    \item Ja, $\lim_{x \to 2} f(x) = 5$ eftersom vänster- och högergränsvärdet är lika.
    
    \item Ja, funktionen är kontinuerlig i $x = 2$ eftersom $\lim_{x \to 2} f(x) = f(2) = 5$.
\end{enumerate}
\end{facitbox}

\subsection*{Uppgift 19 (A)}
\begin{facitbox}
\begin{align*}
\lim_{h \to 0} \frac{(x + h)^2 - x^2}{h} &= \lim_{h \to 0} \frac{x^2 + 2xh + h^2 - x^2}{h} \\
&= \lim_{h \to 0} \frac{2xh + h^2}{h} \\
&= \lim_{h \to 0} (2x + h) = 2x
\end{align*}
\end{facitbox}

\end{document}
