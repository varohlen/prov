\documentclass[12pt]{article}
\usepackage[utf8]{inputenc}
\usepackage[swedish]{babel}
\usepackage{amsmath}
\usepackage{amssymb}
\usepackage{geometry}
\usepackage{graphicx}
\usepackage{enumitem}
\geometry{a4paper, margin=1in}

\title{Övningsbank: Algebra och Funktioner \\ \large Matematik 3b}
\author{}
\date{}

\begin{document}

\maketitle

\section*{Instruktioner}
Denna övningsbank innehåller uppgifter inom följande områden:
\begin{itemize}
    \item Polynom och polynomekvationer
    \item Rationella uttryck
    \item Gränsvärden
\end{itemize}

Uppgifterna är märkta med svårighetsgrad (E, C, A) och är baserade på tidigare nationella prov samt det centrala innehållet i Matematik 3b enligt GY25.

\newpage

\section{Polynom och Polynomekvationer}

\subsection*{Uppgift P1 (E)}
Bestäm graden och koefficienten för den högsta gradtermen i polynomet
\[
p(x) = 3x^4 - 2x^3 + 5x - 7
\]

\subsection*{Uppgift P2 (E)}
Givet polynomet $p(x) = x^3 - 4x^2 + x + 6$.
\begin{enumerate}[label=\alph*)]
    \item Beräkna $p(2)$.
    \item Är $x = 2$ ett nollställe till $p(x)$? Motivera ditt svar.
\end{enumerate}

\subsection*{Uppgift P3 (E)}
Lös ekvationen $x^3 - 8 = 0$ genom att faktorisera med konjugatregeln.

\subsection*{Uppgift P4 (C)}
Lös polynomekvationen $x^3 + 2x^2 - 5x - 6 = 0$ fullständigt, givet att $x = -1$ är en rot.

\subsection*{Uppgift P5 (C)}
Ett polynom $p(x)$ av tredje graden har nollställena $x = -2$, $x = 1$ och $x = 3$. Dessutom gäller att $p(0) = 12$.
\begin{enumerate}[label=\alph*)]
    \item Skriv polynomet på faktoriserad form.
    \item Bestäm polynomet på standardform.
\end{enumerate}

\subsection*{Uppgift P6 (C)}
Polynomet $p(x) = x^3 + ax^2 + bx - 12$ har nollställena $x = 1$ och $x = -3$. Bestäm konstanterna $a$ och $b$.

\subsection*{Uppgift P7 (C)}
Lös ekvationen $2x^4 - 8x^2 = 0$ fullständigt.

\subsection*{Uppgift P8 (A)}
Visa att polynomet $p(x) = x^4 - 3x^3 + 2x^2 + 2x - 4$ är delbart med $x^2 - 3x + 2$. Utför sedan polynomdivisionen och ange kvoten.

\subsection*{Uppgift P9 (A)}
Ett polynom $p(x)$ av fjärde graden har nollställena $x = -1$ (dubbel rot), $x = 2$ och $x = 4$. Polynomet går genom punkten $(0, 8)$.
\begin{enumerate}[label=\alph*)]
    \item Bestäm polynomet.
    \item Beskriv polynomets beteende vid nollställena.
\end{enumerate}

\begin{center}
\includegraphics[width=0.7\textwidth]{../../../resources/images/polynom_dubbel_rot.png}
\end{center}

\newpage

\section{Rationella Uttryck}

\subsection*{Uppgift R1 (E)}
Förenkla uttrycket
\[
\frac{x^2 - 16}{x + 4}
\]

\subsection*{Uppgift R2 (E)}
För vilket värde på $x$ är uttrycket $\frac{2x + 5}{x - 3}$ inte definierat?

\subsection*{Uppgift R3 (E)}
Förenkla uttrycket
\[
\frac{3x + 6}{x^2 + 2x}
\]

\subsection*{Uppgift R4 (C)}
Förenkla uttrycket
\[
\frac{x^2 + 5x + 6}{x^2 - 9}
\]

\subsection*{Uppgift R5 (C)}
Beräkna
\[
\frac{2}{x + 1} + \frac{3}{x - 2}
\]
och skriv svaret som ett enda rationellt uttryck.

\subsection*{Uppgift R6 (C)}
Förenkla uttrycket
\[
\frac{x^2 - 4}{x^2 + 4x + 4} \cdot \frac{x + 2}{x - 2}
\]

\subsection*{Uppgift R7 (C)}
Lös ekvationen
\[
\frac{3}{x - 1} = \frac{2}{x + 2}
\]

\subsection*{Uppgift R8 (A)}
Bestäm konstanterna $A$ och $B$ så att likheten
\[
\frac{5x - 1}{x^2 - 4} = \frac{A}{x - 2} + \frac{B}{x + 2}
\]
gäller för alla $x$ där uttrycken är definierade.

\subsection*{Uppgift R9 (A)}
Förenkla uttrycket
\[
\frac{1}{x} - \frac{1}{x + h}
\]
och skriv svaret som ett enda rationellt uttryck.

\subsection*{Uppgift R10 (A)}
Lös ekvationen
\[
\frac{x + 1}{x - 1} - \frac{x - 1}{x + 1} = \frac{8}{x^2 - 1}
\]

\newpage

\section{Gränsvärden}

\subsection*{Uppgift G1 (E)}
Bestäm gränsvärdet
\[
\lim_{x \to 3} (2x + 5)
\]

\subsection*{Uppgift G2 (E)}
Bestäm gränsvärdet
\[
\lim_{x \to 2} \frac{x^2 - 4}{x - 2}
\]

\subsection*{Uppgift G3 (E)}
Bestäm gränsvärdet
\[
\lim_{x \to \infty} \frac{4x + 3}{2x - 1}
\]

\subsection*{Uppgift G4 (C)}
Bestäm gränsvärdet
\[
\lim_{x \to -1} \frac{x^2 + 3x + 2}{x + 1}
\]

\subsection*{Uppgift G5 (C)}
Bestäm gränsvärdet
\[
\lim_{x \to \infty} \frac{3x^2 - 5x + 1}{x^2 + 2}
\]

\subsection*{Uppgift G6 (C)}
Funktionen $f(x)$ är definierad som
\[
f(x) = \begin{cases}
x^2 + 1 & \text{om } x < 2 \\
5 & \text{om } x = 2 \\
3x - 1 & \text{om } x > 2
\end{cases}
\]

\begin{center}
\includegraphics[width=0.7\textwidth]{../../../resources/images/styckvis_funktion.png}
\end{center}

\begin{enumerate}[label=\alph*)]
    \item Bestäm $\lim_{x \to 2^-} f(x)$ och $\lim_{x \to 2^+} f(x)$.
    \item Existerar $\lim_{x \to 2} f(x)$? Motivera ditt svar.
    \item Är funktionen kontinuerlig i $x = 2$? Motivera ditt svar.
\end{enumerate}

\subsection*{Uppgift G7 (C)}
Bestäm gränsvärdet
\[
\lim_{x \to \infty} \frac{5x^3 + 2x}{2x^3 - x^2 + 1}
\]

\subsection*{Uppgift G8 (A)}
Bestäm gränsvärdet
\[
\lim_{x \to 1} \frac{x^3 - 1}{x^2 - 1}
\]

\subsection*{Uppgift G9 (A)}
En funktion $g(x)$ är definierad som
\[
g(x) = \frac{x^2 - 9}{x - 3}
\]
för $x \neq 3$.

\begin{center}
\includegraphics[width=0.7\textwidth]{../../../resources/images/hebbar_diskontinuitet.png}
\end{center}

\begin{enumerate}[label=\alph*)]
    \item Bestäm $\lim_{x \to 3} g(x)$.
    \item Kan funktionen göras kontinuerlig i $x = 3$ genom att definiera $g(3)$ på lämpligt sätt? I så fall, vilket värde ska $g(3)$ ha?
\end{enumerate}

\subsection*{Uppgift G10 (A)}
Bestäm gränsvärdet
\[
\lim_{h \to 0} \frac{(x + h)^2 - x^2}{h}
\]

\newpage

\section{Blandade Uppgifter och Problemlösning}

\subsection*{Uppgift B1 (C)}
En rektangel har ena sidan $x$ cm och den andra sidan $(10 - x)$ cm.
\begin{enumerate}[label=\alph*)]
    \item Skriv ett uttryck för rektangelns area $A(x)$.
    \item För vilket värde på $x$ blir arean maximal? (Lös algebraiskt utan digitala verktyg)
\end{enumerate}

\subsection*{Uppgift B2 (C)}
Summan av två tal är 20. Produkten av talen är 75. Vilka är talen? Ställ upp en ekvation och lös den.

\subsection*{Uppgift B3 (A)}
En öppen låda ska tillverkas genom att klippa bort kvadrater med sidan $x$ cm från varje hörn av en rektangulär plåt med måtten 20 cm × 30 cm, och sedan vika upp sidorna.
\begin{enumerate}[label=\alph*)]
    \item Skriv ett uttryck för lådans volym $V(x)$.
    \item Bestäm definitionsmängden för $V(x)$.
    \item För vilket värde på $x$ blir volymen maximal? (Använd digitala verktyg)
\end{enumerate}

\subsection*{Uppgift B4 (A)}
En parabel $y = ax^2 + bx + c$ går genom punkterna $(0, 3)$, $(1, 0)$ och $(3, 0)$.
\begin{enumerate}[label=\alph*)]
    \item Bestäm konstanterna $a$, $b$ och $c$.
    \item Bestäm parabelns vertex.
\end{enumerate}

\end{document}
