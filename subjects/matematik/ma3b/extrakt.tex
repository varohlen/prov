\documentclass{article}
\usepackage[T1]{fontenc}
\usepackage[utf8]{inputenc}
\usepackage{amsmath}
\usepackage{amssymb}
\usepackage{geometry}
\geometry{a4paper, margin=1in}

\title{Extrakter från Nationella Prov Matematik 3b: Polynom, Rationella Uttryck och Gränsvärde}
\author{Sammanställda uppgifter med källhänvisning}
\date{\today}

\begin{document}

\maketitle

\section{Polynom, Polynomekvationer och Egenskaper}

Dessa uppgifter rör polynom, derivering, extremvärden, terrasspunkter och lösning av polynomekvationer.

\subsection{Standarduppgifter (E/C-nivå)}
\begin{itemize}
    \item Lös ekvationen $0 = (x^2-1)(x-4)$. (Här är det troligt att ekvationen egentligen skulle vara $0 = (x^2-4)(x-1)$ baserat på centralt innehåll, men den angivna formuleringen används) (NP Ma3b HT12, uppg. 4)
    \item Derivera $f(x) = 10+6x+3x^4$. (NP Ma3b HT12, uppg. 5a)
    \item Bestäm $f'(x)$ då $f(x) = 3x^4 + 7x^2 + 3$. (NP Ma3b HT13, uppg. 1b)
    \item Vilken grad har funktionen $f$ då $f(x) = 3x^4 + 7x^2 + 3$? (NP Ma3b HT13, uppg. 1a)
    \item Lös ekvationen $0 = (x+2)(x-3)(x+4)$. (NP Ma3b HT13, uppg. 4)
    \item Bestäm $f'(x)$ då $f(x) = 4x^3 - 12x + 1$. (NP Ma3b VT14, uppg. 1)
    \item Lös ekvationen $0 = (x+1)(x-1)(x-3)$. (NP Ma3b VT14, uppg. 6)
    \item Bestäm extrempunkternas koordinater och karaktär för $f(x) = 4x^3 - 12x + 1$. (NP Ma3b VT14, uppg. 13)
    \item Bestäm $f'(x)$ då $f(x) = 4x^3 + 7x + 2$. (NP Ma3b HT14, uppg. 1a)
    \item Bestäm ett tal $A$ och ett tal $B$ så att tredjegradsekvationen $x(x+A)(x+B)=0$ får lösningarna $x_1=0$, $x_2=5$ och $x_3 = -7/3$. (NP Ma3b HT14, uppg. 6)
    \item Ange graden för polynomet $5x^5 + 4x^7 + 3x^3 - 8$. (NP Ma3b HT15, uppg. 1)
    \item Bestäm $f'(x)$ då $f(x) = 4x^3 + 5x^2 - 3$. (NP Ma3b VT15, uppg. 5a)
    \item Bestäm $f'(x)$ då $f(x) = 2.5x^5 + 4x^2$. (NP Ma3b VT16, uppg. 5a)
    \item För funktionen $f$ gäller att $f(x) = 3x^2 - 6x - 10$. Bestäm $f'(x)$. (NP Ma3b VT17, uppg. 3a)
    \item För funktionen $f$ gäller att $f(x) = 3x^2 - 6x - 10$. Bestäm $f'(1)$. (NP Ma3b VT17, uppg. 3b)
    \item För funktionen $f(x) = 3x^2 - 6x - 10$: Ett av alternativen A–D är korrekt. (T.ex. Har grafen till funktionen en terrasspunkt?) (NP Ma3b VT17, uppg. 3c)
    \item Bestäm $f'(x)$ då $f(x) = 4x^3 - 12x$. (NP Ma3b VT22, uppg. 6a)
    \item Lös ekvationen $3x^4 = 8x^2 - 4$. (Bikvadratisk ekvation) (NP Ma3b VT22, uppg. 7)
    \item Funktionen $f$ ges av $f(x) = x^3 - 3x^2 + 7$. Använd derivata och bestäm koordinaterna för eventuella maximi-, minimi- och terrasspunkter. (NP Ma3b VT22, uppg. 14)
    \item Har funktionen $f(x) = x^3 + 3x$ terrasspunkt? Motivera. (NP Ma3b VT22, uppg. 16)
\end{itemize}

\subsection{Avancerade uppgifter och resonemang (A-nivå)}
\begin{itemize}
    \item För polynomfunktionen $f$ gäller att $f'(x) > 0$ för alla $x$. Undersök hur många reella lösningar ekvationen $f(x) = 0$ har. (Resonemang, strängt växande funktion) (NP Ma3b HT13, uppg. 22)
    \item För funktionen $f$ gäller att $f'(2)=-1$ och $f''(4)=0$. Bestäm $f'(6)$. (Resonemang om derivatans symmetri) (NP Ma3b HT12, uppg. 24)
    \item Funktionen $f(x) = x^3 + 3x + 10$. Undersök om den har en terrasspunkt. (NP Ma3b HT15, uppg. 22)
    \item För funktionen $f$ gäller att $f'(x) = 3x^2 - 6x + 3$. Undersök hur många reella lösningar ekvationen $f(x)=0$ har. (NP Ma3b VT17, uppg. 17)
    \item För funktionen $f$ gäller att $f(x) = 2.9x^3 + kx^2 + kx$ där $k>0$. Grafen till funktionen har en terrasspunkt för ett visst värde på $k$. Bestäm detta värde på $k$. (NP Ma3b VT16, uppg. 25)
    \item Visa att oavsett vilket värde $a$ har, är tangenternas riktningskoefficienter i punkterna $(a, f(a))$ och $(-a, f(-a))$ lika stora för funktionen $f(x) = kx^3$. (NP Ma3b VT17, uppg. 25)
    \item Bevisa att tangenten till kurvan $y=1/x$ har arean 2 areaenheter oavsett var tangenten tangerar kurvan. (Involverar rationellt uttryck och generell algebra) (NP Ma3b VT14, uppg. 16)
    \item Visa att tangenterna till graferna i punkterna $(a, f(a))$ och $(3a, g(3a))$ är parallella oavsett värde på $a$ då $f(x) = 4x^3$ och $g(x) = 12x$. (NP Ma3b VT22, uppg. 18)
\end{itemize}

\section{Rationella Uttryck och Algebraisk Hantering}

Dessa uppgifter fokuserar på hantering och förenkling av rationella uttryck samt förståelse för deras definitionsmängd.

\begin{itemize}
    \item Förenkla $\frac{(x+3)^5}{(x+3)^{10}}$ så långt som möjligt. (NP Ma3b VT14, uppg. 4a)
    \item Förenkla $\frac{a^2-1}{a^2+a}$ så långt som möjligt. (NP Ma3b VT14, uppg. 4b)
    \item Förenkla $x(x+7)(x-7)+3$ så långt som möjligt. (NP Ma3b HT14, uppg. 5a)
    \item Förenkla $\frac{1}{x+1} - 1$ så långt som möjligt. (NP Ma3b HT14, uppg. 5b)
    \item Förenkla $\frac{x^3+6x}{x^3-5x}$ så långt som möjligt. (NP Ma3b VT22, uppg. 8a)
    \item Förenkla $\frac{x^2+12x+18}{2(x^2-9)}$ så långt som möjligt. (NP Ma3b VT22, uppg. 8b)
    \item För vilket värde på $x$ är uttrycket $\frac{10x+2}{6x-3}$ inte definierat? (NP Ma3b HT15, uppg. 3)
    \item Bestäm konstanten $a$ så att $\frac{x^2-a}{(x-1)(x+3)}$ kan förkortas. (NP Ma3b VT15, uppg. 15)
    \item Förenkla $\frac{(x+4)^3}{(x+4)^4}$ (Förenkling av rationellt uttryck) (NP Ma3b HT13, uppg. 5b)
    \item Ge ett exempel på ett rationellt uttryck som: får värdet 0 då $x=-1$; är inte definierat för $x=3$; är inte definierat för $x=-4$. (NP Ma3b VT13, uppg. 9b)
    \item Lös ekvationen $\frac{2}{x-1} = \frac{2x}{x^2-1}$. (Rationell ekvation) (NP Ma3b VT14, uppg. 14)
    \item För funktionen $f(x) = \frac{x-1}{x-6}$: Har Sofia rätt att största värdet nås vid $x=6$? Motivera. (Resonemang om definitionsmängd) (NP Ma3b HT14, uppg. 19a)
\end{itemize}

\section{Gränsvärde och Kontinuerliga Funktioner}

Dessa uppgifter inkluderar gränsvärdesberäkningar (asymptoter och definition) samt resonemang kring diskreta och kontinuerliga funktioner.

\begin{itemize}
    \item Vilket av alternativen A-D beskrivs bäst med en diskret funktion? (NP Ma3b HT12, uppg. 6)
    \item Bestäm derivatan till $f(x) = A/x$ med hjälp av derivatans definition. (Involverar gränsvärde) (NP Ma3b HT12, uppg. 16)
    \item Förklara varför integralen $\int_0^6 100x^2 dx$ ger ett för litet värde när den används för att räkna ut hur mycket pengar som finns i en burk på Marios 6-årsdag. (Jämförelse diskret/kontinuerlig) (NP Ma3b HT12, uppg. 25)
    \item Rita i koordinatsystemet en skiss som visar hur grafen till funktionen $f$ kan se ut om den är **kontinuerlig**, går genom $(1, 3), (3, 3), (5, 3)$ och $f'(1)>0, f'(3)<0, f'(5)>0$. \textbf{[KOORDINATSYSTEM]} (NP Ma3b VT13, uppg. 4)
    \item Bestäm den övre gränsen för antalet fiskar ($N$) i populationsmodellen $N(t) = 3000 + \frac{15000}{e^{0.5t}}$. (Gränsvärde $t \to \infty$) (NP Ma3b VT13, uppg. 10b)
    \item Bestäm $S'(4)$ då $S$ är en **kontinuerlig funktion** och $S(x+h) = S(x) + hx$. (Derivatans definition) (NP Ma3b VT13, uppg. 24)
    \item Bestäm konstanten $A$ så att $\lim_{x \to \infty} \frac{A x^2 + 4 x}{7 x^2 + 5} = 4/7$. (NP Ma3b HT13, uppg. 15)
    \item Enligt modellen $N(t) = \frac{11}{1+3.4e^{-0.03t}}$ kommer antalet människor att närma sig en övre gräns. Bestäm denna övre gräns. (Gränsvärde $t \to \infty$) (NP Ma3b HT13, uppg. 20b)
    \item Vilka två figurer A-F visar en graf till en diskret funktion? Vilka två visar en graf till en funktion som är **kontinuerlig** för alla $x$? \textbf{[FIGURER]} (NP Ma3b HT14, uppg. 3a), (NP Ma3b HT14, uppg. 3b)
    \item Vad måste gälla för att linjen $y=f(x)$ ska tangera kurvan $y=g(x)$ i den punkt där $x=a$? (Villkor för tangering/Kontinuitet och derivata) (NP Ma3b HT14, uppg. 15)
    \item Bestäm $\lim_{h \to 0} \frac{3^h - 1}{h}$ och svara exakt. (Gränsvärde relaterat till derivatan av $3^x$) (NP Ma3b VT15, uppg. 10)
    \item Bestäm den undre gränsen för vattentemperaturen enligt modellen $T(x) = 17e^{-0.693x} + 5$. (Gränsvärde $x \to \infty$) (NP Ma3b VT15, uppg. 19d)
    \item Bestäm derivatan till $f(x) = 1/(\sqrt{ax})$ med hjälp av derivatans definition. (Involverar gränsvärde) (NP Ma3b VT16, uppg. 17)
    \item Bestäm derivatan till $f(x) = 5/x^2 - a/x^2$ med hjälp av derivatans definition. (NP Ma3b VT22, uppg. 17)
\end{itemize}

\section{Problemlösning och Digitala Verktyg}

Dessa uppgifter kräver modellering, optimering av funktioner (inklusive polynom) och/eller användning av digitala verktyg för att lösa uppgifterna.

\begin{itemize}
    \item **Optimering (Polynom):** Rätblockets maximala volym. Beräkna största möjliga volym med derivata, där sidorna är $3x$, $(6-x)$ och $(6-x)$. \textbf{[FIGUR]} (NP Ma3b HT12, muntlig uppg. 1)
    \item **Linjär Optimering:** Sture (pallar $x$, byråer $y$). Max vinst under tidsbegränsningar. \textbf{[OLIKHETER]} (NP Ma3b HT12, uppg. 20b)
    \item **Optimering (Differensfunktion):** Bestäm det kortaste avståndet i $y$-led mellan graferna till $f(x) = 2x^3 - 6x^2 + 1$ och $g(x) = 7.5x - 1.8$ då $x>0$. (NP Ma3b HT13, muntlig uppg. 1)
    \item **Optimering (Modellering):** Karin ska bygga rektangulära rastgårdar. Bestäm $x$ så att arean blir maximal. \textbf{[FIGUR]} (NP Ma3b HT13, uppg. 12)
    \item **Optimering (Derivata av $V'$):** Vilka värden kan hastigheten $V'(t)$ anta under Albins åtta första levnadsår, där $V(t)$ är en tredjegradsfunktion? (NP Ma3b HT13, uppg. 23)
    \item **Linjär Optimering (A-nivå):** Skräddaren (kostymer/jackor) maximerar vinst $V=300x+250y$ under tygvillkor. (NP Ma3b HT13, uppg. 24)
    \item **Linjär Optimering:** Bagare (Hurtig/Nyttig) maximerar vinst under begränsningar. (NP Ma3b VT14, uppg. 23)
    \item **Optimering (Area/Modeling):** Glasmästaren. Beräkna det mått på bredden som ger spegelns största area, där spegeln har ett hörn på en avskuren kant. \textbf{[FIGUR]} (NP Ma3b VT14, uppg. 25)
    \item **Ekvationslösning (Digitalt verktyg):** Lös ekvationen $4x^3 - 17x = 5$. (NP Ma3b HT14, uppg. 22)
    \item **Optimering (Inkomst):** Beräkna med hjälp av derivata vilken prishöjning $x$ som ger den största dagsinkomsten $f(x) = -1.05x^2 + 5x + 3000$. (NP Ma3b HT14, uppg. 12)
    \item **Linjär Optimering:** Ellen och David (Tvålar). Bestäm antal av varje sort för att tjäna så mycket pengar som möjligt $V=15x+10y$. (NP Ma3b HT15, uppg. 23)
    \item **Optimering (Låda):** Bestäm sidlängden $x$ cm för den låda som har största volymen, given kartongmått ($60 \times 30$ cm). \textbf{[FIGUR]} (NP Ma3b VT13, muntlig uppg. 2)
    \item **Linjär Optimering:** Bestäm det största och det minsta värde som funktionen $V = 500x - 200y$ kan anta under givna olikhetsvillkor. (NP Ma3b VT16, uppg. 24)
    \item **Optimering (Area/Modeling):** Bestäm bredden $x$ på de rektangulära tygservetterna så att arean blir så stor som möjligt, givet kurvan $y = -0.5x^3 + 3x + 3$. (NP Ma3b VT16, uppg. 24)
    \item **Linjär Optimering:** Julius och Sophia (Sittkuddar). Maximera vinst $V=500x+400y$. (NP Ma3b VT22, uppg. 24)
    \item **Digitalt verktyg:** Använd ditt digitala verktyg för att beräkna ett värde på $f'(2)$ för $f(x) = (2x-1)^5$. (NP Ma3b VT22, uppg. 19)
    \item **Area (Integration/Modeling):** Bestäm $a$ så att arean av det streckade området blir lika stort som arean av det gråmarkerade området ($y=e^x$). \textbf{[FIGUR]} (NP Ma3b VT17, uppg. 27)
\end{itemize}

\end{document}
```