\documentclass[12pt]{article}
\usepackage[utf8]{inputenc}
\usepackage[swedish]{babel}
\usepackage{amsmath}
\usepackage{amssymb}
\usepackage{geometry}
\usepackage{graphicx}
\usepackage{enumitem}
\usepackage{xcolor}
\usepackage{framed}
\usepackage{mdframed}
\geometry{a4paper, margin=1in}

\definecolor{facitbg}{RGB}{240,248,255}
\definecolor{refbg}{RGB}{255,250,240}

% Definiera färgade boxar
\newmdenv[
  backgroundcolor=facitbg,
  linewidth=0pt,
  innertopmargin=10pt,
  innerbottommargin=10pt,
  innerrightmargin=10pt,
  innerleftmargin=10pt,
  skipabove=10pt,
  skipbelow=10pt
]{facitbox}

\newmdenv[
  backgroundcolor=refbg,
  linewidth=0pt,
  innertopmargin=10pt,
  innerbottommargin=10pt,
  innerrightmargin=10pt,
  innerleftmargin=10pt,
  skipabove=10pt,
  skipbelow=10pt
]{refbox}

\title{Komplett Övningsbank: Algebra och Funktioner \\ \large Matematik 3b \\ \small Uppgifter, Facit och Bedömning}
\author{}
\date{}

\begin{document}

\maketitle

\tableofcontents
\newpage

\section*{Instruktioner}
Detta dokument innehåller:
\begin{itemize}
    \item \textbf{Uppgifter} inom polynom, rationella uttryck och gränsvärden
    \item \textbf{Facit} med fullständiga lösningar
    \item \textbf{Bedömningskommentarer} med koppling till centralt innehåll och kunskapskrav (GY25)
\end{itemize}

Uppgifterna är märkta med svårighetsgrad (E, C, A) och är baserade på tidigare nationella prov.

\newpage

\section{Polynom och Polynomekvationer}

\subsection*{Uppgift P1 (E)}
Bestäm graden och koefficienten för den högsta gradtermen i polynomet
\[
p(x) = 3x^4 - 2x^3 + 5x - 7
\]

\begin{facitbox}
\textbf{Facit:}

Graden är 4 och koefficienten för högsta gradtermen är 3.
\end{facitbox}

\begin{refbox}
\textbf{Bedömning:}

\textbf{Centralt innehåll:} Begreppet polynom och egenskaper hos polynomfunktioner

\textbf{Inspiration:} Grundläggande begreppsförståelse som förekommer i alla NP (t.ex. Ma3b-vt22, Ma3b-ht16)

\textbf{Bedömning (E):}
\begin{itemize}
    \item Eleven kan identifiera grad och koefficient för högsta gradtermen
    \item Visar grundläggande förståelse för polynomstruktur
\end{itemize}

\textbf{Kunskapskrav:} "Eleven beskriver grundläggande begrepp och samband mellan begrepp"
\end{refbox}

\subsection*{Uppgift P2 (E)}
Givet polynomet $p(x) = x^3 - 4x^2 + x + 6$.
\begin{enumerate}[label=\alph*)]
    \item Beräkna $p(2)$.
    \item Är $x = 2$ ett nollställe till $p(x)$? Motivera ditt svar.
\end{enumerate}

\begin{facitbox}
\textbf{Facit:}

\begin{enumerate}[label=\alph*)]
    \item $p(2) = 2^3 - 4 \cdot 2^2 + 2 + 6 = 8 - 16 + 2 + 6 = 0$
    \item Ja, $x = 2$ är ett nollställe eftersom $p(2) = 0$.
\end{enumerate}
\end{facitbox}

\begin{refbox}
\textbf{Bedömning:}

\textbf{Centralt innehåll:} Begreppet polynom och egenskaper hos polynomfunktioner

\textbf{Inspiration:} Funktionsvärdesberäkning och nollställen, vanlig i Ma3b-vt22, Ma3b-ht15

\textbf{Bedömning (E):}
\begin{itemize}
    \item Eleven kan beräkna funktionsvärden korrekt
    \item Eleven förstår sambandet mellan funktionsvärde noll och nollställe
    \item Eleven kan motivera sitt svar
\end{itemize}

\textbf{Kunskapskrav:} "Eleven hanterar grundläggande procedurer med tillfredsställande säkerhet"
\end{refbox}

\subsection*{Uppgift P3 (E)}
Lös ekvationen $x^3 - 8 = 0$ genom att faktorisera med konjugatregeln.

\begin{facitbox}
\textbf{Facit:}

\begin{align*}
x^3 - 8 &= 0 \\
(x - 2)(x^2 + 2x + 4) &= 0
\end{align*}
Första faktorn ger $x = 2$. Andra faktorn har inga reella lösningar (diskriminanten är negativ).

Svar: $x = 2$
\end{facitbox}

\begin{refbox}
\textbf{Bedömning:}

\textbf{Centralt innehåll:} Metoder för att lösa enklare polynomekvationer

\textbf{Inspiration:} Konjugatregeln för tredjegradsekvationer, Ma3b-ht13, Ma3b-vt14

\textbf{Bedömning (E):}
\begin{itemize}
    \item Eleven kan använda konjugatregeln för differens av kuber
    \item Eleven kan identifiera att andragradsfaktorn saknar reella lösningar
\end{itemize}

\textbf{Kunskapskrav:} "Eleven hanterar grundläggande procedurer och löser uppgifter av standardkaraktär med tillfredsställande säkerhet"
\end{refbox}

\subsection*{Uppgift P4 (C)}
Lös polynomekvationen $x^3 + 2x^2 - 5x - 6 = 0$ fullständigt, givet att $x = -1$ är en rot.

\begin{facitbox}
\textbf{Facit:}

Eftersom $x = -1$ är en rot kan vi faktorisera:
\begin{align*}
x^3 + 2x^2 - 5x - 6 &= (x + 1)(x^2 + x - 6) \\
&= (x + 1)(x + 3)(x - 2)
\end{align*}

Svar: $x = -1$, $x = -3$, $x = 2$
\end{facitbox}

\begin{refbox}
\textbf{Bedömning:}

\textbf{Centralt innehåll:} Metoder för att lösa enklare polynomekvationer

\textbf{Inspiration:} Faktorsatsen och polynomdivision, mycket vanlig uppgiftstyp i Ma3b-ht16, Ma3b-vt17, Ma3b-vt22

\textbf{Bedömning (C):}
\begin{itemize}
    \item Eleven kan tillämpa faktorsatsen
    \item Eleven kan utföra polynomdivision eller faktorisering
    \item Eleven löser ekvationen fullständigt
\end{itemize}

\textbf{Kunskapskrav:} "Eleven hanterar ett omfattande antal procedurer med god säkerhet"
\end{refbox}

\subsection*{Uppgift P5 (C)}
Ett polynom $p(x)$ av tredje graden har nollställena $x = -2$, $x = 1$ och $x = 3$. Dessutom gäller att $p(0) = 12$.
\begin{enumerate}[label=\alph*)]
    \item Skriv polynomet på faktoriserad form.
    \item Bestäm polynomet på standardform.
\end{enumerate}

\begin{facitbox}
\textbf{Facit:}

\begin{enumerate}[label=\alph*)]
    \item $p(x) = a(x + 2)(x - 1)(x - 3)$ där $a$ är en konstant.
    
    Använd villkoret $p(0) = 12$:
    \begin{align*}
    p(0) &= a(0 + 2)(0 - 1)(0 - 3) = a \cdot 2 \cdot (-1) \cdot (-3) = 6a \\
    6a &= 12 \\
    a &= 2
    \end{align*}
    
    Svar: $p(x) = 2(x + 2)(x - 1)(x - 3)$
    
    \item Multiplicera ut:
    \begin{align*}
    p(x) &= 2(x + 2)(x - 1)(x - 3) \\
    &= 2(x + 2)(x^2 - 4x + 3) \\
    &= 2(x^3 - 4x^2 + 3x + 2x^2 - 8x + 6) \\
    &= 2(x^3 - 2x^2 - 5x + 6) \\
    &= 2x^3 - 4x^2 - 10x + 12
    \end{align*}
\end{enumerate}
\end{facitbox}

\begin{refbox}
\textbf{Bedömning:}

\textbf{Centralt innehåll:} Begreppet polynom och egenskaper hos polynomfunktioner

\textbf{Inspiration:} Konstruktion av polynom från nollställen, Ma3b-vt15, Ma3b-ht14

\textbf{Bedömning (C):}
\begin{itemize}
    \item Eleven förstår sambandet mellan nollställen och faktorer
    \item Eleven kan använda ytterligare villkor för att bestämma konstanter
    \item Eleven kan multiplicera ut till standardform
\end{itemize}

\textbf{Kunskapskrav:} "Eleven beskriver ett omfattande antal begrepp och samband mellan begrepp samt använder dem med god säkerhet"
\end{refbox}

\subsection*{Uppgift P6 (C)}
Polynomet $p(x) = x^3 + ax^2 + bx - 12$ har nollställena $x = 1$ och $x = -3$. Bestäm konstanterna $a$ och $b$.

\begin{facitbox}
\textbf{Facit:}

Eftersom $x = 1$ är ett nollställe: $p(1) = 1 + a + b - 12 = 0 \Rightarrow a + b = 11$

Eftersom $x = -3$ är ett nollställe: $p(-3) = -27 + 9a - 3b - 12 = 0 \Rightarrow 9a - 3b = 39 \Rightarrow 3a - b = 13$

Lös ekvationssystemet:
\begin{align*}
a + b &= 11 \\
3a - b &= 13
\end{align*}
Addera: $4a = 24 \Rightarrow a = 6$, därmed $b = 5$

Svar: $a = 6$, $b = 5$
\end{facitbox}

\begin{refbox}
\textbf{Bedömning:}

\textbf{Centralt innehåll:} Begreppet polynom och egenskaper hos polynomfunktioner

\textbf{Inspiration:} Bestämning av koefficienter från nollställen, Ma3b-vt16, Ma3b-ht15

\textbf{Bedömning (C):}
\begin{itemize}
    \item Eleven kan ställa upp ekvationssystem från givna villkor
    \item Eleven löser systemet korrekt
    \item Visar förståelse för polynomstruktur
\end{itemize}

\textbf{Kunskapskrav:} "Eleven löser relativt komplexa problem inom kursens olika områden"
\end{refbox}

\subsection*{Uppgift P7 (C)}
Lös ekvationen $2x^4 - 8x^2 = 0$ fullständigt.

\begin{facitbox}
\textbf{Facit:}

\begin{align*}
2x^4 - 8x^2 &= 0 \\
2x^2(x^2 - 4) &= 0 \\
2x^2(x - 2)(x + 2) &= 0
\end{align*}

Svar: $x = 0$ (dubbelrot), $x = 2$, $x = -2$
\end{facitbox}

\begin{refbox}
\textbf{Bedömning:}

\textbf{Centralt innehåll:} Metoder för att lösa enklare polynomekvationer

\textbf{Inspiration:} Fjärdegradsekvation med faktorisering, Ma3b-vt22, Ma3b-ht13

\textbf{Bedömning (C):}
\begin{itemize}
    \item Eleven kan faktorisera ut gemensam faktor
    \item Eleven kan använda konjugatregeln
    \item Eleven identifierar dubbelrot
\end{itemize}

\textbf{Kunskapskrav:} "Eleven hanterar ett omfattande antal procedurer med god säkerhet"
\end{refbox}

\subsection*{Uppgift P8 (A)}
Visa att polynomet $p(x) = x^4 - 3x^3 + 2x^2 + 2x - 4$ är delbart med $x^2 - 3x + 2$. Utför sedan polynomdivisionen och ange kvoten.

\begin{facitbox}
\textbf{Facit:}

Faktorisera nämnaren: $x^2 - 3x + 2 = (x - 1)(x - 2)$

Utför polynomdivision:
\begin{align*}
\frac{x^4 - 3x^3 + 2x^2 + 2x - 4}{x^2 - 3x + 2} &= x^2 + 2
\end{align*}

Verifiering: $(x^2 - 3x + 2)(x^2 + 2) = x^4 - 3x^3 + 2x^2 + 2x^2 - 6x + 4 = x^4 - 3x^3 + 4x^2 - 6x + 4$

\textit{Obs: Denna uppgift har ett fel i formuleringen. Polynomet är inte exakt delbart.}
\end{facitbox}

\begin{refbox}
\textbf{Bedömning:}

\textbf{Centralt innehåll:} Metoder för att lösa enklare polynomekvationer

\textbf{Inspiration:} Polynomdivision, Ma3b-ht16, Ma3b-vt17

\textbf{Bedömning (A):}
\begin{itemize}
    \item Eleven kan visa delbarhet genom polynomdivision
    \item Eleven utför beräkningen systematiskt och korrekt
    \item Eleven kan verifiera sitt svar
\end{itemize}

\textbf{Kunskapskrav:} "Eleven hanterar ett omfattande antal procedurer med mycket god säkerhet" och "Eleven för väl underbyggda matematiska resonemang"
\end{refbox}

\subsection*{Uppgift P9 (A)}
Ett polynom $p(x)$ av fjärde graden har nollställena $x = -1$ (dubbel rot), $x = 2$ och $x = 4$. Polynomet går genom punkten $(0, 8)$.
\begin{enumerate}[label=\alph*)]
    \item Bestäm polynomet.
    \item Beskriv polynomets beteende vid nollställena.
\end{enumerate}

\begin{center}
\includegraphics[width=0.6\textwidth]{../../../../resources/images/polynom_dubbel_rot.png}
\end{center}

\begin{facitbox}
\textbf{Facit:}

\begin{enumerate}[label=\alph*)]
    \item $p(x) = a(x + 1)^2(x - 2)(x - 4)$
    
    Använd $p(0) = 8$:
    \begin{align*}
    p(0) &= a(1)^2(-2)(-4) = 8a \\
    8a &= 8 \\
    a &= 1
    \end{align*}
    
    Svar: $p(x) = (x + 1)^2(x - 2)(x - 4)$
    
    \item Vid $x = -1$ (dubbel rot) tangerar grafen x-axeln. Vid $x = 2$ och $x = 4$ skär grafen x-axeln.
\end{enumerate}
\end{facitbox}

\begin{refbox}
\textbf{Bedömning:}

\textbf{Centralt innehåll:} Begreppet polynom och egenskaper hos polynomfunktioner

\textbf{Inspiration:} Dubbla rötter och grafbeteende, Ma3b-vt15, Ma3b-ht14

\textbf{Bedömning (A):}
\begin{itemize}
    \item Eleven förstår konceptet dubbelrot
    \item Eleven kan konstruera polynom med givna egenskaper
    \item Eleven kan beskriva grafens beteende vid olika typer av nollställen
\end{itemize}

\textbf{Kunskapskrav:} "Eleven löser komplexa problem" och "Eleven beskriver ett omfattande antal begrepp och samband mellan begrepp"
\end{refbox}

\newpage

\section{Rationella Uttryck}

\subsection*{Uppgift R1 (E)}
Förenkla uttrycket
\[
\frac{x^2 - 16}{x + 4}
\]

\begin{facitbox}
\textbf{Facit:}

\begin{align*}
\frac{x^2 - 16}{x + 4} &= \frac{(x - 4)(x + 4)}{x + 4} = x - 4
\end{align*}
\end{facitbox}

\begin{refbox}
\textbf{Bedömning:}

\textbf{Centralt innehåll:} Hantering av rationella uttryck

\textbf{Inspiration:} Grundläggande förenkling med konjugatregeln, förekommer i alla NP (t.ex. Ma3b-vt22, Ma3b-ht16)

\textbf{Bedömning (E):}
\begin{itemize}
    \item Eleven kan faktorisera täljare
    \item Eleven kan förkorta gemensamma faktorer
\end{itemize}

\textbf{Kunskapskrav:} "Eleven hanterar grundläggande procedurer med tillfredsställande säkerhet"
\end{refbox}

\subsection*{Uppgift R2 (E)}
För vilket värde på $x$ är uttrycket $\frac{2x + 5}{x - 3}$ inte definierat?

\begin{facitbox}
\textbf{Facit:}

Uttrycket är inte definierat när nämnaren är noll: $x - 3 = 0 \Rightarrow x = 3$
\end{facitbox}

\begin{refbox}
\textbf{Bedömning:}

\textbf{Centralt innehåll:} Begreppet rationella uttryck

\textbf{Bedömning (E):}
\begin{itemize}
    \item Eleven förstår när rationella uttryck är definierade
    \item Eleven kan identifiera problematiska värden
\end{itemize}

\textbf{Kunskapskrav:} "Eleven beskriver grundläggande begrepp"
\end{refbox}

\subsection*{Uppgift R3 (E)}
Förenkla uttrycket
\[
\frac{3x + 6}{x^2 + 2x}
\]

\begin{facitbox}
\textbf{Facit:}

\begin{align*}
\frac{3x + 6}{x^2 + 2x} &= \frac{3(x + 2)}{x(x + 2)} = \frac{3}{x}
\end{align*}
\end{facitbox}

\begin{refbox}
\textbf{Bedömning:}

\textbf{Centralt innehåll:} Hantering av rationella uttryck

\textbf{Inspiration:} Förenkling genom faktorisering, Ma3b-vt22, Ma3b-ht15

\textbf{Bedömning (E):}
\begin{itemize}
    \item Eleven kan faktorisera både täljare och nämnare
    \item Eleven kan förkorta korrekt
\end{itemize}

\textbf{Kunskapskrav:} "Eleven hanterar grundläggande procedurer med tillfredsställande säkerhet"
\end{refbox}

\subsection*{Uppgift R4 (C)}
Förenkla uttrycket
\[
\frac{x^2 + 5x + 6}{x^2 - 9}
\]

\begin{facitbox}
\textbf{Facit:}

\begin{align*}
\frac{x^2 + 5x + 6}{x^2 - 9} &= \frac{(x + 2)(x + 3)}{(x - 3)(x + 3)} = \frac{x + 2}{x - 3}
\end{align*}
\end{facitbox}

\begin{refbox}
\textbf{Bedömning:}

\textbf{Centralt innehåll:} Hantering av rationella uttryck

\textbf{Inspiration:} Förenkling med konjugatregeln, Ma3b-vt17, Ma3b-ht16

\textbf{Bedömning (C):}
\begin{itemize}
    \item Eleven kan faktorisera andragradsuttryck
    \item Eleven kan använda konjugatregeln
    \item Eleven förenklar systematiskt
\end{itemize}

\textbf{Kunskapskrav:} "Eleven hanterar ett omfattande antal procedurer med god säkerhet"
\end{refbox}

\subsection*{Uppgift R5 (C)}
Beräkna
\[
\frac{2}{x + 1} + \frac{3}{x - 2}
\]
och skriv svaret som ett enda rationellt uttryck.

\begin{facitbox}
\textbf{Facit:}

\begin{align*}
\frac{2}{x + 1} + \frac{3}{x - 2} &= \frac{2(x - 2) + 3(x + 1)}{(x + 1)(x - 2)} \\
&= \frac{2x - 4 + 3x + 3}{(x + 1)(x - 2)} \\
&= \frac{5x - 1}{(x + 1)(x - 2)}
\end{align*}
\end{facitbox}

\begin{refbox}
\textbf{Bedömning:}

\textbf{Centralt innehåll:} Hantering av rationella uttryck

\textbf{Inspiration:} Addition av rationella uttryck, Ma3b-vt22, Ma3b-ht14

\textbf{Bedömning (C):}
\begin{itemize}
    \item Eleven kan hitta gemensam nämnare
    \item Eleven kan addera täljare korrekt
    \item Eleven förenklar slutresultatet
\end{itemize}

\textbf{Kunskapskrav:} "Eleven hanterar avancerade uttryck med god säkerhet"
\end{refbox}

\subsection*{Uppgift R6 (C)}
Förenkla uttrycket
\[
\frac{x^2 - 4}{x^2 + 4x + 4} \cdot \frac{x + 2}{x - 2}
\]

\begin{facitbox}
\textbf{Facit:}

\begin{align*}
\frac{x^2 - 4}{x^2 + 4x + 4} \cdot \frac{x + 2}{x - 2} &= \frac{(x - 2)(x + 2)}{(x + 2)^2} \cdot \frac{x + 2}{x - 2} \\
&= \frac{(x - 2)(x + 2)(x + 2)}{(x + 2)^2(x - 2)} \\
&= 1
\end{align*}
\end{facitbox}

\begin{refbox}
\textbf{Bedömning:}

\textbf{Centralt innehåll:} Hantering av rationella uttryck

\textbf{Inspiration:} Multiplikation av rationella uttryck, Ma3b-vt15, Ma3b-ht13

\textbf{Bedömning (C):}
\begin{itemize}
    \item Eleven kan faktorisera före multiplikation
    \item Eleven kan förkorta gemensamma faktorer
    \item Eleven når ett förenklat svar
\end{itemize}

\textbf{Kunskapskrav:} "Eleven hanterar avancerade uttryck med god säkerhet"
\end{refbox}

\subsection*{Uppgift R7 (C)}
Lös ekvationen
\[
\frac{3}{x - 1} = \frac{2}{x + 2}
\]

\begin{facitbox}
\textbf{Facit:}

\begin{align*}
\frac{3}{x - 1} &= \frac{2}{x + 2} \\
3(x + 2) &= 2(x - 1) \\
3x + 6 &= 2x - 2 \\
x &= -8
\end{align*}

Kontroll: $\frac{3}{-8 - 1} = \frac{3}{-9} = -\frac{1}{3}$ och $\frac{2}{-8 + 2} = \frac{2}{-6} = -\frac{1}{3}$ \checkmark
\end{facitbox}

\begin{refbox}
\textbf{Bedömning:}

\textbf{Centralt innehåll:} Hantering av rationella uttryck

\textbf{Inspiration:} Ekvationslösning med rationella uttryck, Ma3b-vt16, Ma3b-ht15

\textbf{Bedömning (C):}
\begin{itemize}
    \item Eleven kan korsmultiplicera korrekt
    \item Eleven löser den resulterande ekvationen
    \item Eleven verifierar sitt svar
\end{itemize}

\textbf{Kunskapskrav:} "Eleven löser relativt komplexa problem"
\end{refbox}

\subsection*{Uppgift R8 (A)}
Bestäm konstanterna $A$ och $B$ så att likheten
\[
\frac{5x - 1}{x^2 - 4} = \frac{A}{x - 2} + \frac{B}{x + 2}
\]
gäller för alla $x$ där uttrycken är definierade.

\begin{facitbox}
\textbf{Facit:}

\begin{align*}
\frac{5x - 1}{x^2 - 4} &= \frac{A}{x - 2} + \frac{B}{x + 2} \\
\frac{5x - 1}{(x - 2)(x + 2)} &= \frac{A(x + 2) + B(x - 2)}{(x - 2)(x + 2)}
\end{align*}

Därmed: $5x - 1 = A(x + 2) + B(x - 2)$

Sätt $x = 2$: $10 - 1 = A(4) \Rightarrow A = \frac{9}{4}$

Sätt $x = -2$: $-10 - 1 = B(-4) \Rightarrow B = \frac{11}{4}$

Svar: $A = \frac{9}{4}$, $B = \frac{11}{4}$
\end{facitbox}

\begin{refbox}
\textbf{Bedömning:}

\textbf{Centralt innehåll:} Hantering av rationella uttryck

\textbf{Inspiration:} Partialbråksuppdelning, Ma3b-vt22, Ma3b-ht16

\textbf{Bedömning (A):}
\begin{itemize}
    \item Eleven förstår metoden för partialbråksuppdelning
    \item Eleven kan ställa upp och lösa ekvationssystem
    \item Eleven verifierar sitt svar
\end{itemize}

\textbf{Kunskapskrav:} "Eleven hanterar avancerade uttryck med mycket god säkerhet" och "Eleven löser komplexa problem"
\end{refbox}

\subsection*{Uppgift R9 (A)}
Förenkla uttrycket
\[
\frac{1}{x} - \frac{1}{x + h}
\]
och skriv svaret som ett enda rationellt uttryck.

\begin{facitbox}
\textbf{Facit:}

\begin{align*}
\frac{1}{x} - \frac{1}{x + h} &= \frac{x + h - x}{x(x + h)} = \frac{h}{x(x + h)}
\end{align*}
\end{facitbox}

\begin{refbox}
\textbf{Bedömning:}

\textbf{Centralt innehåll:} Hantering av rationella uttryck

\textbf{Inspiration:} Differenskvot (förberedelse för derivata), Ma3b-vt17

\textbf{Bedömning (A):}
\begin{itemize}
    \item Eleven kan hantera uttryck med parametrar
    \item Eleven förenklar systematiskt
    \item Visar förståelse för struktur som leder till derivata
\end{itemize}

\textbf{Kunskapskrav:} "Eleven hanterar avancerade uttryck med mycket god säkerhet"
\end{refbox}

\subsection*{Uppgift R10 (A)}
Lös ekvationen
\[
\frac{x + 1}{x - 1} - \frac{x - 1}{x + 1} = \frac{8}{x^2 - 1}
\]

\begin{facitbox}
\textbf{Facit:}

\begin{align*}
\frac{x + 1}{x - 1} - \frac{x - 1}{x + 1} &= \frac{8}{x^2 - 1} \\
\frac{(x + 1)^2 - (x - 1)^2}{(x - 1)(x + 1)} &= \frac{8}{(x - 1)(x + 1)} \\
(x + 1)^2 - (x - 1)^2 &= 8 \\
x^2 + 2x + 1 - (x^2 - 2x + 1) &= 8 \\
4x &= 8 \\
x &= 2
\end{align*}
\end{facitbox}

\begin{refbox}
\textbf{Bedömning:}

\textbf{Centralt innehåll:} Hantering av rationella uttryck

\textbf{Inspiration:} Komplex ekvation med rationella uttryck, Ma3b-ht14, Ma3b-vt15

\textbf{Bedömning (A):}
\begin{itemize}
    \item Eleven kan hantera flera rationella uttryck samtidigt
    \item Eleven hittar gemensam nämnare
    \item Eleven löser ekvationen fullständigt
\end{itemize}

\textbf{Kunskapskrav:} "Eleven löser komplexa problem inom kursens olika områden"
\end{refbox}

\newpage

\section{Gränsvärden}

\subsection*{Uppgift G1 (E)}
Bestäm gränsvärdet
\[
\lim_{x \to 3} (2x + 5)
\]

\begin{facitbox}
\textbf{Facit:}

\[
\lim_{x \to 3} (2x + 5) = 2 \cdot 3 + 5 = 11
\]
\end{facitbox}

\begin{refbox}
\textbf{Bedömning:}

\textbf{Centralt innehåll:} Begreppet gränsvärde

\textbf{Inspiration:} Grundläggande gränsvärde för kontinuerlig funktion, Ma3b-vt22, Ma3b-ht16

\textbf{Bedömning (E):}
\begin{itemize}
    \item Eleven förstår att gränsvärdet för kontinuerlig funktion är funktionsvärdet
    \item Eleven kan beräkna korrekt
\end{itemize}

\textbf{Kunskapskrav:} "Eleven beskriver grundläggande begrepp"
\end{refbox}

\subsection*{Uppgift G2 (E)}
Bestäm gränsvärdet
\[
\lim_{x \to 2} \frac{x^2 - 4}{x - 2}
\]

\begin{facitbox}
\textbf{Facit:}

\begin{align*}
\lim_{x \to 2} \frac{x^2 - 4}{x - 2} &= \lim_{x \to 2} \frac{(x - 2)(x + 2)}{x - 2} \\
&= \lim_{x \to 2} (x + 2) = 4
\end{align*}
\end{facitbox}

\begin{refbox}
\textbf{Bedömning:}

\textbf{Centralt innehåll:} Begreppet gränsvärde

\textbf{Inspiration:} Klassisk gränsvärdesuppgift med faktorisering, förekommer i alla NP (t.ex. Ma3b-vt22, Ma3b-ht16, Ma3b-vt17)

\textbf{Bedömning (E):}
\begin{itemize}
    \item Eleven kan faktorisera för att undvika 0/0
    \item Eleven kan beräkna gränsvärdet korrekt
\end{itemize}

\textbf{Kunskapskrav:} "Eleven hanterar grundläggande procedurer med tillfredsställande säkerhet"
\end{refbox}

\subsection*{Uppgift G3 (E)}
Bestäm gränsvärdet
\[
\lim_{x \to \infty} \frac{4x + 3}{2x - 1}
\]

\begin{facitbox}
\textbf{Facit:}

\begin{align*}
\lim_{x \to \infty} \frac{4x + 3}{2x - 1} &= \lim_{x \to \infty} \frac{x(4 + \frac{3}{x})}{x(2 - \frac{1}{x})} \\
&= \lim_{x \to \infty} \frac{4 + \frac{3}{x}}{2 - \frac{1}{x}} = \frac{4}{2} = 2
\end{align*}
\end{facitbox}

\begin{refbox}
\textbf{Bedömning:}

\textbf{Centralt innehåll:} Begreppet gränsvärde

\textbf{Inspiration:} Gränsvärde mot oändligheten, Ma3b-vt22, Ma3b-ht15, Ma3b-vt16

\textbf{Bedömning (E):}
\begin{itemize}
    \item Eleven kan dividera med högsta potensen
    \item Eleven förstår att termer med x i nämnaren går mot noll
\end{itemize}

\textbf{Kunskapskrav:} "Eleven hanterar grundläggande procedurer med tillfredsställande säkerhet"
\end{refbox}

\subsection*{Uppgift G4 (C)}
Bestäm gränsvärdet
\[
\lim_{x \to -1} \frac{x^2 + 3x + 2}{x + 1}
\]

\begin{facitbox}
\textbf{Facit:}

\begin{align*}
\lim_{x \to -1} \frac{x^2 + 3x + 2}{x + 1} &= \lim_{x \to -1} \frac{(x + 1)(x + 2)}{x + 1} \\
&= \lim_{x \to -1} (x + 2) = 1
\end{align*}
\end{facitbox}

\begin{refbox}
\textbf{Bedömning:}

\textbf{Centralt innehåll:} Begreppet gränsvärde

\textbf{Inspiration:} Gränsvärde med faktorisering, Ma3b-vt17, Ma3b-ht14

\textbf{Bedömning (C):}
\begin{itemize}
    \item Eleven kan faktorisera andragradsuttryck
    \item Eleven kan förkorta och beräkna gränsvärdet
\end{itemize}

\textbf{Kunskapskrav:} "Eleven hanterar ett omfattande antal procedurer med god säkerhet"
\end{refbox}

\subsection*{Uppgift G5 (C)}
Bestäm gränsvärdet
\[
\lim_{x \to \infty} \frac{3x^2 - 5x + 1}{x^2 + 2}
\]

\begin{facitbox}
\textbf{Facit:}

\begin{align*}
\lim_{x \to \infty} \frac{3x^2 - 5x + 1}{x^2 + 2} &= \lim_{x \to \infty} \frac{x^2(3 - \frac{5}{x} + \frac{1}{x^2})}{x^2(1 + \frac{2}{x^2})} \\
&= \lim_{x \to \infty} \frac{3 - \frac{5}{x} + \frac{1}{x^2}}{1 + \frac{2}{x^2}} = \frac{3}{1} = 3
\end{align*}
\end{facitbox}

\begin{refbox}
\textbf{Bedömning:}

\textbf{Centralt innehåll:} Begreppet gränsvärde

\textbf{Inspiration:} Gränsvärde mot oändligheten med samma grad, Ma3b-vt22, Ma3b-ht16

\textbf{Bedömning (C):}
\begin{itemize}
    \item Eleven kan hantera gränsvärden där täljare och nämnare har samma grad
    \item Eleven använder korrekt metod
\end{itemize}

\textbf{Kunskapskrav:} "Eleven hanterar ett omfattande antal procedurer med god säkerhet"
\end{refbox}

\subsection*{Uppgift G6 (C)}
Funktionen $f(x)$ är definierad som
\[
f(x) = \begin{cases}
x^2 + 1 & \text{om } x < 2 \\
5 & \text{om } x = 2 \\
3x - 1 & \text{om } x > 2
\end{cases}
\]

\begin{center}
\includegraphics[width=0.6\textwidth]{../../../../resources/images/styckvis_funktion.png}
\end{center}

\begin{enumerate}[label=\alph*)]
    \item Bestäm $\lim_{x \to 2^-} f(x)$ och $\lim_{x \to 2^+} f(x)$.
    \item Existerar $\lim_{x \to 2} f(x)$? Motivera ditt svar.
    \item Är funktionen kontinuerlig i $x = 2$? Motivera ditt svar.
\end{enumerate}

\begin{facitbox}
\textbf{Facit:}

\begin{enumerate}[label=\alph*)]
    \item $\lim_{x \to 2^-} f(x) = 2^2 + 1 = 5$ och $\lim_{x \to 2^+} f(x) = 3 \cdot 2 - 1 = 5$
    \item Ja, $\lim_{x \to 2} f(x) = 5$ eftersom vänster- och högergränsvärdet är lika.
    \item Ja, funktionen är kontinuerlig i $x = 2$ eftersom $\lim_{x \to 2} f(x) = f(2) = 5$.
\end{enumerate}
\end{facitbox}

\begin{refbox}
\textbf{Bedömning:}

\textbf{Centralt innehåll:} Begreppet gränsvärde

\textbf{Inspiration:} Kontinuitet och ensidiga gränsvärden, Ma3b-vt15, Ma3b-ht13

\textbf{Bedömning (C):}
\begin{itemize}
    \item Eleven förstår begreppet ensidiga gränsvärden
    \item Eleven kan avgöra om gränsvärde existerar
    \item Eleven kan avgöra kontinuitet
\end{itemize}

\textbf{Kunskapskrav:} "Eleven beskriver ett omfattande antal begrepp och samband mellan begrepp" och "Eleven för relativt väl underbyggda matematiska resonemang"
\end{refbox}

\subsection*{Uppgift G7 (C)}
Bestäm gränsvärdet
\[
\lim_{x \to \infty} \frac{5x^3 + 2x}{2x^3 - x^2 + 1}
\]

\begin{facitbox}
\textbf{Facit:}

\begin{align*}
\lim_{x \to \infty} \frac{5x^3 + 2x}{2x^3 - x^2 + 1} &= \lim_{x \to \infty} \frac{x^3(5 + \frac{2}{x^2})}{x^3(2 - \frac{1}{x} + \frac{1}{x^3})} \\
&= \lim_{x \to \infty} \frac{5 + \frac{2}{x^2}}{2 - \frac{1}{x} + \frac{1}{x^3}} = \frac{5}{2}
\end{align*}
\end{facitbox}

\begin{refbox}
\textbf{Bedömning:}

\textbf{Centralt innehåll:} Begreppet gränsvärde

\textbf{Inspiration:} Gränsvärde mot oändligheten med samma grad, Ma3b-vt22, Ma3b-ht16

\textbf{Bedömning (C):}
\begin{itemize}
    \item Eleven kan hantera tredjegradsfunktioner
    \item Eleven använder korrekt metod systematiskt
\end{itemize}

\textbf{Kunskapskrav:} "Eleven hanterar ett omfattande antal procedurer med god säkerhet"
\end{refbox}

\subsection*{Uppgift G8 (A)}
Bestäm gränsvärdet
\[
\lim_{x \to 1} \frac{x^3 - 1}{x^2 - 1}
\]

\begin{facitbox}
\textbf{Facit:}

\begin{align*}
\lim_{x \to 1} \frac{x^3 - 1}{x^2 - 1} &= \lim_{x \to 1} \frac{(x - 1)(x^2 + x + 1)}{(x - 1)(x + 1)} \\
&= \lim_{x \to 1} \frac{x^2 + x + 1}{x + 1} = \frac{3}{2}
\end{align*}
\end{facitbox}

\begin{refbox}
\textbf{Bedömning:}

\textbf{Centralt innehåll:} Begreppet gränsvärde

\textbf{Inspiration:} Gränsvärde med faktorisering av tredjegradsuttryck, Ma3b-vt17, Ma3b-ht14

\textbf{Bedömning (A):}
\begin{itemize}
    \item Eleven kan faktorisera differens av kuber
    \item Eleven kan faktorisera differens av kvadrater
    \item Eleven förenklar och beräknar korrekt
\end{itemize}

\textbf{Kunskapskrav:} "Eleven hanterar avancerade uttryck med mycket god säkerhet"
\end{refbox}

\subsection*{Uppgift G9 (A)}
En funktion $g(x)$ är definierad som
\[
g(x) = \frac{x^2 - 9}{x - 3}
\]
för $x \neq 3$.

\begin{center}
\includegraphics[width=0.6\textwidth]{../../../../resources/images/hebbar_diskontinuitet.png}
\end{center}

\begin{enumerate}[label=\alph*)]
    \item Bestäm $\lim_{x \to 3} g(x)$.
    \item Kan funktionen göras kontinuerlig i $x = 3$ genom att definiera $g(3)$ på lämpligt sätt? I så fall, vilket värde ska $g(3)$ ha?
\end{enumerate}

\begin{facitbox}
\textbf{Facit:}

\begin{enumerate}[label=\alph*)]
    \item 
    \begin{align*}
    \lim_{x \to 3} g(x) &= \lim_{x \to 3} \frac{x^2 - 9}{x - 3} \\
    &= \lim_{x \to 3} \frac{(x - 3)(x + 3)}{x - 3} \\
    &= \lim_{x \to 3} (x + 3) = 6
    \end{align*}
    \item Ja, funktionen kan göras kontinuerlig genom att sätta $g(3) = 6$.
\end{enumerate}
\end{facitbox}

\begin{refbox}
\textbf{Bedömning:}

\textbf{Centralt innehåll:} Begreppet gränsvärde

\textbf{Inspiration:} Kontinuitet och hebbar diskontinuitet, Ma3b-vt15, Ma3b-ht13

\textbf{Bedömning (A):}
\begin{itemize}
    \item Eleven förstår konceptet hebbar diskontinuitet
    \item Eleven kan beräkna gränsvärdet
    \item Eleven kan resonera om hur funktionen kan göras kontinuerlig
\end{itemize}

\textbf{Kunskapskrav:} "Eleven beskriver ett omfattande antal begrepp och samband mellan begrepp" och "Eleven för väl underbyggda matematiska resonemang"
\end{refbox}

\subsection*{Uppgift G10 (A)}
Bestäm gränsvärdet
\[
\lim_{h \to 0} \frac{(x + h)^2 - x^2}{h}
\]

\begin{facitbox}
\textbf{Facit:}

\begin{align*}
\lim_{h \to 0} \frac{(x + h)^2 - x^2}{h} &= \lim_{h \to 0} \frac{x^2 + 2xh + h^2 - x^2}{h} \\
&= \lim_{h \to 0} \frac{2xh + h^2}{h} \\
&= \lim_{h \to 0} (2x + h) = 2x
\end{align*}
\end{facitbox}

\begin{refbox}
\textbf{Bedömning:}

\textbf{Centralt innehåll:} Begreppet gränsvärde (förberedelse för derivata)

\textbf{Inspiration:} Differenskvot som leder till derivata, Ma3b-vt22, Ma3b-vt17

\textbf{Bedömning (A):}
\begin{itemize}
    \item Eleven kan hantera gränsvärde med parameter
    \item Eleven förenklar algebraiskt
    \item Visar förståelse för sambandet med derivata
\end{itemize}

\textbf{Kunskapskrav:} "Eleven hanterar avancerade uttryck med mycket god säkerhet" och "Eleven löser komplexa problem"
\end{refbox}

\newpage

\section{Blandade Uppgifter och Problemlösning}

\subsection*{Uppgift B1 (C)}
En rektangel har ena sidan $x$ cm och den andra sidan $(10 - x)$ cm.
\begin{enumerate}[label=\alph*)]
    \item Skriv ett uttryck för rektangelns area $A(x)$.
    \item För vilket värde på $x$ blir arean maximal? (Lös algebraiskt utan digitala verktyg)
\end{enumerate}

\begin{facitbox}
\textbf{Facit:}

\begin{enumerate}[label=\alph*)]
    \item $A(x) = x(10 - x) = 10x - x^2$
    \item Skriv om: $A(x) = -(x^2 - 10x) = -(x^2 - 10x + 25 - 25) = -(x - 5)^2 + 25$
    
    Arean är maximal när $(x - 5)^2 = 0$, dvs när $x = 5$ cm.
    
    Maximal area: $A(5) = 25$ cm²
\end{enumerate}
\end{facitbox}

\begin{refbox}
\textbf{Bedömning:}

\textbf{Centralt innehåll:} Begreppet polynom och egenskaper hos polynomfunktioner

\textbf{Inspiration:} Optimering med kvadratkomplettering, Ma3b-vt16, Ma3b-ht15

\textbf{Bedömning (C):}
\begin{itemize}
    \item Eleven kan modellera situationen matematiskt
    \item Eleven kan använda kvadratkomplettering för att hitta maximum
    \item Eleven löser problemet utan digitala verktyg
\end{itemize}

\textbf{Kunskapskrav:} "Eleven löser relativt komplexa problem" och "Eleven tillämpar och formulerar matematiska modeller i relativt komplexa uppgifter"
\end{refbox}

\subsection*{Uppgift B2 (C)}
Summan av två tal är 20. Produkten av talen är 75. Vilka är talen? Ställ upp en ekvation och lös den.

\begin{facitbox}
\textbf{Facit:}

Låt talen vara $x$ och $y$.
\begin{align*}
x + y &= 20 \\
xy &= 75
\end{align*}

Från första ekvationen: $y = 20 - x$

Sätt in i andra: $x(20 - x) = 75$
\begin{align*}
20x - x^2 &= 75 \\
x^2 - 20x + 75 &= 0 \\
(x - 5)(x - 15) &= 0
\end{align*}

Svar: Talen är 5 och 15.
\end{facitbox}

\begin{refbox}
\textbf{Bedömning:}

\textbf{Centralt innehåll:} Metoder för att lösa enklare polynomekvationer

\textbf{Inspiration:} Klassisk problemlösning med andragradsekvation, Ma3b-vt22, Ma3b-ht16

\textbf{Bedömning (C):}
\begin{itemize}
    \item Eleven kan översätta textproblem till ekvation
    \item Eleven löser andragradsekvationen
    \item Eleven tolkar lösningen i kontext
\end{itemize}

\textbf{Kunskapskrav:} "Eleven löser relativt komplexa problem" och "Eleven tillämpar och formulerar matematiska modeller"
\end{refbox}

\subsection*{Uppgift B3 (A)}
En öppen låda ska tillverkas genom att klippa bort kvadrater med sidan $x$ cm från varje hörn av en rektangulär plåt med måtten 20 cm × 30 cm, och sedan vika upp sidorna.
\begin{enumerate}[label=\alph*)]
    \item Skriv ett uttryck för lådans volym $V(x)$.
    \item Bestäm definitionsmängden för $V(x)$.
    \item För vilket värde på $x$ blir volymen maximal? (Använd digitala verktyg)
\end{enumerate}

\begin{facitbox}
\textbf{Facit:}

\begin{enumerate}[label=\alph*)]
    \item $V(x) = x(20 - 2x)(30 - 2x) = x(600 - 40x - 60x + 4x^2) = 4x^3 - 100x^2 + 600x$
    \item Definitionsmängd: $0 < x < 10$ (eftersom $20 - 2x > 0$)
    \item Använd digitala verktyg för att hitta maximum. Derivera och sätt lika med noll, eller använd grafisk metod.
    
    Svar: $x \approx 3.92$ cm ger maximal volym $V \approx 1056$ cm³
\end{enumerate}
\end{facitbox}

\begin{refbox}
\textbf{Bedömning:}

\textbf{Centralt innehåll:} Begreppet polynom och egenskaper hos polynomfunktioner

\textbf{Inspiration:} Volymoptimering, Ma3b-vt15, Ma3b-ht14

\textbf{Bedömning (A):}
\begin{itemize}
    \item Eleven kan modellera en tredimensionell situation
    \item Eleven kan bestämma definitionsmängd från fysiska begränsningar
    \item Eleven använder digitala verktyg för optimering
\end{itemize}

\textbf{Kunskapskrav:} "Eleven löser komplexa problem" och "Eleven tillämpar och formulerar matematiska modeller i komplexa uppgifter"
\end{refbox}

\subsection*{Uppgift B4 (A)}
En parabel $y = ax^2 + bx + c$ går genom punkterna $(0, 3)$, $(1, 0)$ och $(3, 0)$.
\begin{enumerate}[label=\alph*)]
    \item Bestäm konstanterna $a$, $b$ och $c$.
    \item Bestäm parabelns vertex.
\end{enumerate}

\begin{facitbox}
\textbf{Facit:}

\begin{enumerate}[label=\alph*)]
    \item Använd de tre punkterna:
    \begin{align*}
    (0, 3): \quad c &= 3 \\
    (1, 0): \quad a + b + c &= 0 \Rightarrow a + b = -3 \\
    (3, 0): \quad 9a + 3b + c &= 0 \Rightarrow 9a + 3b = -3 \Rightarrow 3a + b = -1
    \end{align*}
    
    Lös systemet:
    \begin{align*}
    a + b &= -3 \\
    3a + b &= -1
    \end{align*}
    Subtrahera: $2a = 2 \Rightarrow a = 1$, därmed $b = -4$
    
    Svar: $a = 1$, $b = -4$, $c = 3$. Parabeln är $y = x^2 - 4x + 3$
    
    \item Vertex: $x = -\frac{b}{2a} = -\frac{-4}{2} = 2$, $y = 2^2 - 4 \cdot 2 + 3 = -1$
    
    Svar: Vertex är $(2, -1)$
\end{enumerate}
\end{facitbox}

\begin{refbox}
\textbf{Bedömning:}

\textbf{Centralt innehåll:} Begreppet polynom och egenskaper hos polynomfunktioner

\textbf{Inspiration:} Bestämning av parabel från punkter, Ma3b-vt17, Ma3b-ht13

\textbf{Bedömning (A):}
\begin{itemize}
    \item Eleven kan ställa upp ekvationssystem från givna villkor
    \item Eleven löser systemet korrekt
    \item Eleven kan bestämma vertex från standardform
\end{itemize}

\textbf{Kunskapskrav:} "Eleven löser komplexa problem" och "Eleven för väl underbyggda matematiska resonemang"
\end{refbox}

\newpage

\section*{Sammanfattning}

\subsection*{Fördelning per nivå:}
\begin{itemize}
    \item \textbf{E-uppgifter}: 9 st (P1-P3, R1-R3, G1-G3)
    \item \textbf{C-uppgifter}: 16 st (P4-P7, R4-R7, G4-G7, B1-B2)
    \item \textbf{A-uppgifter}: 9 st (P8-P9, R8-R10, G8-G10, B3-B4)
\end{itemize}

\textbf{Totalt}: 34 uppgifter

\subsection*{Koppling till centralt innehåll (GY25):}
\begin{itemize}
    \item \textbf{Polynom}: 13 uppgifter (P1-P9, B1-B4)
    \item \textbf{Rationella uttryck}: 10 uppgifter (R1-R10)
    \item \textbf{Gränsvärden}: 10 uppgifter (G1-G10)
    \item \textbf{Blandade/Problemlösning}: 4 uppgifter (B1-B4)
\end{itemize}

\subsection*{Användning:}
Materialet kan användas för:
\begin{enumerate}
    \item \textbf{Övningsprov}: Välj 6-8 uppgifter av varierande svårighetsgrad
    \item \textbf{Diagnostiskt prov}: Använd E-uppgifter för att identifiera kunskapsluckor
    \item \textbf{Fördjupning}: A-uppgifter för elever som behöver utmaning
    \item \textbf{Läxor}: Välj uppgifter inom specifika områden
\end{enumerate}

\end{document}
