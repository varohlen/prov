\documentclass[12pt]{article}
\usepackage[utf8]{inputenc}
\usepackage[swedish]{babel}
\usepackage{amsmath}
\usepackage{amssymb}
\usepackage{geometry}
\usepackage{graphicx}
\usepackage{enumitem}
\geometry{a4paper, margin=1in}

\title{Facit: Övningsbank Algebra och Funktioner \\ \large Matematik 3b}
\author{}
\date{}

\begin{document}

\maketitle

\section{Polynom och Polynomekvationer}

\subsection*{Uppgift P1}
Graden är 4 och koefficienten för högsta gradtermen är 3.

\subsection*{Uppgift P2}
\begin{enumerate}[label=\alph*)]
    \item $p(2) = 2^3 - 4 \cdot 2^2 + 2 + 6 = 8 - 16 + 2 + 6 = 0$
    \item Ja, $x = 2$ är ett nollställe eftersom $p(2) = 0$.
\end{enumerate}

\subsection*{Uppgift P3}
\begin{align*}
x^3 - 8 &= 0 \\
(x - 2)(x^2 + 2x + 4) &= 0
\end{align*}
Första faktorn ger $x = 2$. Andra faktorn har inga reella lösningar (diskriminanten är negativ).

Svar: $x = 2$

\subsection*{Uppgift P4}
Eftersom $x = -1$ är en rot kan vi faktorisera:
\begin{align*}
x^3 + 2x^2 - 5x - 6 &= (x + 1)(x^2 + x - 6) \\
&= (x + 1)(x + 3)(x - 2)
\end{align*}

Svar: $x = -1$, $x = -3$, $x = 2$

\subsection*{Uppgift P5}
\begin{enumerate}[label=\alph*)]
    \item $p(x) = a(x + 2)(x - 1)(x - 3)$ där $a$ är en konstant.
    
    Använd villkoret $p(0) = 12$:
    \begin{align*}
    p(0) &= a(0 + 2)(0 - 1)(0 - 3) = a \cdot 2 \cdot (-1) \cdot (-3) = 6a \\
    6a &= 12 \\
    a &= 2
    \end{align*}
    
    Svar: $p(x) = 2(x + 2)(x - 1)(x - 3)$
    
    \item Multiplicera ut:
    \begin{align*}
    p(x) &= 2(x + 2)(x - 1)(x - 3) \\
    &= 2(x + 2)(x^2 - 4x + 3) \\
    &= 2(x^3 - 4x^2 + 3x + 2x^2 - 8x + 6) \\
    &= 2(x^3 - 2x^2 - 5x + 6) \\
    &= 2x^3 - 4x^2 - 10x + 12
    \end{align*}
\end{enumerate}

\subsection*{Uppgift P6}
Eftersom $x = 1$ är ett nollställe: $p(1) = 1 + a + b - 12 = 0 \Rightarrow a + b = 11$

Eftersom $x = -3$ är ett nollställe: $p(-3) = -27 + 9a - 3b - 12 = 0 \Rightarrow 9a - 3b = 39 \Rightarrow 3a - b = 13$

Lös ekvationssystemet:
\begin{align*}
a + b &= 11 \\
3a - b &= 13
\end{align*}
Addera: $4a = 24 \Rightarrow a = 6$, därmed $b = 5$

Svar: $a = 6$, $b = 5$

\subsection*{Uppgift P7}
\begin{align*}
2x^4 - 8x^2 &= 0 \\
2x^2(x^2 - 4) &= 0 \\
2x^2(x - 2)(x + 2) &= 0
\end{align*}

Svar: $x = 0$ (dubbelrot), $x = 2$, $x = -2$

\subsection*{Uppgift P8}
Faktorisera nämnaren: $x^2 - 3x + 2 = (x - 1)(x - 2)$

Utför polynomdivision:
\begin{align*}
\frac{x^4 - 3x^3 + 2x^2 + 2x - 4}{x^2 - 3x + 2} &= x^2 + 2
\end{align*}

Verifiering: $(x^2 - 3x + 2)(x^2 + 2) = x^4 - 3x^3 + 2x^2 + 2x^2 - 6x + 4 = x^4 - 3x^3 + 4x^2 - 6x + 4$

Korrigering: Låt mig räkna om...
$(x^2 - 3x + 2)(x^2) = x^4 - 3x^3 + 2x^2$
Återstår: $2x - 4$
$(x^2 - 3x + 2)(2) = 2x^2 - 6x + 4$

Kvoten är $x^2 + 2$ med rest $8x - 8$. Polynomet är inte exakt delbart.

\subsection*{Uppgift P9}
\begin{enumerate}[label=\alph*)]
    \item $p(x) = a(x + 1)^2(x - 2)(x - 4)$
    
    Använd $p(0) = 8$:
    \begin{align*}
    p(0) &= a(1)^2(-2)(-4) = 8a \\
    8a &= 8 \\
    a &= 1
    \end{align*}
    
    Svar: $p(x) = (x + 1)^2(x - 2)(x - 4)$
    
    \item Vid $x = -1$ (dubbel rot) tangerar grafen x-axeln. Vid $x = 2$ och $x = 4$ skär grafen x-axeln.
\end{enumerate}

\newpage

\section{Rationella Uttryck}

\subsection*{Uppgift R1}
\begin{align*}
\frac{x^2 - 16}{x + 4} &= \frac{(x - 4)(x + 4)}{x + 4} = x - 4
\end{align*}

\subsection*{Uppgift R2}
Uttrycket är inte definierat när nämnaren är noll: $x - 3 = 0 \Rightarrow x = 3$

\subsection*{Uppgift R3}
\begin{align*}
\frac{3x + 6}{x^2 + 2x} &= \frac{3(x + 2)}{x(x + 2)} = \frac{3}{x}
\end{align*}

\subsection*{Uppgift R4}
\begin{align*}
\frac{x^2 + 5x + 6}{x^2 - 9} &= \frac{(x + 2)(x + 3)}{(x - 3)(x + 3)} = \frac{x + 2}{x - 3}
\end{align*}

\subsection*{Uppgift R5}
\begin{align*}
\frac{2}{x + 1} + \frac{3}{x - 2} &= \frac{2(x - 2) + 3(x + 1)}{(x + 1)(x - 2)} \\
&= \frac{2x - 4 + 3x + 3}{(x + 1)(x - 2)} \\
&= \frac{5x - 1}{(x + 1)(x - 2)}
\end{align*}

\subsection*{Uppgift R6}
\begin{align*}
\frac{x^2 - 4}{x^2 + 4x + 4} \cdot \frac{x + 2}{x - 2} &= \frac{(x - 2)(x + 2)}{(x + 2)^2} \cdot \frac{x + 2}{x - 2} \\
&= \frac{(x - 2)(x + 2)(x + 2)}{(x + 2)^2(x - 2)} \\
&= 1
\end{align*}

\subsection*{Uppgift R7}
\begin{align*}
\frac{3}{x - 1} &= \frac{2}{x + 2} \\
3(x + 2) &= 2(x - 1) \\
3x + 6 &= 2x - 2 \\
x &= -8
\end{align*}

Kontroll: $\frac{3}{-8 - 1} = \frac{3}{-9} = -\frac{1}{3}$ och $\frac{2}{-8 + 2} = \frac{2}{-6} = -\frac{1}{3}$ ✓

\subsection*{Uppgift R8}
\begin{align*}
\frac{5x - 1}{x^2 - 4} &= \frac{A}{x - 2} + \frac{B}{x + 2} \\
\frac{5x - 1}{(x - 2)(x + 2)} &= \frac{A(x + 2) + B(x - 2)}{(x - 2)(x + 2)}
\end{align*}

Därmed: $5x - 1 = A(x + 2) + B(x - 2)$

Sätt $x = 2$: $10 - 1 = A(4) \Rightarrow A = \frac{9}{4}$

Sätt $x = -2$: $-10 - 1 = B(-4) \Rightarrow B = \frac{11}{4}$

Svar: $A = \frac{9}{4}$, $B = \frac{11}{4}$

\subsection*{Uppgift R9}
\begin{align*}
\frac{1}{x} - \frac{1}{x + h} &= \frac{x + h - x}{x(x + h)} = \frac{h}{x(x + h)}
\end{align*}

\subsection*{Uppgift R10}
\begin{align*}
\frac{x + 1}{x - 1} - \frac{x - 1}{x + 1} &= \frac{8}{x^2 - 1} \\
\frac{(x + 1)^2 - (x - 1)^2}{(x - 1)(x + 1)} &= \frac{8}{(x - 1)(x + 1)} \\
(x + 1)^2 - (x - 1)^2 &= 8 \\
x^2 + 2x + 1 - (x^2 - 2x + 1) &= 8 \\
4x &= 8 \\
x &= 2
\end{align*}

\newpage

\section{Gränsvärden}

\subsection*{Uppgift G1}
\[
\lim_{x \to 3} (2x + 5) = 2 \cdot 3 + 5 = 11
\]

\subsection*{Uppgift G2}
\begin{align*}
\lim_{x \to 2} \frac{x^2 - 4}{x - 2} &= \lim_{x \to 2} \frac{(x - 2)(x + 2)}{x - 2} \\
&= \lim_{x \to 2} (x + 2) = 4
\end{align*}

\subsection*{Uppgift G3}
\begin{align*}
\lim_{x \to \infty} \frac{4x + 3}{2x - 1} &= \lim_{x \to \infty} \frac{x(4 + \frac{3}{x})}{x(2 - \frac{1}{x})} \\
&= \lim_{x \to \infty} \frac{4 + \frac{3}{x}}{2 - \frac{1}{x}} = \frac{4}{2} = 2
\end{align*}

\subsection*{Uppgift G4}
\begin{align*}
\lim_{x \to -1} \frac{x^2 + 3x + 2}{x + 1} &= \lim_{x \to -1} \frac{(x + 1)(x + 2)}{x + 1} \\
&= \lim_{x \to -1} (x + 2) = 1
\end{align*}

\subsection*{Uppgift G5}
\begin{align*}
\lim_{x \to \infty} \frac{3x^2 - 5x + 1}{x^2 + 2} &= \lim_{x \to \infty} \frac{x^2(3 - \frac{5}{x} + \frac{1}{x^2})}{x^2(1 + \frac{2}{x^2})} \\
&= \lim_{x \to \infty} \frac{3 - \frac{5}{x} + \frac{1}{x^2}}{1 + \frac{2}{x^2}} = \frac{3}{1} = 3
\end{align*}

\subsection*{Uppgift G6}
\begin{enumerate}[label=\alph*)]
    \item $\lim_{x \to 2^-} f(x) = 2^2 + 1 = 5$ och $\lim_{x \to 2^+} f(x) = 3 \cdot 2 - 1 = 5$
    \item Ja, $\lim_{x \to 2} f(x) = 5$ eftersom vänster- och högergränsvärdet är lika.
    \item Ja, funktionen är kontinuerlig i $x = 2$ eftersom $\lim_{x \to 2} f(x) = f(2) = 5$.
\end{enumerate}

\subsection*{Uppgift G7}
\begin{align*}
\lim_{x \to \infty} \frac{5x^3 + 2x}{2x^3 - x^2 + 1} &= \lim_{x \to \infty} \frac{x^3(5 + \frac{2}{x^2})}{x^3(2 - \frac{1}{x} + \frac{1}{x^3})} \\
&= \lim_{x \to \infty} \frac{5 + \frac{2}{x^2}}{2 - \frac{1}{x} + \frac{1}{x^3}} = \frac{5}{2}
\end{align*}

\subsection*{Uppgift G8}
\begin{align*}
\lim_{x \to 1} \frac{x^3 - 1}{x^2 - 1} &= \lim_{x \to 1} \frac{(x - 1)(x^2 + x + 1)}{(x - 1)(x + 1)} \\
&= \lim_{x \to 1} \frac{x^2 + x + 1}{x + 1} = \frac{3}{2}
\end{align*}

\subsection*{Uppgift G9}
\begin{enumerate}[label=\alph*)]
    \item 
    \begin{align*}
    \lim_{x \to 3} g(x) &= \lim_{x \to 3} \frac{x^2 - 9}{x - 3} \\
    &= \lim_{x \to 3} \frac{(x - 3)(x + 3)}{x - 3} \\
    &= \lim_{x \to 3} (x + 3) = 6
    \end{align*}
    \item Ja, funktionen kan göras kontinuerlig genom att sätta $g(3) = 6$.
\end{enumerate}

\subsection*{Uppgift G10}
\begin{align*}
\lim_{h \to 0} \frac{(x + h)^2 - x^2}{h} &= \lim_{h \to 0} \frac{x^2 + 2xh + h^2 - x^2}{h} \\
&= \lim_{h \to 0} \frac{2xh + h^2}{h} \\
&= \lim_{h \to 0} (2x + h) = 2x
\end{align*}

\newpage

\section{Blandade Uppgifter och Problemlösning}

\subsection*{Uppgift B1}
\begin{enumerate}[label=\alph*)]
    \item $A(x) = x(10 - x) = 10x - x^2$
    \item Skriv om: $A(x) = -(x^2 - 10x) = -(x^2 - 10x + 25 - 25) = -(x - 5)^2 + 25$
    
    Arean är maximal när $(x - 5)^2 = 0$, dvs när $x = 5$ cm.
    
    Maximal area: $A(5) = 25$ cm²
\end{enumerate}

\subsection*{Uppgift B2}
Låt talen vara $x$ och $y$.
\begin{align*}
x + y &= 20 \\
xy &= 75
\end{align*}

Från första ekvationen: $y = 20 - x$

Sätt in i andra: $x(20 - x) = 75$
\begin{align*}
20x - x^2 &= 75 \\
x^2 - 20x + 75 &= 0 \\
(x - 5)(x - 15) &= 0
\end{align*}

Svar: Talen är 5 och 15.

\subsection*{Uppgift B3}
\begin{enumerate}[label=\alph*)]
    \item $V(x) = x(20 - 2x)(30 - 2x) = x(600 - 40x - 60x + 4x^2) = 4x^3 - 100x^2 + 600x$
    \item Definitionsmängd: $0 < x < 10$ (eftersom $20 - 2x > 0$)
    \item Använd digitala verktyg för att hitta maximum. Derivera och sätt lika med noll, eller använd grafisk metod.
    
    Svar: $x \approx 3.92$ cm ger maximal volym $V \approx 1056$ cm³
\end{enumerate}

\subsection*{Uppgift B4}
\begin{enumerate}[label=\alph*)]
    \item Använd de tre punkterna:
    \begin{align*}
    (0, 3): \quad c &= 3 \\
    (1, 0): \quad a + b + c &= 0 \Rightarrow a + b = -3 \\
    (3, 0): \quad 9a + 3b + c &= 0 \Rightarrow 9a + 3b = -3 \Rightarrow 3a + b = -1
    \end{align*}
    
    Lös systemet:
    \begin{align*}
    a + b &= -3 \\
    3a + b &= -1
    \end{align*}
    Subtrahera: $2a = 2 \Rightarrow a = 1$, därmed $b = -4$
    
    Svar: $a = 1$, $b = -4$, $c = 3$. Parabeln är $y = x^2 - 4x + 3$
    
    \item Vertex: $x = -\frac{b}{2a} = -\frac{-4}{2} = 2$, $y = 2^2 - 4 \cdot 2 + 3 = -1$
    
    Svar: Vertex är $(2, -1)$
\end{enumerate}

\end{document}
