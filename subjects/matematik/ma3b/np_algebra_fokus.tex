\documentclass{article}
\usepackage[T1]{fontenc}
\usepackage[utf8]{inputenc}
\usepackage{amsmath}
\usepackage{amssymb}
\usepackage{geometry}
\geometry{a4paper, margin=1in}

\title{Nationella Prov Ma3b: Fokus Algebra \\ \large Polynom, Rationella Uttryck, Gränsvärden}
\author{Strikt urval från NP-uppgifter}
\date{\today}

\begin{document}

\maketitle

\section{Polynom och Polynomekvationer}
\begin{itemize}
    \item Lös ekvationen $0 = (x^2-1)(x-4)$. (NP Ma3b HT12, uppg. 4)
    \item Vilken grad har funktionen $f$ då $f(x) = 3x^4 + 7x^2 + 3$? (NP Ma3b HT13, uppg. 1a)
    \item Lös ekvationen $0 = (x+2)(x-3)(x+4)$. (NP Ma3b HT13, uppg. 4)
    \item Lös ekvationen $0 = (x+1)(x-1)(x-3)$. (NP Ma3b VT14, uppg. 6)
    \item Bestäm ett tal $A$ och ett tal $B$ så att tredjegradsekvationen $x(x+A)(x+B)=0$ får lösningarna $x_1=0$, $x_2=5$ och $x_3 = -7/3$. (NP Ma3b HT14, uppg. 6)
    \item Ange graden för polynomet $5x^5 + 4x^7 + 3x^3 - 8$. (NP Ma3b HT15, uppg. 1)
    \item Lös ekvationen $3x^4 = 8x^2 - 4$. (Bikvadratisk ekvation) (NP Ma3b VT22, uppg. 7)
    \item För polynomfunktionen $f$ gäller att $f'(x) > 0$ för alla $x$. Undersök hur många reella lösningar ekvationen $f(x) = 0$ har. (Resonemang, strängt växande funktion) (NP Ma3b HT13, uppg. 22)
\end{itemize}

\section{Rationella Uttryck}
\begin{itemize}
    \item Förenkla $\frac{(x+3)^5}{(x+3)^{10}}$ så långt som möjligt. (NP Ma3b VT14, uppg. 4a)
    \item Förenkla $\frac{a^2-1}{a^2+a}$ så långt som möjligt. (NP Ma3b VT14, uppg. 4b)
    \item Förenkla $x(x+7)(x-7)+3$ så långt som möjligt. (NP Ma3b HT14, uppg. 5a)
    \item Förenkla $\frac{1}{x+1} - 1$ så långt som möjligt. (NP Ma3b HT14, uppg. 5b)
    \item Förenkla $\frac{x^3+6x}{x^3-5x}$ så långt som möjligt. (NP Ma3b VT22, uppg. 8a)
    \item Förenkla $\frac{x^2+12x+18}{2(x^2-9)}$ så långt som möjligt. (NP Ma3b VT22, uppg. 8b)
    \item För vilket värde på $x$ är uttrycket $\frac{10x+2}{6x-3}$ inte definierat? (NP Ma3b HT15, uppg. 3)
    \item Bestäm konstanten $a$ så att $\frac{x^2-a}{(x-1)(x+3)}$ kan förkortas. (NP Ma3b VT15, uppg. 15)
    \item Förenkla $\frac{(x+4)^3}{(x+4)^4}$ (Förenkling av rationellt uttryck) (NP Ma3b HT13, uppg. 5b)
    \item Ge ett exempel på ett rationellt uttryck som: får värdet 0 då $x=-1$; är inte definierat för $x=3$; är inte definierat för $x=-4$. (NP Ma3b VT13, uppg. 9b)
    \item Lös ekvationen $\frac{2}{x-1} = \frac{2x}{x^2-1}$. (Rationell ekvation) (NP Ma3b VT14, uppg. 14)
\end{itemize}

\section{Gränsvärden och Derivatans Definition}
\begin{itemize}
    \item Bestäm derivatan till $f(x) = A/x$ med hjälp av derivatans definition. (Involverar gränsvärde) (NP Ma3b HT12, uppg. 16)
    \item Bestäm den övre gränsen för antalet fiskar ($N$) i populationsmodellen $N(t) = 3000 + \frac{15000}{e^{0.5t}}$. (Gränsvärde $t \to \infty$) (NP Ma3b VT13, uppg. 10b)
    \item Bestäm konstanten $A$ så att $\lim_{x \to \infty} \frac{A x^2 + 4 x}{7 x^2 + 5} = 4/7$. (NP Ma3b HT13, uppg. 15)
    \item Enligt modellen $N(t) = \frac{11}{1+3.4e^{-0.03t}}$ kommer antalet människor att närma sig en övre gräns. Bestäm denna övre gräns. (Gränsvärde $t \to \infty$) (NP Ma3b HT13, uppg. 20b)
    \item Bestäm $\lim_{h \to 0} \frac{3^h - 1}{h}$ och svara exakt. (Gränsvärde relaterat till derivatan av $3^x$) (NP Ma3b VT15, uppg. 10)
    \item Bestäm den undre gränsen för vattentemperaturen enligt modellen $T(x) = 17e^{-0.693x} + 5$. (Gränsvärde $x \to \infty$) (NP Ma3b VT15, uppg. 19d)
    \item Bestäm derivatan till $f(x) = 1/(\sqrt{ax})$ med hjälp av derivatans definition. (Involverar gränsvärde) (NP Ma3b VT16, uppg. 17)
    \item Bestäm derivatan till $f(x) = 5/x^2 - a/x^2$ med hjälp av derivatans definition. (NP Ma3b VT22, uppg. 17)
\end{itemize}

\end{document}
