\documentclass{exam}
\usepackage{graphicx}
\usepackage{xcolor}
\usepackage{framed}

% Format Header and footer
\pagestyle{headandfoot}
\header{\footnotesize Klass:\\Namn:}{\Large\textbf{Fysiologi II - Bedömningsanvisningar}\\\medskip\small Nervsystemet, rörelseapparaten och endokrina organsystemet}{\footnotesize BIOBIO02 - 2025\\Viktor Arohlén}
\headrule
\footrule
\setlength{\columnsep}{0.25cm}
\footer{}{Sida \thepage}{}

\newenvironment{answer}
  {\begin{framed}\color{blue}\textbf{Bedömning:} }
  {\end{framed}}

\begin{document}
\section*{Instruktioner}
Provet består av två delar \\
    - Grundläggande frågor, svara kortfattat (\textit{14 poäng})\\
    - Fördjupande frågor, svara mer omfattande (\textit{10 poäng} + 2 bonuspoäng)

\subsection*{Poäng}
Antalet poäng är markerat för varje fråga. Totalt \textbf{12 frågor} och \textbf{24 poäng}.\\ \textit{För godkänt resultat krävs 10 poäng.}

\vspace{5mm} %5mm vertical space
\begin{center}
\fbox{\fbox{\parbox{6in}{\centering
\textbf{Grundläggande frågor}: svara kortfattat (\textbf{14 poäng})
}}}
\end{center}
\begin{questions}

\question I vilken ordning sker de olika stegen i en aktionspotential i en nervcell? (\textbf{2 poäng})

\begin{itemize}
  \item Repolarisation av membranet
  \item Spänningsstyrda kaliumkanaler öppnas – kalium strömmar ut
  \item En retning når tröskelvärdet
  \item Spänningsstyrda natriumkanaler öppnas – natrium strömmar in
  \item Depolarisation av membranet
  \item Hyperpolarisation (refraktärperiod)
  \item Vilopotential upprätthålls av natrium-kaliumpumpen
\end{itemize}

\begin{answer}
\textbf{Korrekt ordning:}
\begin{enumerate}
  \item Vilopotential upprätthålls av natrium-kaliumpumpen
  \item En retning når tröskelvärdet
  \item Spänningsstyrda natriumkanaler öppnas – natrium strömmar in
  \item Depolarisation av membranet
  \item Spänningsstyrda kaliumkanaler öppnas – kalium strömmar ut
  \item Repolarisation av membranet
  \item Hyperpolarisation (refraktärperiod)
\end{enumerate}

\textbf{Poängbedömning:}
\begin{itemize}
  \item 2 poäng: Korrekt ordning på alla steg
  \item 1 poäng: Minst 5 steg i korrekt ordning
  \item 0 poäng: Färre än 5 steg i korrekt ordning
\end{itemize}
\end{answer}
\vspace{5mm} %5mm vertical space

\question Ge exempel på en \textbf{äkta led} och en \textbf{oäkta led} och vad de har för likheter och skillnader. (\textbf{2 poäng})
\vspace{10mm}

\begin{answer}
\textbf{Korrekt svar:}

\textbf{Äkta led (synovialled):} T.ex. knäled, armbågsled, höftled, axelled.
\begin{itemize}
  \item Har ledkapsel med ledvätska
  \item Ledhuvud och ledpanna täckta av ledbrosk
  \item Möjliggör större rörlighet
\end{itemize}

\textbf{Oäkta led:} T.ex. fogarna mellan ryggkotor, bäckenfogen, skallens fogar.
\begin{itemize}
  \item Saknar ledkapsel och ledvätska
  \item Förbinds av bindväv, brosk eller ben
  \item Mer begränsad rörlighet
\end{itemize}

\textbf{Likheter:}
\begin{itemize}
  \item Båda förbinder ben med varandra
  \item Båda möjliggör viss rörlighet (om än i olika grad)
\end{itemize}

\textbf{Skillnader:}
\begin{itemize}
  \item Äkta leder har ledkapsel och ledvätska, oäkta leder saknar detta
  \item Äkta leder har större rörlighet än oäkta leder
  \item Äkta leder har ledbrosk, oäkta leder har andra typer av vävnad mellan benen
\end{itemize}

\textbf{Poängbedömning:}
\begin{itemize}
  \item 2 poäng: Korrekt exempel på både äkta och oäkta led, samt relevanta likheter och skillnader
  \item 1 poäng: Korrekt exempel på antingen äkta eller oäkta led, samt några likheter eller skillnader
  \item 0 poäng: Felaktiga exempel eller avsaknad av likheter och skillnader
\end{itemize}
\end{answer}
\vspace{5mm}
