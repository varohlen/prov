\documentclass[12pt,a4paper]{article}

% --- PACKAGES ---
\usepackage[swedish]{babel}
\usepackage[utf8]{inputenc}
\usepackage{graphicx}
\usepackage{geometry}
\usepackage{hyperref}
\usepackage{tabularx}
\usepackage{ragged2e}

% --- SETTINGS ---
\geometry{a4paper, margin=1in}
\hypersetup{
    colorlinks=true,
    linkcolor=blue,
    filecolor=magenta,      
    urlcolor=cyan,
}
\parindent 0pt
\parskip 1em

% --- DOCUMENT ---
\begin{document}

\begin{center}
    {\huge \bfseries Uppgift: Evolutionsteorin på prov}
    \vspace{0.5em}
    \hrule
\end{center}

\vspace{1em}

\textbf{Syfte:} Att genom praktiska exempel och jämförelser förstå de grundläggande skillnaderna mellan evolutionsteorin, Lamarckism och Intelligent Design, samt att kunna argumentera för varför endast evolutionsteorin är vetenskapligt hållbar.

\textbf{Instruktioner:} Använd sammanfattningarna på \href{https://summor.se/nak2/5}{summor.se} för att lösa följande uppgifter. Arbeta i par eller smågrupper.

\newpage

\section*{Del 1: Fallstudien – Hur fick giraffen sin långa hals?}

Förklara uppkomsten av giraffens långa hals utifrån tre olika perspektiv. Beskriv kortfattat hur varje ``teori'' skulle resonera.

\subsection*{1. Lamarcks förklaring}
(Ledtråd: Tänk på förvärvade egenskaper.)

\vspace{5em}

\subsection*{2. Darwins förklaring (Evolutionsteorin)}
(Ledtråd: Tänk på variation, konkurrens och naturligt urval.)

\vspace{5em}

\subsection*{3. Intelligent Designs förklaring}
(Ledtråd: Tänk på syften och medveten design.)

\vspace{5em}

\begin{center}
    \includegraphics[width=0.5\textwidth]{images/giraff.jpg}
\end{center}

\newpage

\section*{Del 2: Det vetenskapliga testet}

Fyll i tabellen nedan. För varje alternativ, svara ``Ja'', ``Nej'' eller ``Delvis'' och motivera kort ert svar i rutan.

\vspace{1em}

\newcolumntype{L}{>{\RaggedRight\arraybackslash}X}

\begin{tabularx}{\textwidth}{|>{\RaggedRight}p{4cm}|L|L|L|}
\hline
\textbf{Kriterium för vetenskap} & \textbf{Evolutionsteorin} & \textbf{Lamarckism} & \textbf{Intelligent Design} \\ \hline

\textbf{Är den testbar/falsifierbar?} \newline \small (Kan man utforma ett experiment som skulle kunna motbevisa den?) & \vspace{4cm} & \vspace{4cm} & \vspace{4cm} \\ \hline

\textbf{Har den förklaringskraft?} \newline \small (Förklarar den många olika observationer, t.ex. fossiler, DNA-likheter och "dålig design"?) & \vspace{4cm} & \vspace{4cm} & \vspace{4cm} \\ \hline

\textbf{Bygger den på naturliga processer?} \newline \small (Används endast förklaringar som följer kända naturlagar, utan övernaturliga inslag?) & \vspace{4cm} & \vspace{4cm} & \vspace{4cm} \\ \hline
\end{tabularx}

\newpage

\section*{Del 3: Slutsats och argument}

1. Baserat på era svar i tabellen, vilken av de tre förklaringsmodellerna uppfyller kraven för en vetenskaplig teori?

\vspace{5em}

2. Den återkommande laryngealnerven (nerven i giraffens hals som tar en lång omväg) kallas ibland ett exempel på ``dålig design''. Varför är detta ett starkt argument \textbf{för} evolutionsteorin och \textbf{emot} Intelligent Design?

\vspace{5em}

\begin{center}
    \includegraphics[width=0.8\textwidth]{images/laryngeal.png}
\end{center}

\newpage

\section*{Del 4: Argumentation och analys}

Svara på de två följande argumentationsuppgifterna.

\subsection*{4a) Bemöt argumentet om komplexitet}

Föreställ dig att du försöker förklara evolutionsteorin för en vän som säger:

\textit{``Jag tror inte på evolutionen. Det är omöjligt att något så komplicerat som ett öga kan ha uppstått av en slump.''}

Formulera ett svar till din vän. Använd och förklara begreppen \textbf{variation}, \textbf{naturligt urval} och \textbf{gradvis utveckling} i ditt svar för att visa hur komplexa organ kan uppstå över lång tid utan en designer.

\vspace{8em}

\subsection*{4b) Bemöt det Lamarckistiska argumentet}

Din vän säger sedan: ``Men jag läste att giraffer fick långa halsar för att de sträckte på sig. Är inte det en enklare förklaring? Egenskaper man skaffar sig borde väl gå i arv?''

Förklara för din vän varför denna Lamarckistiska idé inte stämmer. Använd dina kunskaper om \textbf{DNA}, \textbf{gener} och \textbf{ärftlighet} för att förklara varför förvärvade egenskaper (som större muskler från träning) inte kan ärvas av ens barn.

\end{document}