\documentclass{exam}
\usepackage{graphicx}
\usepackage{xcolor}

%Format Header and footer
\pagestyle{headandfoot}
\header{\footnotesize Klass:\\Namn:}{\Large\textbf{Celldelning och mutationer 2025 - FACIT}}{\footnotesize  BIOBIO01 - 2025\\Viktor Arohlén}
\headrule
\footrule
\setlength{\columnsep}{0.25cm}
%\setlength{\columnseprule}{1pt}
\footer{}{Sida \thepage}{}
%\extrafootheight{-2cm}

% Define a command for facit sections
\newcommand{\facit}[1]{\textcolor{red}{\textbf{Facit:} #1}}
\newcommand{\bedomning}[1]{\textcolor{blue}{\textbf{Bedömningsanvisning:} #1}}

\begin{document}

% Kryssfrågor

\vspace{5mm}
\begin{center}
\fbox{\fbox{\parbox{6in}{\centering
\textbf{10 kryssfrågor}: markera endast ett alternativ. (1 poäng per fråga.)
}}}
\end{center}
\vspace{5mm}

\begin{questions}

% Kryssfrågor (nya men liknande)

\question Vad är syftet med \textbf{mitos}?
\begin{checkboxes}
   \choice Bilda könsceller
   \correctchoice Skapa två genetiskt identiska celler
   \choice Minska antalet kromosomer
   \choice Producera energi
\end{checkboxes}
\facit{Alternativ B: Skapa två genetiskt identiska celler}
\bedomning{1 poäng för korrekt svar. Inga delpoäng.}

\vspace{5mm}\hrule\vspace{5mm}

\question Vad är en \textbf{genmutation}?
\begin{checkboxes}
   \choice En förändring i antalet kromosomer
   \correctchoice En förändring i enskilda kvävebaser i DNA
   \choice En fördubbling av arvsmassan
   \choice En celldelning utan cytokines
\end{checkboxes}
\facit{Alternativ B: En förändring i enskilda kvävebaser i DNA}
\bedomning{1 poäng för korrekt svar. Inga delpoäng.}

\vspace{5mm}\hrule\vspace{5mm}

\question Vad kallas det när en cell dör på ett \textbf{kontrollerat sätt}?
\begin{checkboxes}
   \choice Nekros
   \correctchoice Apoptos
   \choice Mutation
   \choice Replikation
\end{checkboxes}
\facit{Alternativ B: Apoptos}
\bedomning{1 poäng för korrekt svar. Inga delpoäng.}

\vspace{5mm}\hrule\vspace{5mm}

\question Hur många \textbf{kromosomer} har en mänsklig \textbf{kroppscell} efter mitos?
\begin{checkboxes}
   \choice 23
   \correctchoice 46
   \choice 92
   \choice 44
\end{checkboxes}
\facit{Alternativ B: 46}
\bedomning{1 poäng för korrekt svar. Inga delpoäng.}

\vspace{5mm}\hrule\vspace{5mm}

\question Vilken typ av \textbf{mutation} kan gå i arv till nästa \textbf{generation}?
\begin{checkboxes}
   \choice Somatisk mutation
   \correctchoice Genetisk mutation i könsceller
   \choice Mutation i hudceller
   \choice Mutation i leverceller
\end{checkboxes}
\facit{Alternativ B: Genetisk mutation i könsceller}
\bedomning{1 poäng för korrekt svar. Inga delpoäng.}

\break

\question Vad är \textbf{kromatin}?
\begin{checkboxes}
   \choice Ett annat namn för DNA
   \choice Proteiner i cellkärnan
   \correctchoice DNA och proteiner
   \choice Fler än två kromosomer
\end{checkboxes}
\facit{Alternativ C: DNA och proteiner}
\bedomning{1 poäng för korrekt svar. Inga delpoäng.}

\vspace{5mm}\hrule\vspace{5mm}

\question Vilken av följande kan \textbf{INTE orsaka mutationer}?
\begin{checkboxes}
   \choice Strålning
   \choice Virus
   \choice Normal celldelning
   \correctchoice Inget av ovan
\end{checkboxes}
\facit{Alternativ D: Inget av ovan}
\bedomning{1 poäng för korrekt svar. Inga delpoäng. Alla alternativen kan orsaka mutationer.}

\vspace{5mm}\hrule\vspace{5mm}


\question Vad händer med \textbf{kromosomantalet} i cellerna efter \textbf{meios}?
\begin{checkboxes}
   \choice Det fördubblas
   \choice Det är oförändrat
   \correctchoice Det halveras
   \choice Det tredubblas
\end{checkboxes}
\facit{Alternativ C: Det halveras}
\bedomning{1 poäng för korrekt svar. Inga delpoäng.}

\vspace{5mm}\hrule\vspace{5mm}

\question Vilket av följande kännetecknar \textbf{cancerceller}?
\begin{checkboxes}
   \choice De slutar dela sig tidigt
   \correctchoice De delar sig okontrollerat
   \choice De har alltid färre kromosomer
   \choice De kan inte mutera
\end{checkboxes}
\facit{Alternativ B: De delar sig okontrollerat}
\bedomning{1 poäng för korrekt svar. Inga delpoäng.}

\vspace{5mm}\hrule\vspace{5mm}
\question Vilken av följande \textbf{könskromosomuppsättningar} är inte förenlig med \textbf{liv} hos människan?
\begin{checkboxes}
   \choice X 
   \choice XXY
   \choice XYY
   \correctchoice YY
\end{checkboxes}
\facit{Alternativ D: YY}
\bedomning{1 poäng för korrekt svar. Inga delpoäng. YY är inte förenligt med liv eftersom X-kromosomen innehåller gener som är nödvändiga för överlevnad.}

\break

\vspace{5mm}
\begin{center}
\fbox{\fbox{\parbox{6in}{\centering
\textbf{4 kortsvarsfrågor}: Svara kortfattat på frågorna nedan. Använd relevanta begrepp och figurer där det passar. (2 poäng per fråga)
}}}
\end{center}
\vspace{5mm}

\question Vad menas med en \textbf{haploid} respektive en \textbf{diploid} cell? Ge exempel på var i kroppen dessa finns.
\vspace{5mm}

\facit{
\begin{itemize}
    \item \textbf{Haploid cell:} En cell med en enkel uppsättning kromosomer (n). Exempel: könsceller (ägg och spermier).
    \item \textbf{Diploid cell:} En cell med dubbel uppsättning kromosomer (2n). Exempel: alla kroppsceller (somatiska celler) som hudceller, muskelceller, nervceller, etc.
\end{itemize}
}

\bedomning{
\begin{itemize}
    \item 1 poäng för korrekt definition av haploid (n) och diploid (2n) cell.
    \item 1 poäng för korrekta exempel på var dessa celltyper finns i kroppen.
    \item För full poäng krävs både definition och exempel.
\end{itemize}
}
\vspace{10mm}

\question Matcha vad som händer i \textbf{mitos} och \textbf{meios} med rätt fas.\\

\textbf{Alternativ:}
\begin{itemize}
  \item[A.] Kromosomer radar upp sig parvis i cellens mittplan.
  \item[B.] Systerkromatider dras isär till varsin cellpol.
  \item[C.] Överkorsning sker mellan homologa kromosomer.
  \item[D.] Kromosomer börjar kondenseras och blir synliga i en kroppscell.
  \item[E.] Kärnmembranet återbildas runt kromosomerna.
\end{itemize}

\textbf{Faser:}
\begin{itemize}
  \item[1.] Profas
  \item[2.] Profas I
  \item[3.] Metafas
  \item[4.] Anafas
  \item[5.] Telofas
\end{itemize}

\facit{
\begin{itemize}
  \item A - 3 (Metafas)
  \item B - 4 (Anafas)
  \item C - 2 (Profas I)
  \item D - 1 (Profas)
  \item E - 5 (Telofas)
\end{itemize}
}

\bedomning{
\begin{itemize}
  \item 0,5 poäng per korrekt matchning.
  \item För full poäng (2p) krävs minst 4 korrekta matchningar.
  \item 1 poäng för 2-3 korrekta matchningar.
  \item 0 poäng för 0-1 korrekta matchningar.
\end{itemize}
}
\vspace{5mm}

\question Vad menas med \textbf{"programmerad celldöd"} och varför är det viktigt?
\vspace{5mm}

\facit{
Programmerad celldöd (apoptos) är en kontrollerad process där cellen självdestruerar på ett organiserat sätt. Det är viktigt för:
\begin{itemize}
  \item Utveckling av organ och vävnader (t.ex. bildning av fingrar genom att celler mellan dem dör)
  \item Eliminering av skadade eller potentiellt cancerframkallande celler
  \item Reglering av immunsystemet
  \item Upprätthållande av cellbalans i vävnader (homeostas)
  \item Avlägsnande av överflödiga celler
\end{itemize}
}

\bedomning{
\begin{itemize}
  \item 1 poäng för korrekt definition av programmerad celldöd/apoptos som en kontrollerad process.
  \item 1 poäng för att nämna minst två korrekta biologiska betydelser.
  \item För full poäng krävs både definition och biologisk betydelse.
\end{itemize}
}
\vspace{10mm}

\question Vad innebär en \textbf{"trisomi"} och ge ett exempel?
\vspace{5mm}

\facit{
Trisomi innebär att det finns tre exemplar av en kromosom istället för det normala två (2n+1). Exempel:
\begin{itemize}
  \item Downs syndrom (trisomi 21)
  \item Edwards syndrom (trisomi 18)
  \item Pataus syndrom (trisomi 13)
  \item Klinefelters syndrom (XXY, trisomi av könskromosomer)
\end{itemize}
}

\bedomning{
\begin{itemize}
  \item 1 poäng för korrekt definition av trisomi (tre exemplar av en kromosom).
  \item 1 poäng för minst ett korrekt exempel.
  \item För full poäng krävs både definition och exempel.
\end{itemize}
}

\break

\vspace{5mm} %5mm vertical space
\begin{center}
\fbox{\fbox{\parbox{6in}{\centering
\textbf{2 frisvarsfrågor}:  Svara på utrymmet under frågan. Anvönd relevanta begrepp och figurer. (4 poäng per fråga)
}}}
\end{center}

\question
Hos människor och många djur skiljer sig könscellerna tydligt åt i både storlek och funktion:
\begin{itemize}
  \item \textbf{Anisogami:} Äggcellen är stor och näringsrik, spermien är liten och rörlig.
\end{itemize}

Hos vissa organismer (t.ex. många alger och svampar) förekommer istället:
\begin{itemize}
  \item \textbf{Isogami:} Könscellerna är lika stora och ofta likartade i form. Det finns ingen tydlig uppdelning i "hona" och "hane".
\end{itemize}

Exempel: Grönalgen \textit{Chlamydomonas}, där två lika stora könsceller smälter samman vid befruktning.

\begin{center}
    \includegraphics[width=0.28\textwidth]{chlamydomonas.jpg}
    
    \small{Bild: \textit{Chlamydomonas}, en encellig grönalg som förökar sig med isogami.}
\end{center}

\textbf{Fråga:} Vad kan det ha för biologisk betydelse att könscellerna är lika stora och likartade (isogami)?
\begin{itemize}
  \item Resonera kring möjliga \textbf{fördelar} och \textbf{nackdelar} med isogami jämfört med anisogami.
  \item Ta hjälp av dina kunskaper om \textbf{celldelning}, \textbf{energiförbrukning} och \textbf{befruktning}.
\end{itemize}

\facit{
Möjliga fördelar med isogami:
\begin{itemize}
  \item \textbf{Resurseffektivitet:} Båda könscellerna bidrar lika mycket med resurser till zygoten.
  \item \textbf{Energibesparing:} Ingen cell behöver investera extra energi i att producera näringsämnen.
  \item \textbf{Flexibilitet:} Varje individ kan potentiellt para sig med vilken annan individ som helst.
  \item \textbf{Snabbare celldelning:} Mindre celler kan produceras snabbare genom mitos/meios.
  \item \textbf{Enklare celldelningsprocess:} Likartade celler kan använda samma mekanismer.
\end{itemize}

Möjliga nackdelar med isogami:
\begin{itemize}
  \item \textbf{Mindre näring till zygoten:} Mindre totala resurser för den tidiga utvecklingen.
  \item \textbf{Lägre rörlighet:} Om båda cellerna är medelstora har de lägre rörlighet än små spermier.
  \item \textbf{Färre avkommor:} Anisogami tillåter produktion av många små spermier, vilket ökar chansen för befruktning.
  \item \textbf{Mindre genetisk variation:} Färre potentiella kombinationer jämfört med system där många spermier tävlar.
  \item \textbf{Begränsad miljöanpassning:} Fungerar bäst i vattenmiljöer där cellerna lätt kan mötas.
\end{itemize}
}

\bedomning{
\begin{itemize}
  \item 4 poäng: Utförligt resonemang som täcker både fördelar och nackdelar med isogami, med tydlig koppling till celldelning, energiförbrukning och befruktning. Minst 3-4 relevanta punkter från varje kategori med biologiskt korrekta förklaringar.
  
  \item 3 poäng: Bra resonemang med flera fördelar och nackdelar, men med mindre utförliga förklaringar eller färre punkter. Kopplingen till de biologiska processerna är tydlig men inte lika djupgående.
  
  \item 2 poäng: Grundläggande resonemang med några fördelar och nackdelar. Viss koppling till biologiska processer men med begränsad förklaring.
  
  \item 1 poäng: Enkelt resonemang med få punkter och begränsad biologisk koppling.
  
  \item 0 poäng: Inget relevant resonemang eller helt felaktiga påståenden.
\end{itemize}
}

\break

\question
De flesta kromosomavvikelser leder till missfall eftersom cellerna inte fungerar. Men det finns undantag där individen kan överleva och ibland leva ett relativt normalt liv.

\begin{itemize}
  \item \textbf{Downs syndrom:} Tre exemplar av kromosom 21 (trisomi 21). Påverkar utvecklingen men är förenligt med liv.
  \item \textbf{Turners syndrom:} Endast en X-kromosom (45,X).
  \item \textbf{Klinefelters syndrom:} En extra X-kromosom hos pojke (47,XXY).
\end{itemize}

\textbf{Fråga:} Varför kan vissa kromosomavvikelser – som Downs syndrom och vissa könskromosomavvikelser – vara förenliga med liv och ibland leda till vuxen ålder, medan andra avvikelser inte är det?

\begin{itemize}
  \item Resonera utifrån \textbf{celldelning}, \textbf{genbalans} och \textbf{kromosomernas funktion}.
\end{itemize}

\facit{
Varför vissa kromosomavvikelser är förenliga med liv:
\begin{itemize}
  \item \textbf{Genmängd och genbalans:} 
    \begin{itemize}
      \item Kromosom 21 är en av de minsta kromosomerna med relativt få gener (ca 200-300).
      \item Könskromosomer har specialiserade mekanismer för dosreglering (X-inaktivering).
      \item Mindre störning i den totala genbalansen jämfört med avvikelser i större kromosomer.
    \end{itemize}
  
  \item \textbf{Kromosomernas funktion:}
    \begin{itemize}
      \item Vissa kromosomer innehåller fler livsviktiga gener än andra.
      \item Könskromosomer har många gener som inte är essentiella för grundläggande cellprocesser.
      \item X-kromosomen har mekanismer för att stänga av extra kopior (Barr-kropp).
    \end{itemize}
    
  \item \textbf{Celldelning och utveckling:}
    \begin{itemize}
      \item Celler med vissa avvikelser kan fortfarande genomgå normal mitos.
      \item Vissa avvikelser påverkar främst specifika vävnader eller utvecklingsstadier.
      \item Kompensationsmekanismer kan utvecklas under fosterutvecklingen.
    \end{itemize}
\end{itemize}

Varför andra kromosomavvikelser inte är förenliga med liv:
\begin{itemize}
  \item \textbf{Genmängd och genbalans:}
    \begin{itemize}
      \item Större kromosomer innehåller fler gener, vilket ger större obalans.
      \item Avsaknad av hela kromosomer leder till brist på essentiella gener.
      \item Avsaknad av X-kromosomen är inte förenlig med liv (YY är inte livsdugligt).
    \end{itemize}
    
  \item \textbf{Kromosomernas funktion:}
    \begin{itemize}
      \item Vissa kromosomer innehåller gener som styr grundläggande cellprocesser.
      \item Avvikelser i kromosomer med gener för tidig embryoutveckling är ofta letala.
    \end{itemize}
    
  \item \textbf{Celldelning och utveckling:}
    \begin{itemize}
      \item Vissa avvikelser stör normal celldelning och därmed utvecklingen.
      \item Störningar i genreglering kan påverka kritiska utvecklingssteg.
    \end{itemize}
\end{itemize}
}

\bedomning{
\begin{itemize}
  \item 4 poäng: Utförligt resonemang som täcker alla tre aspekter (celldelning, genbalans och kromosomernas funktion) med biologiskt korrekta förklaringar. Tydlig jämförelse mellan avvikelser som är förenliga med liv och de som inte är det. Minst 4-5 relevanta punkter med förklaringar.
  
  \item 3 poäng: Bra resonemang som täcker minst två av aspekterna med biologiskt korrekta förklaringar. Jämförelse mellan olika typer av avvikelser finns men är mindre utförlig. 3-4 relevanta punkter.
  
  \item 2 poäng: Grundläggande resonemang som täcker minst en aspekt med någorlunda korrekta förklaringar. Viss jämförelse mellan olika typer av avvikelser. 2-3 relevanta punkter.
  
  \item 1 poäng: Enkelt resonemang med få punkter och begränsad biologisk koppling.
  
  \item 0 poäng: Inget relevant resonemang eller helt felaktiga påståenden.
\end{itemize}
}

\end{questions}

\end{document}
