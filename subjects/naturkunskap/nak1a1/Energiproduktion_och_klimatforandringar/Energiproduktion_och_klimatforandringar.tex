\documentclass{exam}
\usepackage{graphicx} 


%Format Header and footer
\pagestyle{headandfoot}
\header{\footnotesize Klass:\\Namn:}{\Large\textbf{Energiproduktion och klimatförändringar}}{\footnotesize Nak1a1\\2024}
\headrule
\footrule
\setlength{\columnsep}{0.25cm}
%\setlength{\columnseprule}{1pt}
\footer{}{Sida \thepage}{}
%\extrafootheight{-2cm}

\begin{document}
\vspace{5mm} %5mm vertical space
\begin{center}
\fbox{\fbox{\parbox{6in}{\centering
\textbf{10 kryssfrågor}: markera endast ett alternativ. (1 poäng per fråga.)
}}}
\end{center}
\vspace{5mm} %5mm vertical space

\begin{questions}

\question Vad av följande är \textbf{sant} för \textbf{energiprincipen?}
\begin{checkboxes}
   \choice Energi kan skapas
   \choice Energi kan förstöras
   \correctchoice Energi kan omvandlas
   \choice Energi kan växa
\end{checkboxes}

\vspace{5mm} 
\hrule 
\vspace{5mm} 

\question Vad är \textbf{joule}, \textbf{kalorier}, och \textbf{wattimme}?
\begin{checkboxes}
   \correctchoice Enheter för att mäta energi
   \choice Olika sätt energi kan användas
   \choice Internationella standarder för energikvalitet
   \choice De är inte jämförbara
\end{checkboxes}

\vspace{5mm} 
\hrule 
\vspace{5mm} 

\question Vad av följande \textbf{stämmer inte} för \textbf{fossila bränslen?}
\begin{checkboxes}
   \choice Fossila bränslen utgör majoriteten av energiproduktionen i \textit{världen}
   \choice Fossila bränslen är en resurs som kan ta slut
   \choice Fossila bränslen utgör majoriteten av alla utsläpp som bidrar till global uppvärmning
   \correctchoice Fossila bränslen bildar radioaktivt restavfall
\end{checkboxes}

\vspace{5mm} 
\hrule 
\vspace{5mm} 

\question Vad av följande är \textbf{inte} en \textbf{växthusgas}?
\begin{checkboxes}
   \choice Koldioxid
   \choice Metan
   \correctchoice Syrgas
   \choice Vattenånga
\end{checkboxes}
\vspace{5mm} 
\hrule 
\vspace{5mm} 

\question Vad är \textbf{klimatkompensering}?
\begin{checkboxes}
   \choice När en person gör livsstilsval för att reducera sitt ekologiska fotavtryck
   \correctchoice Att kompensera koldioxidutsläpp med andra åtgärder som reducerar klimatpåverkan
   \choice Att ett land utför åtgärder för att anpassa sig utefter kommande klimatförändringar
   \choice När företag ändrar sin miljöpolicy
\end{checkboxes}

\break
\question Vilken av följande \textbf{energiomvandlingar} beskriver bäst \textbf{vattenkraft}?
\begin{checkboxes}
   \choice Strålningsenergi $\to$ Kemisk energi $\to$ Elektrisk energi
   \correctchoice Lägesenergi $\to$ Rörelseenergi $\to$ Elektrisk energi
   \choice Rörelseenergi $\to$ Kemisk energi $\to$ Elektrisk energi
   \choice Rörelseenergi $\to$ Lägesenergi $\to$ Elektrisk energi
\end{checkboxes}

\vspace{5mm} 
\hrule 
\vspace{5mm} 

\question Vad är \textbf{klimatanpassning}?
\begin{checkboxes}
   \choice När en person gör livsstilsval för att reducera sitt ekologiska fotavtryck
   \choice Att kompensera koldioxidutsläpp med andra åtgärder som reducerar klimatpåverkan
   \correctchoice Att ett land utför åtgärder för att anpassa sig utefter kommande klimatförändringar
   \choice När företag ändrar sin miljöpolicy
\end{checkboxes}

\vspace{5mm} 
\hrule 
\vspace{5mm} 

\question Vad innebär en \textbf{zoonos}?
\begin{checkboxes}
   \correctchoice En sjukdom som sprider sig mellan arter
   \choice En sjukdom som har sitt ursprung i Asien
   \choice En sjukdom som främst drabbar djurparker
   \choice En sjukdom som får global spridning
\end{checkboxes}

\vspace{5mm} 
\hrule 
\vspace{5mm} 

\question Hur har \textbf{mängden energi} förändrats sedan universum skapades?
\begin{checkboxes}
   \choice Den har ökat
   \choice Den har minskat
   \correctchoice Den har varit konstant
   \choice Den har både ökat och minskat
\end{checkboxes}

\vspace{5mm} 
\hrule 
\vspace{5mm} 

\question Vilken är den \textbf{vanligaste gasen} i atmosfären?
\begin{checkboxes}
   \choice Syrgas
   \correctchoice Kvävgas
   \choice Koldioxid
   \choice Ozon
\end{checkboxes}

\break


\vspace{5mm} %5mm vertical space
\begin{center}
\fbox{\fbox{\parbox{6in}{\centering
\textbf{3 kortsvarsfrågor}: Svara kortfattat på utrymmet under frågan.  (2 poäng per fråga)
}}}
\end{center}
\vspace{5mm} %5mm vertical space
\question Markera vilken energiproduktion som är \textbf{förnyelsebar}, \textbf{icke-förnyelsebar} eller \textbf{inte är en energiproduktion}

\begin{tabular}{p{0.45\textwidth}p{0.45\textwidth}}
  \begin{minipage}[t]{\linewidth}
    \begin{itemize}
      \item[\textbf{A.}] Fossila bränslen
      \item[\textbf{B.}] Vattenkraft
      \item[\textbf{C.}] Solceller
      \item[\textbf{D.}] Litiumbatteri
      \item[\textbf{E.}] Kärnkraft
      \item[\textbf{E.}] Vågkraft
    \end{itemize}
  \end{minipage}
  &
  \begin{minipage}[t]{\linewidth}
    \begin{itemize}
      \item[\textbf{1.}] Icke-förnyelsebar
      \item[\textbf{2.}] Förnyelsebar
      \item[\textbf{3.}] Ej energiproduktion
    \end{itemize}
  \end{minipage}
\end{tabular}
\vspace{5mm} 
\hrule 
\vspace{5mm}
\question 
Nämn \textbf{3} konsekvenser av klimatförändringar (global uppvärmning).

\vspace{60mm} 
\hrule 
\vspace{5mm}
\question 
\textbf{Kärnkraft} bidrar inte till den globala uppvärmningen genom utsläpp. Det är trots det en energikälla många betraktar som miljöfarlig. Varför?

\break
\vspace{5mm} %5mm vertical space
\begin{center}
\fbox{\fbox{\parbox{6in}{\centering
\textbf{2 frisvarsfrågor}:  Svara på utrymmet under frågan. Använd relevanta begrepp och figurer. (4 poäng per fråga)
}}}
\end{center}
\vspace{5mm} %5mm vertical space
\question 
Skulle det vara möjlligt att ersätta alla \textbf{fossila bränslen} med \textbf{biobränslen}? Förklara och resonera!

\vspace{90mm} 
\hrule 
\vspace{5mm}
\question 
Beskriv \textbf{kolets kretslopp }och förklara hur det kan kopplas ihop med \textbf{global uppvärmning}

\end{questions}

\end{document}

