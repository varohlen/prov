\documentclass{exam}
\usepackage{graphicx} 


%Format Header and footer
\pagestyle{headandfoot}
\header{\footnotesize Klass:\\Namn:}{\Large\textbf{Energiproduktion och klimatförändringar}}{\footnotesize Nak1a1\\2025}
\headrule
\footrule
\setlength{\columnsep}{0.25cm}
%\setlength{\columnseprule}{1pt}
\footer{}{Sida \thepage}{}
%\extrafootheight{-2cm}

\begin{document}
\vspace{5mm} %5mm vertical space
\begin{center}
\fbox{\fbox{\parbox{6in}{\centering
\textbf{10 kryssfrågor}: markera endast ett alternativ. (1 poäng per fråga.)
}}}
\end{center}
\vspace{5mm} %5mm vertical space

\begin{questions}

\question Vad av följande är \textbf{sant} för \textbf{energiprincipen?}
\begin{checkboxes}
   \choice Energi kan skapas
   \choice Energi kan förstöras
   \choice Energin i universum ökar konstant
   \correctchoice Energi kan omvandlas
\end{checkboxes}

\vspace{5mm} 
\hrule 
\vspace{5mm} 

\question \textbf{Vilken} av följande enheter är inte ett mått på energi?
\begin{checkboxes}
   \correctchoice Ampere
   \choice Joule
   \choice Kalorier
   \choice Watt-timmar
\end{checkboxes}

\vspace{5mm} 
\hrule 
\vspace{5mm} 

\question Vad av följande \textbf{stämmer} för \textbf{fossila bränslen?}
\begin{checkboxes}
   \choice Fossila bränslen används inte för energiproduktion
   \choice Fossila bränslen är en resurs som inte kan ta slut
   \correctchoice Fossila bränslen utgör majoriteten av alla utsläpp som bidrar till global uppvärmning
   \choice Fossila bränslen bildar radioaktivt restavfall
\end{checkboxes}

\vspace{5mm} 
\hrule 
\vspace{5mm} 

\question Vad av följande är \textbf{inte} en \textbf{växthusgas}?
\begin{checkboxes}
   \choice Koldioxid
   \choice Metan
   \correctchoice Kvävgas
   \choice Vattenånga
\end{checkboxes}
\vspace{5mm} 
\hrule 
\vspace{5mm} 

\question Vad är huvudorsaken till den \textbf{globala uppvärmningen}?
\begin{checkboxes}
    \choice Ökad solaktivitet
    \correctchoice Utsläpp av växthusgaser
    \choice Naturliga klimatcykler
    \choice Förändringar i jordens bana
\end{checkboxes}
\break
\question Vilken av följande \textbf{energiomvandlingar} beskriver bäst \textbf{vattenkraft}?
\begin{checkboxes}
   \choice Strålningsenergi $\to$ Kemisk energi $\to$ Elektrisk energi
   \correctchoice Lägesenergi $\to$ Rörelseenergi $\to$ Elektrisk energi
   \choice Rörelseenergi $\to$ Kemisk energi $\to$ Elektrisk energi
   \choice Rörelseenergi $\to$ Lägesenergi $\to$ Elektrisk energi
\end{checkboxes}

\vspace{5mm} 
\hrule 
\vspace{5mm} 

\question Vad är en \textbf{nackdel} med solenergi?
\begin{checkboxes}
    \choice Det är en förnyelsebar energikälla
    \choice Det producerar inga växthusgaser
    \correctchoice Det är beroende av väder och tid på dygnet
    \choice Det kräver stora mängder vatten
\end{checkboxes}

\vspace{5mm} 
\hrule 
\vspace{5mm} 

\question Vilken av följande är en \textbf{konsekvens} av smältande polarisar?
\begin{checkboxes}
   \choice Minskad havsnivå
   \correctchoice Ökad havsnivå
   \choice Svalare klimat
   \choice Ökad ozonhalt
\end{checkboxes}

\vspace{5mm} 
\hrule 
\vspace{5mm} 

\question Vilken av följande är en \textbf{konsekvens} av ökad användning av förnyelsebar energi?
\begin{checkboxes}
   \choice Ökad utsläpp av växthusgaser
   \correctchoice Minskad beroende av fossila bränslen
   \choice Ökad försurning av haven
   \choice Minskad biodiversitet
\end{checkboxes}

\vspace{5mm} 
\hrule 
\vspace{5mm} 

\question Vilken är den \textbf{vanligaste gasen} i atmosfären?
\begin{checkboxes}
   \choice Vattenånga
   \correctchoice Kvävgas
   \choice Koldioxid
   \choice Argon
\end{checkboxes}

\break


\vspace{5mm} %5mm vertical space
\begin{center}
\fbox{\fbox{\parbox{6in}{\centering
\textbf{3 kortsvarsfrågor}: Svara kortfattat på utrymmet under frågan.  (2 poäng per fråga)
}}}
\end{center}
\vspace{5mm} %5mm vertical space
\question Markera vilken energiproduktion som är \textbf{förnyelsebar}, \textbf{icke-förnyelsebar} eller \textbf{inte är en energiproduktion}

\begin{tabular}{p{0.45\textwidth}p{0.45\textwidth}}
  \begin{minipage}[t]{\linewidth}
    \begin{itemize}
      \item[\textbf{A.}] Bensin
      \item[\textbf{B.}] Vattenkraft
      \item[\textbf{C.}] Biobränsle
      \item[\textbf{D.}] Blybatteri
      \item[\textbf{E.}] Kärnkraft
      \item[\textbf{E.}] Vågkraft
    \end{itemize}
  \end{minipage}
  &
  \begin{minipage}[t]{\linewidth}
    \begin{itemize}
      \item[\textbf{1.}] Icke-förnyelsebar
      \item[\textbf{2.}] Förnyelsebar
      \item[\textbf{3.}] Ej energiproduktion
    \end{itemize}
  \end{minipage}
\end{tabular}
\vspace{5mm} 
\hrule 
\vspace{5mm}
\question 
Nämn \textbf{3} konsekvenser av klimatförändringar (global uppvärmning).

\vspace{60mm} 
\hrule 
\vspace{5mm}
\question 
\textbf{Biobränsle} är en förnyelsebar energikälla. Det kan ändå påverka mängden koldioxiden i atmosfären. Hur då?

\break
\vspace{5mm} %5mm vertical space
\begin{center}
\fbox{\fbox{\parbox{6in}{\centering
\textbf{1 frisvarsfråga}:  Svara på utrymmet under frågan. Använd relevanta begrepp och figurer. (5 poäng per fråga)
}}}
\end{center}
\vspace{5mm} %5mm vertical space
\question 
\textbf{Diskutera möjligheterna} och \textbf{utmaningarna} med att övergå till en helt förnyelsebar energiförsörjning. Inkludera diskussion om olika energikällor och deras för- och nackdelar.

\end{questions}

\end{document}

