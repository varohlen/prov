\documentclass[a4paper,12pt]{exam}
\pagestyle{headandfoot}
\header{\footnotesize Klass:\\Namn:}{\Large\textbf{Ekologins grunder}}{\footnotesize NAKNAK01b - 2024\\Viktor Arohlén}
\headrule
\footrule
\setlength{\columnsep}{0.25cm}
\footer{}{Sida \thepage}{}
\begin{document}

\section*{Instruktioner}
Provet består av två delar \\
    - Grundläggande frågor, svara kortfattat (\textit{14 poäng})\\
    - Fördjupande frågor, svara mer omfattande (\textit{7 poäng})

\subsection*{Poäng}
Antalet poäng är markerat för varje fråga. Totalt \textbf{14 frågor} och \textbf{21 poäng}.\\ \textit{För godkänt resultat krävs 11 poäng.}

\vspace{5mm} %5mm vertical space
\begin{center}
\fbox{\fbox{\parbox{6in}{\centering
\textbf{Grundläggande frågor}: svara kortfattat (\textbf{14 poäng})
}}}
\end{center}

\begin{questions}
    
\question Vilken av följande är en abiotisk faktor? \textbf{(1 poäng)}
\vspace{5mm}
\begin{checkboxes}
    \choice Rovdjur
    \choice Nedbrytare
    \CorrectChoice Temperatur
    \choice Producent
\end{checkboxes}
\vspace{10mm}

\question Vad innebär fotosyntes? \textbf{(1 poäng)}
\vspace{5mm}
\begin{checkboxes}
    \choice Att växter andas in syre och släpper ut koldioxid
    \CorrectChoice Att växter omvandlar ljusenergi, koldioxid och vatten till syre och energirik näring
    \choice Att djur bryter ner organiskt material
    \choice Att växter och djur lever i symbios
\end{checkboxes}
\vspace{10mm}

\question Vilket alternativ beskriver en näringsväv? \textbf{(1 poäng)}
\vspace{5mm}
\begin{checkboxes}
    \choice En enkel kedja av vem som äter vem
    \choice En översikt över abiotiska faktorer
    \CorrectChoice Ett komplext nätverk av näringskedjor i ett ekosystem
    \choice Ett område där alla organismer har samma nisch
\end{checkboxes}
\vspace{10mm}
\break

\question Vad är ett exempel på en biotisk faktor? \textbf{(1 poäng)}
\vspace{5mm}
\begin{checkboxes}
    \choice Jordmån
    \choice Solljus
    \CorrectChoice Rovdjur
    \choice Temperatur
\end{checkboxes}
\vspace{10mm}

\question Vad är en förstahandskonsument? \textbf{(1 poäng)}
\vspace{5mm}
\begin{checkboxes}
    \CorrectChoice En växtätare
    \choice En växt
    \choice En nedbrytare
    \choice En toppkonsument
\end{checkboxes}
\vspace{10mm}
\question Förklara skillnaden mellan habitat och ekologisk nisch. \textbf{(2 poäng)}
\vspace{30mm}

\question Vad innebär energipyramiden? \textbf{(1 poäng)}
\vspace{5mm}
\begin{checkboxes}
    \choice Energimängden ökar ju högre upp i de trofiska nivåerna man kommer
    \choice All energi överförs från en nivå till nästa
    \CorrectChoice Energin minskar ju högre upp i de trofiska nivåerna man kommer
    \choice Energin är konstant i alla nivåer
\end{checkboxes}
\vspace{10mm}

\question Vilket begrepp beskriver en arts "adress" i ett ekosystem? \textbf{(1 poäng)}
\vspace{5mm}
\begin{checkboxes}
    \CorrectChoice Habitat
    \choice Ekologisk nisch
    \choice Population
    \choice Organismsamhälle
\end{checkboxes}
\break
\question Varför är biologisk mångfald viktig för ett ekosystem? \textbf{(2 poäng)}
\vspace{30mm}
\question Vad är ett exempel på en invasiv art i Sverige? \textbf{(1 poäng)}
\vspace{5mm}
\begin{checkboxes}
    \choice Ekoxe
    \choice Varg
    \choice Gran
    \CorrectChoice Mördarsnigel
\end{checkboxes}
\vspace{10mm}

\question Vad kallas växterna i en näringskedja? \textbf{(1 poäng)}
\vspace{5mm}
\begin{checkboxes}
    \choice Konsumenter
    \choice Nedbrytare
    \choice Toppkonsumenter
    \CorrectChoice Producenter
\end{checkboxes}
\vspace{10mm}

\question Vilken skogstyp är vanligast i Sverige? \textbf{(1 poäng)}
\vspace{5mm}
\begin{checkboxes}
    \choice Lövskog
    \choice Urskog
    \choice Kulturskog
    \CorrectChoice Barrskog
\end{checkboxes}
\vspace{10mm}

\break

\vspace{5mm} %5mm vertical space
\begin{center}
\fbox{\fbox{\parbox{6in}{\centering
\textbf{Fördjupande frågor}: svara mer utförligt (\textbf{7 poäng})
}}}
\end{center}

\question Vilken trofisk nivå bör människor konsumera för att minska sin klimatpåverkan? Förklara varför. \textbf{(3 poäng)}
\vspace{90mm}

\question På vilket sätt påverkar människans populationstillväxt ekosystemens bärkraft och den biologiska mångfalden? \textbf{(4 poäng)}
\vspace{40mm}

\end{questions}
\end{document}
