\documentclass[12pt]{article}
\usepackage[utf8]{inputenc}
\usepackage[swedish]{babel}
\usepackage{amsmath,amssymb}
\usepackage{enumitem}
\usepackage{fancyhdr}
\usepackage{geometry}
\usepackage{xcolor}
\usepackage{tabularx}
\usepackage{booktabs}
\usepackage{colortbl}
\usepackage{longtable}
\usepackage{csquotes}
\usepackage[style=apa, backend=biber, sorting=ynt]{biblatex}
\DeclareLanguageMapping{swedish}{swedish-apa}
\addbibresource{references.bib}
\geometry{a4paper, margin=2.5cm}

\title{Lektionsplanering: Artificiell Intelligens 1}
\author{Viktor Arohlén}
\date{28 maj 2025}

\begin{document}

\maketitle

\section*{Tema: AI och rättvisa – Vem får jobbet?}

\section*{Syfte och bakgrund}

Syftet med denna lektion är att introducera och problematisera etiska aspekter av 
artificiell intelligens (AI) utifrån konkreta exempel som är relevanta för elevernas 
vardag och framtida samhällsengagemang.
\\\\
Lektionen är en del av kursen \textit{Artificiell intelligens 1}, som ges som 
individuellt val och är öppen för elever från såväl samhällsvetenskapsprogrammet som 
naturvetenskapsprogrammet.
\\\\
Eftersom inga särskilda tekniska förkunskaper krävs för kursen — utöver grundläggande 
matematik — är undervisningen utformad för att vara tillgänglig för alla elever oavsett 
tidigare erfarenhet av teknik eller programmering. 
\\\\
Lektionsinnehållet anknyter till det centrala innehållet i kursen, särskilt punkterna:
\begin{itemize}
    \item Etiska dilemman med användandet av AI, däribland transparens.
    \item Demokratiska, sociala, ekonomiska, miljömässiga och säkerhetsmässiga 
          möjligheter och risker med AI-användning samt dess konsekvenser för samhället.
\end{itemize}
Genom att utgå från exempel som AI-användning vid rekrytering får eleverna möjlighet 
att koppla abstrakta begrepp som \textit{algoritmisk bias}, \textit{transparens} och 
\textit{ansvar} till konkreta situationer de kan relatera till. 
\\\\
Momenten som beskrivs i planeringen kan också fördjupas och tas vidare framförallt med hjälp av verkliga
exempel så som \citeyear{dastin2018amazon} när \citeauthor{dastin2018amazon} hittade bias mot kvinnor. Det går även att ta
vidare momenten genom att titta på den andra delen av rekryteringsperspektiv ur ett etik och AI-perspektiv: generering
av personligt brev och CV.

\break

\section*{Lektion 1 (70 minuter)}

% Använder färgad tabell med tydligare struktur
\renewcommand{\arraystretch}{1.3} % Ökar radavståndet i tabellen

\begin{table}[h!]
\centering
\begin{tabularx}{\textwidth}{|p{2cm}|p{4cm}|p{3cm}|X|}
\hline
\rowcolor{gray!20} \textbf{Tid} & \textbf{Aktivitet} & \textbf{Arbetsform} & \textbf{Beskrivning} \\
\hline
0--5 min & \textbf{Introduktion} & Lärargenomgång &Kort presentation av lektionens mål och koppling till kursens centrala innehåll.  \\
\hline
\rowcolor{gray!10} 5--10 min & \textbf{Gruppuppdelning och jobbannons} & Smågrupper (3--4 elever) & Eleverna delas upp i grupper och granskar en jobbannons (Bilaga 1). \\
\hline
5-10 min & \textbf{Granska kandidater} & Smågrupper (3--4 elever) & Eleverna granskar kandidaterna (Bilaga 2) och bedömmer vem som är mest lämpad för tjänsten. \\
\hline
\rowcolor{gray!10} 5--10 min & \textbf{AI-bedömning av kandidater 1} & Smågrupper (3--4 elever) & Eleverna granskar AI poängbedömning av kandidaterna (Bilaga 3) och bedömmer återigen vem som är mest lämpad för tjänsten. \\
\hline
5--10 min & \textbf{AI-bedömning av kandidater 2} & Smågrupper (3--4 elever) & Eleverna granskar förklaringen till AIs bedömning (Bilaga 4). Diskuterar åteigen vilken kandidat som är lämpligast. \\
\hline
\rowcolor{gray!10} 10--15 min & \textbf{Gemensam diskussion} & Helklass & Eleverna diskuterar i helklass vem som är den rimligaste kandidaten och hur AIs bedömningar påverkade dem. \\
\hline
5--10 min & \textbf{Exit ticket} & Individuell & Kort utvärdering kring vad de tar med sig kring uppgiften. \\
\hline
\end{tabularx}
\end{table}
\break

\section*{Lektion 2 (70 minuter)}

\begin{table}[h!]
    \centering
    \begin{tabularx}{\textwidth}{|p{2cm}|p{4cm}|p{3cm}|X|}
    \hline
    \rowcolor{gray!20} \textbf{Tid} & \textbf{Aktivitet} & \textbf{Arbetsform} & \textbf{Beskrivning} \\
    \hline
    5 min & \textbf{Introduktion och repetition} & Lärargenomgång &Kort repetition om  tidigare lektion och information om vad som görs idag.  \\
    \hline
    \rowcolor{gray!10} 10-20 min & \textbf{Video} & Helklass & Tittar på video gemensamt som tar upp bias och AI vid rekrytering (se material och resurser). \\
    \hline
    5-10 min & \textbf{Parvis reflektion} & Parvis & Elever reflekterar över användning av AI vid rekrytering och dess konsekvener. \\
    \hline
    \rowcolor{gray!10} 5--10 min & \textbf{Klassreflektion} & Helklass & Gemensam reflektion utifrån samtal (ej obligatorisk) \\
    \hline
    30--40 min & \textbf{Genomgång av begrepp och teorier} & Helklass & Genomgång av relevanta begrepp för bias och AI, så som bias, transparens, algoritm etc. (kan knyta an till tidigare lektioner) \\
    \hline
    \end{tabularx}
    \end{table}
    

\subsection*{Material och resurser}

\begin{table}[h!]
\centering
\begin{tabularx}{\textwidth}{|p{4cm}|X|}
\hline
\rowcolor{gray!20} \textbf{Resurs} & \textbf{Beskrivning} \\
\hline
\rowcolor{gray!10} Nyhetsklipp från CBS & https://www.youtube.com/watch?v=TP8X-YW1Jr8 \\
\hline
Bias och AI från Google & https://youtu.be/59bMh59JQDo?si=fsG1Joj1NYTQfhmk \\
\hline
\rowcolor{gray!10} Tidningsartikel om Amazons AI & https://www.reuters.com/article/us-amazon-com-jobs-automation-insight-idUSKCN1MK08G \\
\hline
Jobbannons, kandidater och AI-bedömning & Se bilagor för material till lektionen \\
\hline
\end{tabularx}
\end{table}

\break
\subsection*{Kommentarer till lektionsmoment}

Lektionerna syftar som ett avstamp i diskussioner kring etik, bias, och kontroll gällande AI. Tanken är att lektionerna ligger
efter grundläggande begrepp har gåtts genom, men relevanta begrepp bör också repeteras. Den främsta tanken är att ställa elever mot
etiska dilemman de kan relatera till. Framförallt genom att ge dem en uppgift helt utan AI och sedan långsamt inkorporera AI
i uppgiften är tanken att de ska få nya insikter.
\\ \\
Den första lektionen är tänk att enbart utgå från det här laborativa arbetssättet och eftersom jag heller inte haft möjlighet att genomföra momentet
finns det säkert flertalet fallgropar som missats. Utifrån seminariet fick jag dock positiv feedback från mina studiekamrater gällande att jobba med rollspel
och att inte utgå från AI först, utan att istället jobba med människor och sedan ta in AI.
\\ \\
Det andra tillfället syftar istället till att problematisera de etiska dilemman med AI och rekrytering genom att titta på verkliga exempel. Således kan de knyta an
de rollspelsövningar vi gjort till verkligheten och inse att vi arbetat med något som faktiskt är aktuellt idag. Överlag är det här en mer teoretisk lektion där fokus ligger
på förståelse för begrepp så som bias, etik och autonoma agenter.
\\ \\
När det gäller att ta vidare planeringen beror det till stor del på elevunderlaget. Under etikföreläsningarna med Johan tog han upp både statistiska metoder för att upptäcka bias,
men också hur filosofiska teorier kan användas för att identifiera orättvisor. Beroende på hur mycket eleverna har kunskaper om antingen statistiska metoder eller filosofi kan man därför ta det
i två olika riktningar när det kommer till att identifera problemen med AI. Där exempelvis naturelever får fördjupa sig inom de statistiska metoder och hur man kan upptäcka bias,
medan samhällslever kan ta en filosofiskt perspektiv och titta på hur man kan identifiera orättvisor i AI-system.
\\ \\
Vidare hade ännu fler verkliga exempel kunnat användas och exempelvis hade fler verkliga etiska dilemman kunnat introduceras. Närmast till hands ligger troligtvis
självkörande bilar där exempelvis \textcite{moralmachine} hade kunnat användas och skapa diskussioner i ämnet.
\\ \\
Utöver det finns också möjlighet att helt stanna kvar i rekryteringsbias och kanske även anspela på andra skolämnen. Algoritmisk bias är ett stort område och ännu mer exempel och material hade kunnat hämtats från \textcite{barocas2023fairness}
beroende på tidsutrymme, elevgrupp och mer erfarenhet.

\break
\subsection*{AI-användning i examinationsuppgiften}

I denna lektionsplanering har AI använts som ett stödjande verktyg på flera sätt. Jag har använt AI både i form av chattbottar och som integrerad funktion i utvecklingsmiljön Windsurf (en fork av VSCode). 

AI har främst använts för att:
\begin{itemize}
  \item Hjälpa till med formatering och strukturering av LaTeX-dokumentet
  \item Generera material för bilagorna (jobbannons, kandidatprofiler och AI-bedömningar)
  \item Stötta med implementering av APA-referenssystemet
\end{itemize}

Det är värt att notera att all löpande text i lektionsplaneringen är skriven av mig, medan AI har fungerat som ett verktyg för att effektivisera arbetsprocessen och förbättra den tekniska kvaliteten på dokumentet.

\printbibliography[title=Referenser]

\appendix
\break
\section*{Bilagor}

\subsection*{Bilaga 1: Rekryteringsfall: Vem får jobbet?}

\begin{minipage}{\textwidth}
\textbf{Tjänst:} Kundkoordinator på Evenemangsgruppen AB

\textbf{Företagsbeskrivning:}\\  
Evenemangsgruppen AB är ett mellanstort företag som arbetar med att organisera och koordinera kultur- och idrottsevenemang runtom i Sverige. Företaget söker nu en ny kundkoordinator som ska hantera kontakter med kunder, bokningar och enklare projektledning.
\\ \\
\textbf{Önskad profil:}  
\begin{itemize}[leftmargin=*,noitemsep]
\item God kommunikationsförmåga
\item Erfarenhet av att arbeta i team
\item Grundläggande datakunskaper (t.ex. e-post, kalkylprogram)
\item Initiativtagande och lösningsorienterad
\end{itemize}
\end{minipage}

\begin{minipage}{\textwidth}
\vspace{10mm}
\textbf{Uppgift till eleverna:}
Läs om kandidaterna på nästa sida. Välj vem du tycker ska få jobbet – utan att ta hänsyn till AI:s bedömning.
\end{minipage}

\break

\subsection*{Bilaga 2: Kandidater (Steg 1)}

\begin{minipage}{\textwidth}
\textbf{Instruktion:} Läs igenom kandidaternas profiler nedan och bedöm vem som är mest lämpad för tjänsten.
\end{minipage}

\vspace{0.5em}

\begin{longtable}{>{\bfseries}p{2.8cm} p{9cm}}

Namn: & \textbf{Sara Andersson} \\
Ålder: & 28 år \\
Utbildning: & Kandidat i beteendevetenskap \\
Arbetslivserfarenhet: & \begin{minipage}[t]{9cm}Tre års erfarenhet som receptionist på ett gym, där hon även ansvarade för bokningar och kundkontakt.\end{minipage} \\
Övrigt: & Arbetat ideellt på ungdomsgård. Pratar svenska, engelska och arabiska. \\[1.2em]

Namn: & \textbf{Jonas Björk} \\
Ålder: & 24 år \\
Utbildning: & Gymnasieexamen i ekonomi \\
Arbetslivserfarenhet: & \begin{minipage}[t]{9cm}Har jobbat två somrar som administrativ assistent på ett mindre företag.\end{minipage} \\
Övrigt: & Aktiv i föreningslivet. Gillar att planera resor och evenemang i privatlivet. \\[1.2em]

Namn: & \textbf{Fatima Choudhury} \\
Ålder: & 34 år \\
Utbildning: & Omskolad via Komvux till administratör \\
Arbetslivserfarenhet: & \begin{minipage}[t]{9cm}Tidigare undersköterska. Nu praktik och extrajobb inom administration på en skola.\end{minipage} \\
Övrigt: & Tvåbarnsmamma. Har hållit i skolaktiviteter och föräldramöten. \\[1.2em]

Namn: & \textbf{Erik Dahl} \\
Ålder: & 30 år \\
Utbildning: & Kandidat i medie- och kommunikationsvetenskap \\
Arbetslivserfarenhet: & \begin{minipage}[t]{9cm}Jobbat på kundtjänst i fem år, där han också coachat nya medarbetare.\end{minipage} \\
Övrigt: & Gillar struktur och har god datavana. \\

\end{longtable}

\break

\subsection*{Bilaga 3: AI-poängsättning (Steg 2)}

\begin{minipage}{\textwidth}
\textbf{Instruktion till eleverna:}\\  
AI-systemet har analyserat CV:n och fotografier för varje kandidat. Här är systemets poängsättning från 0 till 100. Fundera på om du vill omvärdera ditt val.
\end{minipage}

\vspace{0.5em}

\begin{longtable}{>{\bfseries}p{5cm} p{3cm}}

Sara Andersson & 72 \\ 
Jonas Björk & 80 \\
Fatima Choudhury & 60 \\
Erik Dahl & 85 \\

\end{longtable}

\break

\subsection*{Bilaga 4: AI:s bedömningsgrunder (Steg 3)}

\begin{minipage}{\textwidth}
\textbf{Instruktion till eleverna:}\\  
Nedan följer AI-systemets tolkning av varför varje kandidat fick sin poäng. Diskutera: Håller du med? Är det rimligt att ett AI-system fattar sådana beslut?
\end{minipage}

\vspace{0.5em}

\begin{longtable}{>{\bfseries}p{3cm} p{10cm}}

Sara Andersson & God erfarenhet av kundkontakt, men AI rankade hennes erfarenhet från gymbranschen något lägre än kontorsmiljö. Bonuspoäng för flerspråkighet. \\[0.7em]

Jonas Björk & Trots begränsad erfarenhet matchade hans intressen och tidigare roller väl med AI-modellens data om lyckade anställningar. \\[0.7em]

Fatima Choudhury & AI nedvärderade hennes erfarenhet utanför kontorsmiljö och hade få referensfall i modellen på liknande karriärvägar. \\[0.7em]

Erik Dahl & Fick hög poäng för lång erfarenhet inom kundservice samt att ha coachat andra, vilket tolkas som ledarskapspotential. \\

\end{longtable}

\end{document}

