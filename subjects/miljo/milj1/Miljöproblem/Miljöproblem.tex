\documentclass{exam}
\usepackage{graphicx} 

% Format Header and footer
\pagestyle{headandfoot}
\header{\footnotesize Klass:\\Namn:}{\Large\textbf{Miljöproblem och dess effekter}}{\footnotesize HÅLMIJ0 - 2024\\Viktor Arohlén}
\headrule
\footrule
\setlength{\columnsep}{0.25cm}
\footer{}{Sida \thepage}{}

\begin{document}
\section*{Instruktioner}
Provet består av två delar \\
    - Grundläggande frågor, flervalsfrågor (\textit{12 poäng})\\
    - Fördjupande frågor, svara mer omfattande (\textit{8 poäng})

\subsection*{Poäng}
Antalet poäng är markerat för varje fråga. Totalt \textbf{14 frågor} och \textbf{20 poäng}.\\ \textit{För godkänt resultat krävs 9 poäng.}

\vspace{5mm}
\begin{center}
\fbox{\fbox{\parbox{6in}{\centering
\textbf{Del 1}: Flervalsfrågor (\textbf{12 poäng})
}}}
\end{center}
\begin{questions}
\vspace{5mm} 

\question Vilket påstående om växthuseffekten är korrekt? (\textbf{1 poäng})
\begin{checkboxes}
    \choice Den naturliga växthuseffekten är skadlig för livet på jorden
    \choice Den förstärkta växthuseffekten orsakas främst av naturliga processer
    \CorrectChoice Den naturliga växthuseffekten håller jordens medeltemperatur omkring 15°C
    \choice Utan växthuseffekten skulle jorden vara varmare än idag
\end{checkboxes}

\vspace{5mm} 
\hrule 
\vspace{5mm} 

\question Vad menas med "jordens andetag"? (\textbf{1 poäng})
\begin{checkboxes}
    \choice Dagliga variationer i koldioxidhalten
    \CorrectChoice Årstidsvariationer i koldioxidhalten på grund av växternas fotosyntes
    \choice Havens upptag och avgivning av koldioxid
    \choice Vulkanutbrott som släpper ut koldioxid
\end{checkboxes}

\vspace{5mm} 
\hrule 
\vspace{5mm} 

\question Vilket påstående om UV-strålning är korrekt? (\textbf{1 poäng})
\begin{checkboxes}
    \choice UV-C är den minst skadliga typen av UV-strålning
    \CorrectChoice UV-B behövs för D-vitaminproduktion men kan orsaka hudcancer
    \choice UV-A tränger inte igenom atmosfären
    \choice All UV-strålning är skadlig för människor
\end{checkboxes}

\vspace{5mm} 
\hrule 
\vspace{5mm} 

\question Vilken funktion har ozonskiktet i stratosfären? (\textbf{1 poäng})
\begin{checkboxes}
    \choice Det reglerar jordens temperatur
    \CorrectChoice Det absorberar skadlig UV-strålning
    \choice Det producerar syre till atmosfären
    \choice Det förhindrar luftföroreningar
\end{checkboxes}

\break
\vspace{5mm} 
\question Vilket påstående om marknära ozon är korrekt? (\textbf{1 poäng})
\begin{checkboxes}
    \choice Det är en viktig del av det skyddande ozonskiktet
    \choice Det bildas genom naturliga processer i marken
    \CorrectChoice Det är en luftförorening som kan skada lungor och växtlighet
    \choice Det hjälper till att rena luften från föroreningar
\end{checkboxes}

\vspace{5mm} 
\hrule 
\vspace{5mm} 

\question Hur påverkar partikelstorlek luftföroreningars hälsoeffekter? (\textbf{1 poäng})
\begin{checkboxes}
    \choice Större partiklar är farligare eftersom de väger mer
    \CorrectChoice Ultrafina partiklar är farligast eftersom de kan ta sig in i blodomloppet
    \choice Partikelstorleken har ingen betydelse för hälsoeffekterna
    \choice Mellanstora partiklar är farligast eftersom de fastnar i lungorna
\end{checkboxes}

\vspace{5mm} 
\hrule 
\vspace{5mm} 

\question Vilken är den huvudsakliga källan till kväveoxider i stadsmiljö? (\textbf{1 poäng})
\begin{checkboxes}
    \choice Jordbruk
    \CorrectChoice Vägtrafik
    \choice Naturlig kvävefixering
    \choice Industriprocesser
\end{checkboxes}

\vspace{5mm} 
\hrule 
\vspace{5mm} 

\question Vilket påstående om försurning är korrekt? (\textbf{1 poäng})
\begin{checkboxes}
    \choice Försurning påverkar endast vattenmiljöer
    \choice Naturlig försurning sker snabbare än antropogen försurning
    \CorrectChoice Försurning kan leda till utlakning av metaller i marken
    \choice Kalkning av sjöar förvärrar försurningsproblemen
\end{checkboxes}

\vspace{5mm} 
\hrule 
\vspace{5mm} 

\question Vilken av dessa reaktioner leder till försurning? (\textbf{1 poäng})
\begin{checkboxes}
    \choice $\mathrm{2KOH + CO_2 \rightarrow K_2CO_3 + H_2O}$
    \CorrectChoice $\mathrm{2SO_2 + 2H_2O + O_2 \rightarrow 2H_2SO_4}$
    \choice $\mathrm{CaCO_3 + H_2O \rightarrow Ca(OH)_2 + CO_2}$
    \choice $\mathrm{NaCl + H_2O \rightarrow Na^+ + Cl^- + H_2O}$
\end{checkboxes}

\break
\vspace{5mm} 
\question Vilken åtgärd är mest effektiv för att minska övergödning? (\textbf{1 poäng})
\begin{checkboxes}
    \choice Öka användningen av fossila bränslen
    \CorrectChoice Minska näringsläckage från jordbruk
    \choice Öka användningen av vägsalt
    \choice Plantera mer barrskog
\end{checkboxes}

\vspace{5mm} 
\hrule 
\vspace{5mm} 

\question Vad kännetecknar ett övergött vattendrag? (\textbf{1 poäng})
\begin{checkboxes}
    \choice Ökad biologisk mångfald
    \choice Minskad algblomning
    \CorrectChoice Syrebrist i bottenvattnet
    \choice Klarare vatten
\end{checkboxes}

\vspace{5mm} 
\hrule 
\vspace{5mm} 

\question Vilket påstående om miljögifter är korrekt? (\textbf{1 poäng})
\begin{checkboxes}
    \choice Alla gifter bryts ned naturligt inom ett år
    \choice Bioackumulation betyder att gifter späds ut i näringskedjan
    \CorrectChoice Fettlösliga gifter kan ansamlas i näringskedjan
    \choice Vattenlösliga gifter är alltid farligare än fettlösliga
\end{checkboxes}

\break

\begin{center}
\fbox{\fbox{\parbox{6in}{\centering
\textbf{Del 2}: Fördjupande frågor (\textbf{8 poäng})
}}}
\end{center}

\question
Förklara hur den naturliga växthuseffekten fungerar och på vilket sätt den skiljer sig från den förstärkta växthuseffekten. (\textbf{2 poäng}) \\ 
\\ Diskutera även de etiska aspekterna kring utvecklade länders historiska utsläpp i relation till utvecklingsländers rätt till ekonomisk utveckling. (\textbf{2 poäng})
\vspace{80mm}

\question
Redogör för ozonets dubbla roll i miljön. Jämför ozonets funktion och påverkan i stratosfären med dess effekter som luftförorening på marknivå. (\textbf{2 poäng}) \\ 
\\ Inkludera även en diskussion om hur hanteringen av ozonförtunningen kan ge lärdomar för arbetet med andra globala miljöproblem. (\textbf{2 poäng})
\vspace{60mm}

\end{questions}

\end{document}