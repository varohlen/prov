\documentclass{exam}
\usepackage{graphicx} 


%Format Header and footer
\pagestyle{headandfoot}
\header{\footnotesize Klass:\\Namn:}{\Large\textbf{Evolution}}{\footnotesize  BIOBIO01 - 2024\\Viktor Arohlén}
\headrule
\footrule
\setlength{\columnsep}{0.25cm}
%\setlength{\columnseprule}{1pt}
\footer{}{Sida \thepage}{}
%\extrafootheight{-2cm}

\begin{document}
\vspace{5mm} %5mm vertical space
\begin{center}
\fbox{\fbox{\parbox{6in}{\centering
\textbf{Grundläggande frågor}: svara kortfattat (bedöms på E- till C-nivå)
}}}
\end{center}
\vspace{5mm} %5mm vertical space
\begin{questions}
\question Förklara följande begrepp

\begin{itemize}
  \item Mutation 
  \vspace{10mm}
  \item Naturligt urval
  \vspace{10mm}
  \item Selektionstryck
  \vspace{10mm}
  \item Hybridisering
\end{itemize}
\vspace{10mm} %5mm vertical space
\question
Ge tre exempel på hur vi kan bevisa evolutionsteorin
\vspace{30mm} 
\question
Om vi vill studera evolutionen, vilka arter bör vi studera? Motivera!
\vspace{30mm} 
\question
Vilket av följande begrepp beskriver situationen där en fjärils färgmorf (färgvariant) är mer attraktiv för en partner, vilket leder till att färgmorfen blir vanligare i populationen?
\vspace{5mm}
\begin{checkboxes}
    \choice sexuell selektion
    \choice genetisk drift
    \choice flaskhalseffekt
    \choice naturligt urval
\end{checkboxes}
\break
\question 
Vilket av följande alternativ förklarar bäst begreppet samevolution?
\vspace{5mm}
\begin{checkboxes}
\choice En egenskap hos en art förs vidare trots att den egentligen inte fyller någon funktion.

\choice Liknande egenskaper hos arter som inte har en gemensam föregångare (förfader) som har denna egenskap.

\choice En art utvecklar på nytt en egenskap som försvunnit i ett tidigare skeda av artens evolution.

\choice En arts egenskaper utvecklas i samspel med en annan art.
\end{checkboxes}

\vspace{10mm}
\question
Vilket typ av urval visualiserar grafen nedan? Ge ett exempel på när ett sådant urval sker

\begin{figure}[h]
    \centering
    \includegraphics[width=0.7\linewidth]{06f57399-163d-4981-a6b3-9370f8174f29.png}
\end{figure}
\break
\vspace{5mm} %5mm vertical space
\begin{center}
\fbox{\fbox{\parbox{6in}{\centering
\textbf{Fördjupande frågor}: svara mer utförligt (bedöms på E- till A-nivå)
}}}
\end{center}
\question En stor problematik idag är multiresistenta bakterier, det vill säga bakterier som är okänsliga för flera olika sorters antibiotika. För hundra år sedan existerade knappt sådana bakterier. Förklara med hjälp av naturvetenskapliga evolutionära begrepp hur problematiken har utvecklas.
\vspace{50mm}
\question
Valar, delfiner och hajar är alla stora djur som lever i havet. Hur liknar de varandra och hur skiljer de sig från varandra? Förklara med evolutionära begrepp

Hjälplista:\\
Divergent och konvergent evolution\\
Däggdjur, broskfiskar\\
\begin{figure}[h]
    \centering
    \includegraphics[width=0.4\linewidth]{download.jpeg}
\end{figure}

\break

\begin{figure}[h]
    \centering
    \includegraphics[width=0.7\linewidth]{Screenshot 2024-03-25 12.02.23.png}
\end{figure}
\question
Ovan ser ni fyra djurs spår i den svenska skogen\\
1. Ekorre\\
2. Björn \\
3. Älg \\
4. Lodjur \\
\\
Utifrån spåren, diskutera hur djuren har anpassat sig till sin levnadsmiljö och varandra. Använd naturvetenskapliga och evolutionära begrepp. 
\end{questions}





\end{document}
