\documentclass{exam}
\usepackage{graphicx} 


%Format Header and footer
\pagestyle{headandfoot}
\header{\footnotesize Klass:\\Namn:}{\Large\textbf{Läxförhör: Liv och celler}}{\footnotesize Bio1\\2024}
\headrule
\footrule
\setlength{\columnsep}{0.25cm}
%\setlength{\columnseprule}{1pt}
\footer{}{Sida \thepage}{}
%\extrafootheight{-2cm}

\begin{document}
\vspace{5mm} %5mm vertical space
\begin{center}
\fbox{\fbox{\parbox{6in}{\centering
\vspace{1mm}
\textbf{8 frågor} och \textbf{13 poäng} \\ Flera alternativ kan vara korrekt på kryssfrågor. \\ Svara kortfattat.
\vspace{1mm}
}}}
\end{center}
\vspace{5mm} %5mm vertical space

\begin{questions}

\question Vilka av följande meningar är en \textbf{prövbar hypotes}? (1 poäng)
\begin{checkboxes}
   \choice Finns det liv i rymden?
   \choice Luktärt växer snabbare i en varmare miljö.
   \choice Alla som heter Kim är kvinnor.
   \choice Det finns fler än ett universum.
\end{checkboxes}

\vspace{5mm} 
\hrule 
\vspace{5mm} 

\question Vad innebär det \textbf{morfologiska perspektivet} gällande artbegreppet? (1 poäng)
\begin{checkboxes}
   \choice Att man studerar utseende och beteende hos en individ för att avgöra vilken art det tillhör
   \choice Att man studerar en individs DNA för att avgöra vilken art den tillhör
   \choice Att man skiljer på arter genom deras geografiska position och miljö
   \choice Att individer som kan få avkomma som är fertil är samma art
\end{checkboxes}

\vspace{5mm} 
\hrule 
\vspace{5mm} 

\question Förklara följande begrepp och ge ett exempel på en art som du skulle beskriva med begreppet: (2 poäng)
    \vspace{5mm} 
\begin{itemize}
     \item \textbf{Autotrof}
     \vspace{10mm} 
     \item \textbf{Heterotrof}
     \vspace{10mm} 
\end{itemize}

\vspace{5mm} 
\hrule 
\vspace{5mm} 

\question Vad innebär att ett ämne är \textbf{organiskt} ? (1 poäng)
\begin{checkboxes}
   \choice En kemisk förening som finns hos levande organismer
   \choice En kemisk förening som innehåller grundämnet kol
   \choice Det är ett äldre begrepp för att beskriva biokemi
   \choice Det är ett samlingsbegrepp för alla kemiska ämnen som finns i livsmedel
\end{checkboxes}
\vspace{5mm} 
\hrule 
\vspace{5mm} 

\question Vilka av följande arter har \textbf{eukaryota celler}? (1 poäng)
\begin{checkboxes}
   \choice Homo sapiens (Människa)
   \choice Yersinia pestis (Bakterie)
   \choice Canthareluus cibarius (Kantarell)
   \choice Helianthus anuus (Solros)
\end{checkboxes}

\break


\vspace{5mm} 
\vspace{5mm}
\question Nämn \textbf{3 organeller} och beskriv deras funktion kortfattat (3 poäng):
\vspace{40mm} 
\hrule 
\vspace{5mm}
\question 
Vilka av följande grundämnen är viktiga för levande organismer på jorden? (1 poäng)
\begin{checkboxes}
   \choice Kol
   \choice Guld
   \choice Uran
   \choice Kväve
\end{checkboxes}
\vspace{5mm}
\hrule 
\vspace{5mm}
\question
Carl von Linné skapade fyra riken för sin indelning av organismer på jorden: djur, svampar, växter och stenar.
\\ \\
Ett av de rikena kvalificeras inte som liv idag. Vilket och vad krävs för att något ska kvalificeras som liv? (3 poäng)

\break


\end{questions}

\end{document}

