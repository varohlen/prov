\documentclass{exam}
\usepackage{graphicx}

%Format Header and footer
\pagestyle{headandfoot}
\header{\footnotesize Klass:\\Namn:}{\Large\textbf{Biologi 1: Kursprov}}{\footnotesize 2025\\Viktor Arohlén}
\headrule
\footrule
\setlength{\columnsep}{0.25cm}
\footer{}{Sida \thepage}{}

\begin{document}

\section*{Instruktioner}
Provet består av två delar:\\
- Kryssfrågor, endast ett alternativ är rätt om inget annat anges (\textit{13 poäng})\\
- Fördjupande/frisvarsfrågor, svara mer omfattande (\textit{12 poäng})

\subsection*{Poäng}
Varje fråga har angivet poängvärde. Provet bedöms enbart efter poäng.\\

\vspace{5mm}
\begin{center}
\fbox{\fbox{\parbox{6in}{\centering
\textbf{Kryssfrågor}: markera endast ett alternativ, om inget annat anges. (\textbf{13 poäng})
}}}
\end{center}
\vspace{5mm}
\begin{questions}

% --- DNA & RNA ---
\question (1p) En \textbf{nukleotid} består av:
\begin{checkboxes}
   \choice Aminosyra, kvävebas och deoxiribos
   \choice Kvävebas, ribos och en fosfatgrupp
   \choice Deoxiribos, ribos och fosfatgrupper(er)
   \correctchoice Kvävebas, sockermolekyl (ribos eller deoxiribos) och fosfatgrupp(er)
\end{checkboxes}
\vspace{5mm}

\question (1p) Vilken av följande processer är \textbf{inte} en del av \textbf{proteinsyntesen}?
\begin{checkboxes}
    \choice Transkription
    \choice Translation
    \correctchoice Replikation
    \choice Splicing
\end{checkboxes}
\vspace{5mm}

\question (1p) Vilken av följande är en funktion av \textbf{tRNA}?
\begin{checkboxes}
\choice Att bära genetisk information från DNA till ribosomen
\correctchoice Att bära aminosyror till ribosomen
\choice Att bilda ribosomernas struktur
\choice Att katalysera kemiska reaktioner
\end{checkboxes}
\vspace{5mm}

\question (1p) Vilka av följande är \textbf{kvävebaspar i DNA}?
\begin{checkboxes}
   \choice Guanin - Tymin
   \choice Uracil - Tymin
   \correctchoice Adenin - Tymin
   \correctchoice Guanin - Cytosin
\end{checkboxes}
\vspace{5mm}
\break
\question (1p) Vad av följande stämmer \textbf{inte} för \textbf{RNA}?
\begin{checkboxes}
   \choice Innehåller ribos
   \choice Innehåller kvävebasen uracil
   \correctchoice Innehåller kvävebasen tymin
   \choice Består av en enkelsträng (helix)
\end{checkboxes}
\vspace{5mm}

\question (1p) Hur \textbf{binder aminosyror} till varandra?
\begin{checkboxes}
   \choice Vätebindningar
   \choice Jonbindning
   \correctchoice Peptidbindning
   \choice Kemisk bindning
\end{checkboxes}
\vspace{5mm}

% --- EVOLUTION ---
\question (1p) Vilket av följande begrepp beskriver bäst \textbf{naturligt urval}?
\begin{checkboxes}
    \choice Att individer förändras för att passa miljön
    \choice Att alla individer har lika stor chans att överleva
    \correctchoice Att de \textbf{bäst anpassade} överlever och för sina gener vidare
    \choice Att miljön alltid förändras
\end{checkboxes}
\vspace{5mm}

\question (1p) Vad menas med \textbf{selektionstryck}?
\begin{checkboxes}
    \choice Inre faktorer som påverkar individen
    \choice Att alla individer överlever lika bra
    \correctchoice \textbf{Yttre faktorer} som påverkar vilka individer som klarar sig bäst
    \choice Att arter alltid utvecklas mot ökad komplexitet
\end{checkboxes}
\vspace{5mm}

\question (1p) Vilket av följande begrepp beskriver situationen där en fjärils färgmorf (färgvariant) är mer attraktiv för en partner, vilket leder till att färgmorfen blir vanligare i populationen?
\begin{checkboxes}
    \correctchoice Sexuell selektion
    \choice Genetisk drift
    \choice Flaskhalseffekt
    \choice Naturligt urval
\end{checkboxes}
\vspace{5mm}

\question (1p) Vilket av följande alternativ förklarar bäst begreppet \textbf{samevolution}?
\begin{checkboxes}
\choice En egenskap hos en art förs vidare trots att den egentligen inte fyller någon funktion.
\choice Liknande egenskaper hos arter som inte har en gemensam föregångare (förfader) som har denna egenskap.
\choice En art utvecklar på nytt en egenskap som försvunnit i ett tidigare skede av artens evolution.
\correctchoice En arts egenskaper utvecklas i samspel med en annan art.
\end{checkboxes}
\vspace{5mm}
\break
\question (1p) Vad är den \textbf{huvudsakliga skillnaden} mellan \textbf{DNA} och \textbf{RNA}?
\begin{checkboxes}
    \choice DNA är enkelsträngat, RNA är dubbelsträngat
    \correctchoice DNA innehåller \textbf{deoxiribos} och \textbf{tymin}, RNA innehåller \textbf{ribos} och \textbf{uracil}
    \choice DNA finns bara i cytoplasman
    \choice RNA lagrar genetisk information permanent
\end{checkboxes}
\vspace{5mm}

\question (1p) Vilken \textbf{funktion} har \textbf{mRNA} i cellen?
\begin{checkboxes}
   \choice Bygger upp cellmembranet
   \choice Transporterar aminosyror till ribosomen
   \correctchoice För över genetisk information från DNA till ribosomen
   \choice Katalyserar kemiska reaktioner
\end{checkboxes}
\vspace{5mm}

\question (1p) Vilken av följande är ett exempel på \textbf{genetisk drift}?
\begin{checkboxes}
   \choice När de bäst anpassade individerna överlever och för sina gener vidare
   \choice När miljön förändras och påverkar vilka gener som är fördelaktiga
   \correctchoice När slumpen gör att vissa gener blir vanligare eller ovanligare i en liten population
   \choice När två arter utvecklas liknande egenskaper utan gemensam förfader
\end{checkboxes}
\vspace{5mm}

\end{questions}

\break

\begin{center}
\fbox{\fbox{\parbox{6in}{\centering
\textbf{Fördjupande/frisvarsfrågor}: svara utförligt på frågorna nedan. (\textbf{12 poäng})
}}}
\end{center}
\vspace{5mm}
\begin{questions}

\question (2p) Beskriv skillnaden mellan \textbf{DNA} och \textbf{RNA}.
\vspace{40mm}

\question (2p) Beskriv vad som sker under \textbf{transkription} och \textbf{translation} i cellen. Ge exempel på var i cellen processerna sker och vilka huvudaktörer som är inblandade.
\vspace{40mm}

\question (2p) Ge tre exempel på hur vi kan bevisa \textbf{evolutionsteorin}.
\vspace{40mm}

\newpage

\question (2p) Vad innebär \textbf{naturligt urval}?
\vspace{40mm}

\question (2p) Förklara varför vissa \textbf{mutationer} inte får någon effekt på individen.
\vspace{40mm}

\question (2p) Resonera kring hur \textbf{antibiotikaresistens} kan utvecklas hos bakterier och varför detta är ett växande problem.
\vspace{40mm}

\end{questions}

\end{document}
