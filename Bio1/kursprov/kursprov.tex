\documentclass{exam}
\usepackage{graphicx}

%Format Header and footer
\pagestyle{headandfoot}
\header{\footnotesize Klass:\\Namn:}{\Large\textbf{Biologi 1: Kursprov}}{\footnotesize 2025\\Viktor Arohlén}
\headrule
\footrule
\setlength{\columnsep}{0.25cm}
\footer{}{Sida \thepage}{}

\begin{document}

\section*{Instruktioner}
Provet består av två delar:\\
- Kryssfrågor, endast ett alternativ är rätt om inget annat anges (\textit{15 poäng})\\
- Fördjupande/frisvarsfrågor, svara mer omfattande (\textit{12 poäng})

\subsection*{Poäng}
Antalet poäng är markerat för varje fråga. Totalt \textbf{cirka 18 frågor} och \textbf{27 poäng}.\\ \textit{För godkänt resultat krävs 11 poäng.}

\vspace{5mm}
\begin{center}
\fbox{\fbox{\parbox{6in}{\centering
\textbf{Kryssfrågor}: markera endast ett alternativ, om inget annat anges. (\textbf{15 poäng})
}}}
\end{center}
\vspace{5mm}

\begin{questions}

% --- DNA & RNA ---
\question \textbf{Vad består en nukleotid av?}
\begin{checkboxes}
   \choice Aminosyra, \textbf{kv\"avebas} och deoxiribos
   \choice \textbf{Kvävebas}, ribos och en fosfatgrupp
   \choice Deoxiribos, ribos och fosfatgrupper(er)
   \correctchoice \textbf{Kvävebas}, sockermolekyl (\textbf{ribos} eller \textbf{deoxiribos}) och \textbf{fosfatgrupp(er)}
\end{checkboxes}
\vspace{5mm}
\hline

\question Vilken av följande processer är \textbf{inte} en del av \textbf{proteinsyntesen}?
\begin{checkboxes}
    \choice \textbf{Transkription}
    \choice \textbf{Translation}
    \correctchoice \textbf{Replikation}
    \choice \textbf{Splicing}
\end{checkboxes}
\vspace{5mm}
\hline

% --- NY VARIANT: DNA & RNA ---
\question Vad är den \textbf{huvudsakliga skillnaden} mellan \textbf{DNA} och \textbf{RNA}?
\begin{checkboxes}
    \choice DNA är enkelsträngat, RNA är dubbelsträngat
    \correctchoice DNA innehåller \textbf{deoxiribos} och \textbf{tymin}, RNA innehåller \textbf{ribos} och \textbf{uracil}
    \choice DNA finns bara i cytoplasman
    \choice RNA lagrar genetisk information permanent
\end{checkboxes}
\vspace{5mm}
\hline

% --- EVOLUTION ---
\question Vilket av följande begrepp beskriver bäst \textbf{naturligt urval}?
\begin{checkboxes}
    \choice Att individer förändras för att passa miljön
    \choice Att alla individer har lika stor chans att överleva
    \correctchoice Att de \textbf{bäst anpassade} överlever och för sina gener vidare
    \choice Att miljön alltid förändras
\end{checkboxes}
\vspace{5mm}
\hline

% --- NY VARIANT: EVOLUTION ---
\question Vad menas med \textbf{selektionstryck}?
\begin{checkboxes}
    \choice Inre faktorer som påverkar individen
    \choice Att alla individer överlever lika bra
    \correctchoice \textbf{Yttre faktorer} som påverkar vilka individer som klarar sig bäst
    \choice Att arter alltid utvecklas mot ökad komplexitet
\end{checkboxes}
\vspace{5mm}
\hline

\question Vilket av följande påståenden om \textbf{mutationer} är sant?
\begin{checkboxes}
    \choice Alla mutationer leder till sjukdom
    \choice Mutationer kan aldrig gå i arv
    \correctchoice En \textbf{mutation} kan vara positiv, negativ eller neutral
    \choice Mutationer sker bara hos djur
\end{checkboxes}
\vspace{5mm}
\hline

\end{questions}

\break

\vspace{5mm}
\begin{center}
\fbox{\fbox{\parbox{6in}{\centering
\textbf{Fördjupande/frisvarsfrågor}: svara utförligt på frågorna nedan. (\textbf{12 poäng})
}}}
\end{center}
\vspace{5mm}

\begin{questions}

\question \textbf{Beskriv skillnaden mellan DNA och RNA.} (2 poäng)
\vspace{40mm}

\question \textbf{Ge ett exempel på hur en mutation kan påverka en organism.} (2 poäng)
\vspace{40mm}

\question \textbf{Ge två exempel på hur vi kan bevisa evolutionsteorin.} (2 poäng)
\vspace{40mm}

\question \textbf{Vad innebär naturligt urval?} (2 poäng)
\vspace{40mm}

\question \textbf{Förklara varför vissa mutationer inte får någon effekt på individen.} (2 poäng)
\vspace{40mm}

\question \textbf{Vad menas med "hybridisering" inom evolutionen?} (2 poäng)
\vspace{40mm}

\end{questions}

\end{document}
