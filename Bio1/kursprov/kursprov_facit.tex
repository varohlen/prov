\documentclass{exam}
\usepackage{graphicx}
\usepackage{xcolor}

%Format Header and footer
\pagestyle{headandfoot}
\header{\footnotesize Klass:\\Namn:}{\Large\textbf{Biologi 1: Kursprov - FACIT}}{\footnotesize 2025\\Viktor Arohlén}
\headrule
\footrule
\setlength{\columnsep}{0.25cm}
\footer{}{Sida \thepage}{}

\begin{document}

\section*{Instruktioner}
Detta är facit och bedömningsanvisningar för Biologi 1 kursprov.\\
Provet består av två delar:\\
- Kryssfrågor, endast ett alternativ är rätt om inget annat anges (\textit{13 poäng})\\
- Fördjupande/frisvarsfrågor, svara mer omfattande (\textit{12 poäng})

\subsection*{Poäng}
Varje fråga har angivet poängvärde. Provet bedöms enbart efter poäng.\\

\vspace{5mm}
\begin{center}
\fbox{\fbox{\parbox{6in}{\centering
\textbf{Kryssfrågor}: markera endast ett alternativ, om inget annat anges. (\textbf{13 poäng})
}}}
\end{center}
\vspace{5mm}
\begin{questions}

% --- DNA & RNA ---
\question (1p) En \textbf{nukleotid} består av:
\begin{checkboxes}
   \choice Aminosyra, kvävebas och deoxiribos
   \choice Kvävebas, ribos och en fosfatgrupp
   \choice Deoxiribos, ribos och fosfatgrupper(er)
   \correctchoice \textcolor{red}{Kvävebas, sockermolekyl (ribos eller deoxiribos) och fosfatgrupp(er)} \textcolor{blue}{[RÄTT SVAR]}
\end{checkboxes}
\vspace{5mm}

\question (1p) Vilken av följande processer är \textbf{inte} en del av \textbf{proteinsyntesen}?
\begin{checkboxes}
    \choice Transkription
    \choice Translation
    \correctchoice \textcolor{red}{Replikation} \textcolor{blue}{[RÄTT SVAR]}
    \choice Splicing
\end{checkboxes}
\vspace{5mm}

\question (1p) Vilken av följande är en funktion av \textbf{tRNA}?
\begin{checkboxes}
\choice Att bära genetisk information från DNA till ribosomen
\correctchoice \textcolor{red}{Att bära aminosyror till ribosomen} \textcolor{blue}{[RÄTT SVAR]}
\choice Att bilda ribosomernas struktur
\choice Att katalysera kemiska reaktioner
\end{checkboxes}
\vspace{5mm}

\question (1p) Vilka av följande är \textbf{kvävebaspar i DNA}?
\begin{checkboxes}
   \choice Guanin - Tymin
   \choice Uracil - Tymin
   \correctchoice \textcolor{red}{Adenin - Tymin} \textcolor{blue}{[RÄTT SVAR]}
   \correctchoice \textcolor{red}{Guanin - Cytosin} \textcolor{blue}{[RÄTT SVAR]}
\end{checkboxes}
\textcolor{blue}{[OBS: Denna fråga har två rätta svar, ge 1p om båda är korrekt markerade]}
\vspace{5mm}
\break
\question (1p) Vad av följande stämmer \textbf{inte} för \textbf{RNA}?
\begin{checkboxes}
   \choice Innehåller ribos
   \choice Innehåller kvävebasen uracil
   \correctchoice \textcolor{red}{Innehåller kvävebasen tymin} \textcolor{blue}{[RÄTT SVAR]}
   \choice Består av en enkelsträng (helix)
\end{checkboxes}
\vspace{5mm}

\question (1p) Hur \textbf{binder aminosyror} till varandra?
\begin{checkboxes}
   \choice Vätebindningar
   \choice Jonbindning
   \correctchoice \textcolor{red}{Peptidbindning} \textcolor{blue}{[RÄTT SVAR]}
   \choice Kemisk bindning
\end{checkboxes}
\vspace{5mm}

% --- EVOLUTION ---
\question (1p) Vilket av följande begrepp beskriver bäst \textbf{naturligt urval}?
\begin{checkboxes}
    \choice Att individer förändras för att passa miljön
    \choice Att alla individer har lika stor chans att överleva
    \correctchoice \textcolor{red}{Att de \textbf{bäst anpassade} överlever och för sina gener vidare} \textcolor{blue}{[RÄTT SVAR]}
    \choice Att miljön alltid förändras
\end{checkboxes}
\vspace{5mm}

\question (1p) Vad menas med \textbf{selektionstryck}?
\begin{checkboxes}
    \choice Inre faktorer som påverkar individen
    \choice Att alla individer överlever lika bra
    \correctchoice \textcolor{red}{\textbf{Yttre faktorer} som påverkar vilka individer som klarar sig bäst} \textcolor{blue}{[RÄTT SVAR]}
    \choice Att arter alltid utvecklas mot ökad komplexitet
\end{checkboxes}
\vspace{5mm}

\question (1p) Vilket av följande begrepp beskriver situationen där en fjärils färgmorf (färgvariant) är mer attraktiv för en partner, vilket leder till att färgmorfen blir vanligare i populationen?
\begin{checkboxes}
    \correctchoice \textcolor{red}{Sexuell selektion} \textcolor{blue}{[RÄTT SVAR]}
    \choice Genetisk drift
    \choice Flaskhalseffekt
    \choice Naturligt urval
\end{checkboxes}
\vspace{5mm}

\question (1p) Vilket av följande alternativ förklarar bäst begreppet \textbf{samevolution}?
\begin{checkboxes}
\choice En egenskap hos en art förs vidare trots att den egentligen inte fyller någon funktion.
\choice Liknande egenskaper hos arter som inte har en gemensam föregångare (förfader) som har denna egenskap.
\choice En art utvecklar på nytt en egenskap som försvunnit i ett tidigare skede av artens evolution.
\correctchoice \textcolor{red}{En arts egenskaper utvecklas i samspel med en annan art.} \textcolor{blue}{[RÄTT SVAR]}
\end{checkboxes}
\vspace{5mm}
\break
\question (1p) Vad är den \textbf{huvudsakliga skillnaden} mellan \textbf{DNA} och \textbf{RNA}?
\begin{checkboxes}
    \choice DNA är enkelsträngat, RNA är dubbelsträngat
    \correctchoice \textcolor{red}{DNA innehåller \textbf{deoxiribos} och \textbf{tymin}, RNA innehåller \textbf{ribos} och \textbf{uracil}} \textcolor{blue}{[RÄTT SVAR]}
    \choice DNA finns bara i cytoplasman
    \choice RNA lagrar genetisk information permanent
\end{checkboxes}
\vspace{5mm}

\question (1p) Vilken \textbf{funktion} har \textbf{mRNA} i cellen?
\begin{checkboxes}
   \choice Bygger upp cellmembranet
   \choice Transporterar aminosyror till ribosomen
   \correctchoice \textcolor{red}{För över genetisk information från DNA till ribosomen} \textcolor{blue}{[RÄTT SVAR]}
   \choice Katalyserar kemiska reaktioner
\end{checkboxes}
\vspace{5mm}

\question (1p) Vilken av följande är ett exempel på \textbf{genetisk drift}?
\begin{checkboxes}
   \choice När de bäst anpassade individerna överlever och för sina gener vidare
   \choice När miljön förändras och påverkar vilka gener som är fördelaktiga
   \correctchoice \textcolor{red}{När slumpen gör att vissa gener blir vanligare eller ovanligare i en liten population} \textcolor{blue}{[RÄTT SVAR]}
   \choice När två arter utvecklas liknande egenskaper utan gemensam förfader
\end{checkboxes}
\vspace{5mm}

\end{questions}

\break

\begin{center}
\fbox{\fbox{\parbox{6in}{\centering
\textbf{Fördjupande/frisvarsfrågor}: svara utförligt på frågorna nedan. (\textbf{12 poäng})
}}}
\end{center}
\vspace{5mm}
\begin{questions}

\question (2p) Beskriv skillnaden mellan \textbf{DNA} och \textbf{RNA}.

\textcolor{blue}{
\textbf{Bedömningsanvisning:}\\
För full poäng (2p) ska eleven nämna minst fyra av följande skillnader:
\begin{itemize}
\item DNA innehåller deoxiribos, RNA innehåller ribos
\item DNA innehåller tymin, RNA innehåller uracil
\item DNA är oftast dubbelsträngat, RNA är oftast enkelsträngat
\item DNA lagrar genetisk information långsiktigt, RNA har mer temporära funktioner
\item DNA finns främst i cellkärnan, RNA finns både i kärnan och cytoplasman
\item DNA är mer stabilt än RNA
\end{itemize}
För 1p krävs minst två korrekta skillnader.
}
\vspace{10mm}

\question (2p) Beskriv vad som sker under \textbf{transkription} och \textbf{translation} i cellen. Ge exempel på var i cellen processerna sker och vilka huvudaktörer som är inblandade.

\textcolor{blue}{
\textbf{Bedömningsanvisning:}\\
För full poäng (2p) ska eleven beskriva:
\begin{itemize}
\item \textbf{Transkription:} DNA kopieras till mRNA i cellkärnan med hjälp av enzymet RNA-polymeras
\item \textbf{Translation:} mRNA översätts till protein i ribosomerna (i cytoplasman eller på ER) med hjälp av tRNA och ribosomer
\end{itemize}
För 1p krävs korrekt beskrivning av antingen transkription eller translation med plats och huvudaktörer.
}
\vspace{10mm}

\question (2p) Ge tre exempel på hur vi kan bevisa \textbf{evolutionsteorin}.

\textcolor{blue}{
\textbf{Bedömningsanvisning:}\\
För full poäng (2p) ska eleven nämna tre av följande bevis:
\begin{itemize}
\item Fossil (visar utveckling över tid)
\item Homologa strukturer (liknande strukturer med samma ursprung men olika funktion)
\item Rudimentära organ (organ som förlorat sin funktion)
\item DNA/genetiska bevis (jämförelser av DNA mellan arter)
\item Biogeografi (arters utbredning på jorden)
\item Observerade exempel på evolution (t.ex. antibiotikaresistens, industriell melanism)
\item Embryologiska bevis (likheter i tidiga utvecklingsstadier)
\end{itemize}
För 1p krävs två korrekta exempel.
}
\vspace{10mm}

\newpage

\question (2p) Vad innebär \textbf{naturligt urval}?

\textcolor{blue}{
\textbf{Bedömningsanvisning:}\\
För full poäng (2p) ska eleven:
\begin{itemize}
\item Förklara att naturligt urval innebär att individer med fördelaktiga egenskaper har större chans att överleva och reproducera sig
\item Förklara att detta leder till att fördelaktiga egenskaper blir vanligare i populationen över tid
\item Nämna att variation inom populationen är nödvändig för att naturligt urval ska kunna ske
\end{itemize}
För 1p krävs en grundläggande förklaring av att vissa individer överlever bättre än andra baserat på deras egenskaper.
}
\vspace{10mm}

\question (2p) Förklara varför vissa \textbf{mutationer} inte får någon effekt på individen.

\textcolor{blue}{
\textbf{Bedömningsanvisning:}\\
För full poäng (2p) ska eleven nämna minst tre av följande orsaker:
\begin{itemize}
\item Tysta mutationer (förändrar inte aminosyran pga den genetiska kodens redundans)
\item Mutationer i icke-kodande DNA (introner, mellanliggande DNA)
\item Mutationer i gener som inte uttrycks i den aktuella celltypen
\item Reparationsmekanismer som korrigerar mutationer
\item Mutationer som ger minimal förändring i proteinets struktur/funktion
\end{itemize}
För 1p krävs minst en korrekt förklaring.
}
\vspace{10mm}

\question (2p) Resonera kring hur \textbf{antibiotikaresistens} kan utvecklas hos bakterier och varför detta är ett växande problem.

\textcolor{blue}{
\textbf{Bedömningsanvisning:}\\
För full poäng (2p) ska eleven:
\begin{itemize}
\item Förklara att antibiotikaresistens utvecklas genom naturligt urval där resistenta bakterier överlever och förökar sig
\item Beskriva att resistensgener kan spridas mellan bakterier (horisontell genöverföring)
\item Förklara att överanvändning av antibiotika ökar selektionstrycket för resistens
\item Diskutera konsekvenser (svårbehandlade infektioner, ökad dödlighet, ekonomiska kostnader)
\end{itemize}
För 1p krävs en grundläggande förklaring av hur resistens uppstår genom mutation och selektion.
}
\vspace{10mm}

\end{questions}

\end{document}
