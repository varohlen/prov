\documentclass{exam}
\usepackage{graphicx}

%Format Header and footer
\pagestyle{headandfoot}
\header{\footnotesize Klass:\\Namn:}{\Large\textbf{Celldelning och mutationer 2025}}{\footnotesize  BIOBIO01 - 2025\\Viktor Arohlén}
\headrule
\footrule
\setlength{\columnsep}{0.25cm}
%\setlength{\columnseprule}{1pt}
\footer{}{Sida \thepage}{}
%\extrafootheight{-2cm}

\begin{document}

% Kryssfrågor

\vspace{5mm}
\begin{center}
\fbox{\fbox{\parbox{6in}{\centering
\textbf{10 kryssfrågor}: markera endast ett alternativ. (1 poäng per fråga.)
}}}
\end{center}
\vspace{5mm}

\begin{questions}

% Kryssfrågor (nya men liknande)

\question Vad är syftet med \textbf{mitos}?
\begin{checkboxes}
   \choice Bilda könsceller
   \correctchoice Skapa två genetiskt identiska celler
   \choice Minska antalet kromosomer
   \choice Producera energi
\end{checkboxes}

\vspace{5mm}\hrule\vspace{5mm}

\question Vad är en \textbf{genmutation}?
\begin{checkboxes}
   \choice En förändring i antalet kromosomer
   \correctchoice En förändring i enskilda kvävebaser i DNA
   \choice En fördubbling av arvsmassan
   \choice En celldelning utan cytokines
\end{checkboxes}

\vspace{5mm}\hrule\vspace{5mm}

\question Vad kallas det när en cell dör på ett \textbf{kontrollerat sätt}?
\begin{checkboxes}
   \choice Nekros
   \correctchoice Apoptos
   \choice Mutation
   \choice Replikation
\end{checkboxes}

\vspace{5mm}\hrule\vspace{5mm}

\question Hur många \textbf{kromosomer} har en mänsklig \textbf{kroppscell} efter mitos?
\begin{checkboxes}
   \choice 23
   \correctchoice 46
   \choice 92
   \choice 44
\end{checkboxes}

\vspace{5mm}\hrule\vspace{5mm}

\question Vilken typ av \textbf{mutation} kan gå i arv till nästa \textbf{generation}?
\begin{checkboxes}
   \choice Somatisk mutation
   \correctchoice Genetisk mutation i könsceller
   \choice Mutation i hudceller
   \choice Mutation i leverceller
\end{checkboxes}

\break

\question Vad är \textbf{kromatin}?
\begin{checkboxes}
   \choice Ett annat namn för DNA
   \choice Proteiner i cellkärnan
   \correctchoice DNA och proteiner
   \choice Fler än två kromosomer
\end{checkboxes}

\vspace{5mm}\hrule\vspace{5mm}

\question Vilken av följande kan \textbf{INTE orsaka mutationer}?
\begin{checkboxes}
   \choice Strålning
   \choice Virus
   \choice Normal celldelning
   \correctchoice Inget av ovan
\end{checkboxes}

\vspace{5mm}\hrule\vspace{5mm}


\question Vad händer med \textbf{kromosomantalet} i cellerna efter \textbf{meios}?
\begin{checkboxes}
   \choice Det fördubblas
   \choice Det är oförändrat
   \correctchoice Det halveras
   \choice Det tredubblas
\end{checkboxes}

\vspace{5mm}\hrule\vspace{5mm}

\question Vilket av följande kännetecknar \textbf{cancerceller}?
\begin{checkboxes}
   \choice De slutar dela sig tidigt
   \correctchoice De delar sig okontrollerat
   \choice De har alltid färre kromosomer
   \choice De kan inte mutera
\end{checkboxes}

\vspace{5mm}\hrule\vspace{5mm}
\question Vilken av följande \textbf{könskromosomuppsättningar} är inte förenlig med \textbf{liv} hos människan?
\begin{checkboxes}
   \choice X 
   \choice XXY
   \choice XYY
   \correctchoice YY
\end{checkboxes}

\break

\vspace{5mm}
\begin{center}
\fbox{\fbox{\parbox{6in}{\centering
\textbf{4 kortsvarsfrågor}: Svara kortfattat på frågorna nedan. Använd relevanta begrepp och figurer där det passar. (2 poäng per fråga)
}}}
\end{center}
\vspace{5mm}

\question Vad menas med en \textbf{haploid} respektive en \textbf{diploid} cell? Ge exempel på var i kroppen dessa finns.
\vspace{30mm}

\question Matcha vad som händer i \textbf{mitos} och \textbf{meios} med rätt fas.\\

\textbf{Alternativ:}
\begin{itemize}
  \item[A.] Kromosomer radar upp sig parvis i cellens mittplan.
  \item[B.] Systerkromatider dras isär till varsin cellpol.
  \item[C.] Överkorsning sker mellan homologa kromosomer.
  \item[D.] Kromosomer börjar kondenseras och blir synliga i en kroppscell.
  \item[E.] Kärnmembranet återbildas runt kromosomerna.
\end{itemize}

\textbf{Faser:}
\begin{itemize}
  \item[1.] Profas
  \item[2.] Profas I
  \item[3.] Metafas
  \item[4.] Anafas
  \item[5.] Telofas
\end{itemize}
\vspace{5mm}
\question Vad menas med \textbf{"programmerad celldöd"} och varför är det viktigt?
\vspace{30mm}

\question Vad innebär en \textbf{"trisomi"} och ge ett exempel?
\vspace{25mm}


\break

\vspace{5mm} %5mm vertical space
\begin{center}
\fbox{\fbox{\parbox{6in}{\centering
\textbf{2 frisvarsfrågor}:  Svara på utrymmet under frågan. Anvönd relevanta begrepp och figurer. (4 poäng per fråga)
}}}
\end{center}

\question
Hos människor och många djur skiljer sig könscellerna tydligt åt i både storlek och funktion:
\begin{itemize}
  \item \textbf{Anisogami:} Äggcellen är stor och näringsrik, spermien är liten och rörlig.
\end{itemize}

Hos vissa organismer (t.ex. många alger och svampar) förekommer istället:
\begin{itemize}
  \item \textbf{Isogami:} Könscellerna är lika stora och ofta likartade i form. Det finns ingen tydlig uppdelning i "hona" och "hane".
\end{itemize}

Exempel: Grönalgen \textit{Chlamydomonas}, där två lika stora könsceller smälter samman vid befruktning.

\begin{center}
    \includegraphics[width=0.28\textwidth]{chlamydomonas.jpg}
    
    \small{Bild: \textit{Chlamydomonas}, en encellig grönalg som förökar sig med isogami.}
\end{center}

\textbf{Fråga:} Vad kan det ha för biologisk betydelse att könscellerna är lika stora och likartade (isogami)?
\begin{itemize}
  \item Resonera kring möjliga \textbf{fördelar} och \textbf{nackdelar} med isogami jämfört med anisogami.
  \item Ta hjälp av dina kunskaper om \textbf{celldelning}, \textbf{energiförbrukning} och \textbf{befruktning}.
\end{itemize}

\break

\question
De flesta kromosomavvikelser leder till missfall eftersom cellerna inte fungerar. Men det finns undantag där individen kan överleva och ibland leva ett relativt normalt liv.

\begin{itemize}
  \item \textbf{Downs syndrom:} Tre exemplar av kromosom 21 (trisomi 21). Påverkar utvecklingen men är förenligt med liv.
  \item \textbf{Turners syndrom:} Endast en X-kromosom (45,X).
  \item \textbf{Klinefelters syndrom:} En extra X-kromosom hos pojke (47,XXY).
\end{itemize}

\textbf{Fråga:} Varför kan vissa kromosomavvikelser – som Downs syndrom och vissa könskromosomavvikelser – vara förenliga med liv och ibland leda till vuxen ålder, medan andra avvikelser inte är det?

\begin{itemize}
  \item Resonera utifrån \textbf{celldelning}, \textbf{genbalans} och \textbf{kromosomernas funktion}.
\end{itemize}

\end{questions}

\end{document}
