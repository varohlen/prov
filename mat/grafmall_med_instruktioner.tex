% ---------------------------------------------
% MALL FÖR GRAFUPPGIFTER MED PGFPLOTS (LATEX)
% ---------------------------------------------
% Denna mall ger alltid ett kvadratiskt rutnät på papperet,
% och du kan själv välja intervall, tick-steg och funktion.
% 
% SÅ HÄR ANVÄNDER DU MALLEN:
% 1. Sätt xmin, xmax, ymin, ymax till önskade gränser.
% 2. Välj xtick distance och ytick distance så att grafen blir lättläst.
% 3. Sätt width/(xmax-xmin) = height/(ymax-ymin) för att rutorna ska bli kvadratiska på papperet.
%    (Exempel: 11 x-rutor och 24 y-rutor: width=7cm, height=15.27cm)
% 4. Byt ut FUNKTIONSUTTRYCK mot önskad funktion.
% 5. Kopiera in koden i din LaTeX-fil där du vill ha grafen.
% 
% Exempel på användning finns längst ner!
% ---------------------------------------------

% Lägg i preamble om det inte redan finns:
% \usepackage{pgfplots}
% \pgfplotsset{compat=1.17}

\begin{center}
\begin{tikzpicture}
  \begin{axis}[
    axis lines=middle,
    axis line style/.append style={-},
    grid=both,
    xmin=START_X, xmax=END_X,
    ymin=START_Y, ymax=END_Y,
    samples=200,
    width=GRAF_BREDD,
    height=GRAF_HOJD,
    domain=START_X:END_X,
    xtick distance=XTICK_STEG,
    ytick distance=YTICK_STEG,
    clip=true
  ]
    \addplot[blue, thick] {FUNKTIONSUTTRYCK};
  \end{axis}
\end{tikzpicture}
\end{center}

% -----------
% EXEMPEL 1:
% -----------
% Graf av f(x) = -0.5x^3 + 3x^2 + x + 2, x från -3 till 8, y från 0 till 24,
% 1 ruta per x-enhet, 1 ruta per 2 y-enheter, kvadratiskt rutnät på papperet:
%
% \begin{tikzpicture}
%   \begin{axis}[
%     axis lines=middle,
%     axis line style/.append style={-},
%     grid=both,
%     xmin=-3, xmax=8,
%     ymin=0, ymax=24,
%     samples=200,
%     width=7cm,
%     height=15.27cm,
%     domain=-3:8,
%     xtick distance=1,
%     ytick distance=2,
%     clip=true
%   ]
%     \addplot[blue, thick] {-0.5*x^3 + 3*x^2 + x + 2};
%   \end{axis}
% \end{tikzpicture}
