\documentclass[a4paper,11pt]{article}
\usepackage[utf8]{inputenc}
\usepackage[T1]{fontenc}
\usepackage[swedish]{babel}
\usepackage{amsmath,amssymb,amsfonts}
\usepackage{enumitem}
\usepackage{geometry}
\geometry{margin=2.5cm}

\title{Extraövningar: Kvadreringsregeln och andragradsekvationer}
\author{Genererade av AI}
\date{\today}

\begin{document}

\maketitle

% Kvadreringsregeln
\section{Kvadreringsregeln}
\begin{enumerate}[label=\textbf{\arabic*.}]
    \item Utveckla och förenkla: $(x+4)^2$
    \item Utveckla och förenkla: $(2y-3)^2$
    \item Utveckla och förenkla: $(a-5)^2$
    \item Utveckla och förenkla: $(3x+2)^2$
    \item Utveckla och förenkla: $(x-\frac{1}{3})^2$
    \item Utveckla och förenkla: $(4z+7)^2$
    \item Utveckla och förenkla: $(2x-\frac{5}{2})^2$
    \item Utveckla och förenkla: $(x+1)^2 + (x-1)^2$
\end{enumerate}

% Andragradsekvationer med pq-formel
\section{Andragradsekvationer med pq-formel}
\begin{enumerate}[label=\textbf{\arabic*.}]
    \item Lös ekvationen: $2x^2 = 5x - 3$
    \item Lös ekvationen: $4 + x^2 = 4x$
    \item Lös ekvationen: $3x^2 + 6 = 7x$
    \item Lös ekvationen: $5x = x^2 + 6$
\end{enumerate}

% Problemlösning med andragradsekvation
\section{Problemlösning med andragradsekvation}
\begin{enumerate}[label=\textbf{\arabic*.}]
    \item Summan av två tal är 10 och produkten är 21. Vilka är talen?
    \item En rektangel har arean $30\text{ cm}^2$. Längden är 2 cm längre än bredden. Bestäm rektangelns dimensioner.
    \item Kvadraten på ett tal är lika med 7 gånger talet minskat med 10. Vilket är talet?
    \item En triangel har basen $x$ cm och höjden $x+3$ cm. Arean är $30\text{ cm}^2$. Bestäm basen och höjden.
\end{enumerate}

% Facit
\newpage
\section*{Facit}

\subsection*{Kvadreringsregeln}
\begin{enumerate}[label=\textbf{\arabic*.}]
    \item $(x+4)^2 = x^2 + 8x + 16$
    \item $(2y-3)^2 = 4y^2 - 12y + 9$
    \item $(a-5)^2 = a^2 - 10a + 25$
    \item $(3x+2)^2 = 9x^2 + 12x + 4$
    \item $(x-\frac{1}{3})^2 = x^2 - \frac{2}{3}x + \frac{1}{9}$
    \item $(4z+7)^2 = 16z^2 + 56z + 49$
    \item $(2x-\frac{5}{2})^2 = 4x^2 - 5x + \frac{25}{4}$
    \item $(x+1)^2 + (x-1)^2 = (x^2 + 2x + 1) + (x^2 - 2x + 1) = 2x^2 + 2$
\end{enumerate}

\subsection*{Andragradsekvationer med pq-formel}
\begin{enumerate}[label=\textbf{\arabic*.}]
    \item $2x^2 = 5x - 3 \Rightarrow 2x^2 - 5x + 3 = 0$\\
    pq-form: $x^2 - \frac{5}{2}x + \frac{3}{2} = 0$\\
    $p = -\frac{5}{2}$, $q = \frac{3}{2}$\\
    $x = \frac{5}{4} \pm \sqrt{\left(\frac{5}{4}\right)^2 - \frac{3}{2}} = \frac{5}{4} \pm \sqrt{\frac{25}{16} - \frac{24}{16}} = \frac{5}{4} \pm \frac{1}{4}$\\
    Svar: $x_1 = 1.5$, $x_2 = 1$
    \item $4 + x^2 = 4x \Rightarrow x^2 - 4x + 4 = 0$\\
    pq-form: $x^2 - 4x + 4 = 0$\\
    $p = -4$, $q = 4$\\
    $x = 2 \pm \sqrt{2^2 - 4} = 2 \pm 0$\\
    Svar: $x = 2$
    \item $3x^2 + 6 = 7x \Rightarrow 3x^2 - 7x + 6 = 0$\\
    pq-form: $x^2 - \frac{7}{3}x + 2 = 0$\\
    $p = -\frac{7}{3}$, $q = 2$\\
    $x = \frac{7}{6} \pm \sqrt{\left(\frac{7}{6}\right)^2 - 2}$\\
    $= \frac{7}{6} \pm \sqrt{\frac{49}{36} - \frac{72}{36}} = \frac{7}{6} \pm \sqrt{-\frac{23}{36}}$\\
    Ingen reell lösning (diskriminanten är negativ).
    \item $5x = x^2 + 6 \Rightarrow x^2 - 5x + 6 = 0$\\
    pq-form: $x^2 - 5x + 6 = 0$\\
    $p = -5$, $q = 6$\\
    $x = 2.5 \pm \sqrt{2.5^2 - 6} = 2.5 \pm \sqrt{6.25 - 6} = 2.5 \pm 0.5$\\
    Svar: $x_1 = 3$, $x_2 = 2$
\end{enumerate}

\subsection*{Problemlösning med andragradsekvation}
\begin{enumerate}[label=\textbf{\arabic*.}]
    \item Summan $a + b = 10$, produkten $ab = 21$.
    \newline Ekvation: $x^2 - 10x + 21 = 0$\\
    pq-form: $x = 5 \pm \sqrt{25 - 21} = 5 \pm 2$\\
    Svar: $a = 7$, $b = 3$
    \item Area $A = x(x+2) = 30 \Rightarrow x^2 + 2x - 30 = 0$\\
    pq-form: $x = -1 \pm \sqrt{1 + 30} = -1 \pm \sqrt{31}$\\
    $x \approx 4.6$ (bredd), längd $\approx 6.6$ cm
    \item $x^2 = 7x - 10 \Rightarrow x^2 - 7x + 10 = 0$\\
    pq-form: $x = 3.5 \pm \sqrt{12.25 - 10} = 3.5 \pm 1.5$\\
    Svar: $x_1 = 5$, $x_2 = 2$
    \item Area $= \frac{x(x+3)}{2} = 30 \Rightarrow x^2 + 3x - 60 = 0$\\
    pq-form: $x = -1.5 \pm \sqrt{1.5^2 + 60} = -1.5 \pm 7.87$\\
    $x_1 \approx 6.4$ (bas), höjd $\approx 9.4$ cm
\end{enumerate}

\end{document}

