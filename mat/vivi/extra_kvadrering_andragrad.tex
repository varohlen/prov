\documentclass[a4paper,11pt]{article}
\usepackage[utf8]{inputenc}
\usepackage[T1]{fontenc}
\usepackage[swedish]{babel}
\usepackage{amsmath,amssymb,amsfonts}
\usepackage{enumitem}
\usepackage{geometry}
\geometry{margin=2.5cm}

\title{Extraövningar: Kvadreringsregeln och andragradsekvationer}
\author{Genererade av AI}
\date{\today}

\begin{document}

\maketitle

% Kvadreringsregeln
\section{Kvadreringsregeln}
\begin{enumerate}[label=\textbf{\arabic*.}]
    \item Utveckla och förenkla: $(x+4)^2$
    \item Utveckla och förenkla: $(2y-3)^2$
    \item Utveckla och förenkla: $(a-5)^2$
    \item Utveckla och förenkla: $(3x+2)^2$
    \item Utveckla och förenkla: $(x-\frac{1}{3})^2$
    \item Utveckla och förenkla: $(4z+7)^2$
    \item Utveckla och förenkla: $(2x-\frac{5}{2})^2$
    \item Utveckla och förenkla: $(x+1)^2 + (x-1)^2$
\end{enumerate}

% Andragradsekvationer med pq-formel
\section{Andragradsekvationer med pq-formel}
\begin{enumerate}[label=\textbf{\arabic*.}]
    \item Lös ekvationen: $2x^2 = 5x - 3$
    \item Lös ekvationen: $4 + x^2 = 4x$
    \item Lös ekvationen: $3x^2 + 6 = 7x$
    \item Lös ekvationen: $5x = x^2 + 6$
\end{enumerate}

% Problemlösning med andragradsekvation
\section{Problemlösning med andragradsekvation}
\begin{enumerate}[label=\textbf{\arabic*.}]
    \item Summan av två tal är 10 och produkten är 21. Vilka är talen?
    \item En rektangel har arean $30\text{ cm}^2$. Längden är 2 cm längre än bredden. Bestäm rektangelns dimensioner.
    \item Kvadraten på ett tal är lika med 7 gånger talet minskat med 10. Vilket är talet?
    \item En triangel har basen $x$ cm och höjden $x+3$ cm. Arean är $30\text{ cm}^2$. Bestäm basen och höjden.
\end{enumerate}

\end{document}
