\documentclass[a4paper,11pt]{article}
\usepackage[utf8]{inputenc}
\usepackage[T1]{fontenc}
\usepackage[swedish]{babel}
\usepackage{amsmath,amssymb,amsfonts}
\usepackage{graphicx}
\usepackage{enumitem}
\usepackage{geometry}
\geometry{margin=2.5cm}

\title{Facit: Repetitionsuppgifter -- Matematik 2b}
\author{Vivi Olander}
\date{\today}

\begin{document}

\maketitle

\section{Algebra och parentesmultiplikation}
\begin{enumerate}[label=\textbf{\arabic*.}]
    \item $3x^2 - 10x - 8$
    \item $6a^2 - 13a - 5$
    \item $2x^2 + 15x + 13$
    \item $16x - 3$
    \item $25 - 4y^2$
\end{enumerate}

\section{Konjugat och kvadreringsregler}
\begin{enumerate}[label=\textbf{\arabic*.}]
    \item $16 - 3 = 13$
    \item $x^2 + 10x + 25$
    \item $4a^2 - 12a + 9$
    \item $9x^2 - 4y^2$
    \item $x^2 - x + \frac{1}{4}$
\end{enumerate}

\section{Enkla andragradsekvationer}
\begin{enumerate}[label=\textbf{\arabic*.}]
    \item $x = \pm 4$
    \item $x = \pm 3$
    \item $x = \pm 3$
    \item $x = \pm 2$
    \item $x = 2 \pm 3 = -1$ eller $x = 5$
\end{enumerate}

\section{Andragradsekvationer med nollproduktsmetoden}
\begin{enumerate}[label=\textbf{\arabic*.}]
    \item $x = 0$ eller $x = 5$
    \item $x = 3$ eller $x = -2$
    \item $x = 0$ eller $x = 7$
    \item $x = -\frac{1}{2}$ eller $x = 4$
\end{enumerate}

\section{Andragradsekvationer med lösningsformel (pq-formel)}
\begin{enumerate}[label=\textbf{\arabic*.}]
    \item $x = \frac{6 \pm \sqrt{36-32}}{2} = \frac{6 \pm \sqrt{4}}{2} = \frac{6 \pm 2}{2}$, dvs $x = 4$ eller $x = 2$
    \item $x = \frac{-2 \pm \sqrt{4+32}}{2} = \frac{-2 \pm \sqrt{36}}{2} = \frac{-2 \pm 6}{2}$, dvs $x = 2$ eller $x = -4$
    \item $x = \frac{4 \pm \sqrt{16-16}}{2} = \frac{4 \pm 0}{2} = 2$
    \item $2x^2 - 7x + 3 = 0 \Rightarrow x^2 - \frac{7}{2}x + \frac{3}{2} = 0 \Rightarrow x = \frac{7 \pm \sqrt{49-24}}{4} = \frac{7 \pm \sqrt{25}}{4} = \frac{7 \pm 5}{4}$, dvs $x = 3$ eller $x = \frac{1}{2}$
    \item $3x^2 + 6x - 9 = 0 \Rightarrow x^2 + 2x - 3 = 0 \Rightarrow x = \frac{-2 \pm \sqrt{4+12}}{2} = \frac{-2 \pm \sqrt{16}}{2} = \frac{-2 \pm 4}{2}$, dvs $x = 1$ eller $x = -3$
    \item $5x^2 - 10 = 15x \Rightarrow 5x^2 - 15x - 10 = 0 \Rightarrow x^2 - 3x - 2 = 0 \Rightarrow x = \frac{3 \pm \sqrt{9+8}}{2} = \frac{3 \pm \sqrt{17}}{2}$
\end{enumerate}

\section{Blandade uppgifter}
\begin{enumerate}[label=\textbf{\arabic*.}]
    \item Bredden är $4\text{ cm}$ och längden är $6\text{ cm}$
    \item $A = 9\text{ cm}^2$
    \item $12$ och $13$
    \item $\frac{x^2-9}{(x+1)^2} = \frac{(x-3)(x+3)}{(x+1)^2}$
    \item $\frac{x^2-4}{x-2} = x+2 \Rightarrow \frac{(x-2)(x+2)}{x-2} = x+2 \Rightarrow x+2 = x+2$ (för $x \neq 2$), vilket är sant för alla $x \neq 2$. Ekvationen har alltså oändligt många lösningar, men $x = -2$ är också en lösning efter förkortning.
\end{enumerate}

\section{Blandade repetitionsuppgifter}
\begin{enumerate}[label=\textbf{\arabic*.}]
    \item $6x^2 + 10x - 4$
    \item $25 - 8 = 17$
    \item $x = \pm 5$
    \item $x = 1$ eller $x = -6$
    \item $x = \frac{3 \pm \sqrt{9+16}}{2} = \frac{3 \pm \sqrt{25}}{2} = \frac{3 \pm 5}{2}$, dvs $x = 4$ eller $x = -1$
    \item $(x+3)^2 - (x-3)^2 = (x^2 + 6x + 9) - (x^2 - 6x + 9) = 12x$
    \item $x = \pm 3$
    \item $x = 0$ eller $x = 8$
    \item $2x^2 + x - 6 = 0 \Rightarrow x^2 + \frac{1}{2}x - 3 = 0 \Rightarrow x = \frac{-\frac{1}{2} \pm \sqrt{\frac{1}{4}+12}}{2} = \frac{-\frac{1}{2} \pm \sqrt{\frac{49}{4}}}{2} = \frac{-\frac{1}{2} \pm \frac{7}{2}}{2}$, dvs $x = \frac{3}{2}$ eller $x = -2$
    \item $2(x - 3) - 3(2 - x) = 2x - 6 - 6 + 3x = 5x - 12$
    \item $x = \frac{1}{2}$ eller $x = -3$
    \item $(3 - 2y)^2 = 9 - 12y + 4y^2$
    \item $x = \frac{2 \pm \sqrt{4+60}}{2} = \frac{2 \pm \sqrt{64}}{2} = \frac{2 \pm 8}{2}$, dvs $x = 5$ eller $x = -3$
\end{enumerate}

\end{document}
