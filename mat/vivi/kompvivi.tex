\documentclass[a4paper,11pt]{article}
\usepackage[utf8]{inputenc}
\usepackage[T1]{fontenc}
\usepackage[swedish]{babel}
\usepackage{amsmath,amssymb,amsfonts}
\usepackage{graphicx}
\usepackage{enumitem}
\usepackage{geometry}
\geometry{margin=2.5cm}

\title{Repetitionsuppgifter -- Matematik 2b}
\author{Vivi Olander}
\date{\today}

\begin{document}

\maketitle

\section{Algebra och parentesmultiplikation}

\begin{enumerate}[label=\textbf{\arabic*.}]
    \item Förenkla uttrycket: $(3x + 2)(x - 4)$
    
    \item Utveckla och förenkla: $(2a - 5)(3a + 1)$
    
    \item Beräkna: $(x + 3)(x + 5) - (x - 2)(x + 1)$
    
    \item Förenkla: $2(3x - 4) + 5(2x + 1)$
    
    \item Utveckla och förenkla: $(5 - 2y)(5 + 2y)$
\end{enumerate}

\section{Konjugat och kvadreringsregler}

\begin{enumerate}[label=\textbf{\arabic*.}]
    \item Beräkna med hjälp av konjugatregeln: $(4 + \sqrt{3})(4 - \sqrt{3})$
    
    \item Använd första kvadreringsregeln för att utveckla: $(x + 5)^2$
    
    \item Använd andra kvadreringsregeln för att utveckla: $(2a - 3)^2$
    
    \item Förenkla med hjälp av konjugatregeln: $(3x + 2y)(3x - 2y)$
    
    \item Beräkna med hjälp av lämplig kvadreringsregel: $(x - \frac{1}{2})^2$
\end{enumerate}

\section{Enkla andragradsekvationer}

\begin{enumerate}[label=\textbf{\arabic*.}]
    \item Lös ekvationen: $x^2 = 16$
    
    \item Lös ekvationen: $x^2 - 9 = 0$
    
    \item Lös ekvationen: $2x^2 = 18$
    
    \item Lös ekvationen: $3x^2 - 12 = 0$
    
    \item Lös ekvationen: $(x - 2)^2 = 9$
\end{enumerate}

\section{Andragradsekvationer med nollproduktsmetoden}

\begin{enumerate}[label=\textbf{\arabic*.}]
    \item Lös ekvationen: $x^2 - 5x = 0$
    
    \item Lös ekvationen: $(x - 3)(x + 2) = 0$
    
    \item Lös ekvationen: $x(x - 7) = 0$
    
    \item Lös ekvationen: $(2x + 1)(x - 4) = 0$
\end{enumerate}

\section{Andragradsekvationer med lösningsformel (pq-formel)}

\begin{enumerate}[label=\textbf{\arabic*.}]
    \item Lös ekvationen med pq-formeln: $x^2 - 6x + 8 = 0$
    
    \item Lös ekvationen med pq-formeln: $x^2 + 2x - 8 = 0$
    
    \item Lös ekvationen med pq-formeln: $x^2 - 4x + 4 = 0$
    
    \item Lös ekvationen med pq-formeln: $2x^2 - 7x + 3 = 0$ (Omvandla först till standardform)
    
    \item Lös ekvationen med pq-formeln: $3x^2 + 6x - 9 = 0$ 
    
    \item Lös ekvationen med pq-formeln: $5x^2 - 10 = 15x$ 
\end{enumerate}

\section{Blandade uppgifter}

\begin{enumerate}[label=\textbf{\arabic*.}]
    \item En rektangel har arean $24\text{ cm}^2$. Längden är $3\text{ cm}$ längre än bredden. Bestäm rektangelns dimensioner.
    
    \item En kvadrat har arean $A\text{ cm}^2$. Om sidlängden ökas med $2\text{ cm}$, blir den nya arean $25\text{ cm}^2$. Bestäm värdet på $A$.
    
    \item Produkten av två på varandra följande heltal är $156$. Vilka är talen?
    
    \item Förenkla uttrycket: $\frac{(x+3)(x-3)}{(x+1)^2}$
    
    \item Lös ekvationen: $\frac{x^2-4}{x-2} = x+2$ för $x \neq 2$
\end{enumerate}

\section{Blandade repetitionsuppgifter}

\begin{enumerate}[label=\textbf{\arabic*.}]
    \item Utveckla och förenkla: $(3x - 1)(2x + 4)$
    
    \item Beräkna med hjälp av konjugatregeln: $(5 + 2\sqrt{2})(5 - 2\sqrt{2})$
    
    \item Lös ekvationen: $x^2 - 25 = 0$
    
    \item Lös ekvationen: $(x - 1)(x + 6) = 0$
    
    \item Lös ekvationen med pq-formeln: $x^2 - 3x - 4 = 0$
    
    \item Utveckla och förenkla: $(x + 3)^2 - (x - 3)^2$
    
    \item Lös ekvationen: $3x^2 = 27$
    
    \item Lös ekvationen: $x(x - 8) = 0$
    
    \item Lös ekvationen med pq-formeln: $2x^2 + x - 6 = 0$ 
    
    \item Förenkla uttrycket: $2(x - 3) - 3(2 - x)$
    
    \item Lös ekvationen: $(2x - 1)(x + 3) = 0$
    
    \item Använd andra kvadreringsregeln för att utveckla: $(3 - 2y)^2$
    
    \item Lös ekvationen med pq-formeln: $x^2 - 2x - 15 = 0$
\end{enumerate}

\end{document}