\documentclass[a4paper,11pt]{article}
\usepackage[utf8]{inputenc}
\usepackage[T1]{fontenc}
\usepackage[swedish]{babel}
\usepackage{amsmath,amssymb,amsfonts}
\usepackage{graphicx}
\usepackage{enumitem}
\usepackage{geometry}
\geometry{margin=2.5cm}

\title{Repetitionsuppgifter -- Matematik 2b (Facit)}
\author{Adam Damaj}
\date{\today}

\begin{document}

\maketitle

\section{Algebra och parentesmultiplikation}

\begin{enumerate}[label=\textbf{\arabic*.}]
    \item Förenkla uttrycket: $(3x + 2)(x - 4)$
    \\ \textbf{Facit:} $3x^2 - 10x - 8$
    
    \item Utveckla och förenkla: $(2a - 5)(3a + 1)$
    \\ \textbf{Facit:} $6a^2 - 13a - 5$
    
    \item Beräkna: $(x + 3)(x + 5) - (x - 2)(x + 1)$
    \\ \textbf{Facit:} $9x + 17$
    
    \item Förenkla: $2(3x - 4) + 5(2x + 1)$
    \\ \textbf{Facit:} $16x - 3$
    
    \item Utveckla och förenkla: $(5 - 2y)(5 + 2y)$
    \\ \textbf{Facit:} $25 - 4y^2$
\end{enumerate}


\section{Konjugat och kvadreringsregler}

\begin{enumerate}[label=\textbf{\arabic*.}]
    \item Beräkna med hjälp av konjugatregeln: $(4 + \sqrt{3})(4 - \sqrt{3})$
    \\ \textbf{Facit:} $16 - 3 = 13$
    
    \item Använd första kvadreringsregeln för att utveckla: $(x + 5)^2$
    \\ \textbf{Facit:} $x^2 + 10x + 25$
    
    \item Använd andra kvadreringsregeln för att utveckla: $(2a - 3)^2$
    \\ \textbf{Facit:} $4a^2 - 12a + 9$
    
    \item Förenkla med hjälp av konjugatregeln: $(3x + 2y)(3x - 2y)$
    \\ \textbf{Facit:} $9x^2 - 4y^2$
    
    \item Beräkna med hjälp av lämplig kvadreringsregel: $(x - \frac{1}{2})^2$
    \\ \textbf{Facit:} $x^2 - x + \frac{1}{4}$
\end{enumerate}

\section{Enkla ekvationssystem}

\begin{enumerate}[label=\textbf{\arabic*.}]
    \item Lös ekvationssystemet:
    \begin{align*}
    3x + 2y &= 7\\
    x - y &= 4
    \end{align*}
    \\ \textbf{Facit:} $x=5$, $y=1$
    
    \item Lös ekvationssystemet:
    \begin{align*}
    4x - 3y &= 10\\
    2x + y &= 8
    \end{align*}
    \\ \textbf{Facit:} $x=2$, $y=4$
    
    \item Lös ekvationssystemet:
    \begin{align*}
    x + 2y &= 5\\
    3x - y &= 4
    \end{align*}
    \\ \textbf{Facit:} $x=2$, $y=1.5$
\end{enumerate}

\section{Problemlösning med andragradsfunktioner}

\begin{enumerate}[label=\textbf{\arabic*.}]
    \item En boll kastas rakt uppåt från marken med en utgångshastighet på $20$ m/s. Bollens höjd $h$ (i meter) efter $t$ sekunder ges av funktionen $h(t) = 20t - 5t^2$. 
    \begin{enumerate}[label=\alph*)]
        \item När når bollen sin högsta höjd?
        \\ \textbf{Facit:} $t=2$ sekunder
        \item Hur hög når bollen som högst?
        \\ \textbf{Facit:} $h(2) = 20\cdot2 - 5\cdot4 = 20$ meter
        \item När träffar bollen marken igen?
        \\ \textbf{Facit:} $t=0$ eller $t=4$ sekunder
    \end{enumerate}
    
    \item En rektangel har omkretsen $24$ cm. Låt $x$ vara rektangelns bredd.
    \begin{enumerate}[label=\alph*)]
        \item Uttryck rektangelns längd som en funktion av $x$.
        \\ \textbf{Facit:} $l = 12-x$
        \item Uttryck rektangelns area $A$ som en funktion av $x$.
        \\ \textbf{Facit:} $A = x(12-x)$
        \item Vilka värden kan $x$ anta?
        \\ \textbf{Facit:} $0 < x < 12$
        \item För vilket värde på $x$ blir arean maximal?
        \\ \textbf{Facit:} $x=6$
        \item Vad är den maximala arean?
        \\ \textbf{Facit:} $A=36$ cm$^2$
    \end{enumerate}
\end{enumerate}

\end{document}

