\documentclass[a4paper,11pt]{article}
\usepackage[utf8]{inputenc}
\usepackage[T1]{fontenc}
\usepackage[swedish]{babel}
\usepackage{amsmath,amssymb,amsfonts}
\usepackage{enumitem}
\usepackage{geometry}
\geometry{margin=2.5cm}

\title{Test E-nivå: Algebra, ekvationssystem och andragradsfunktioner}
\author{}
\date{\today}

\begin{document}

\maketitle

\section*{Uppgifter}
\begin{enumerate}[label=\textbf{\arabic*.}]
    % 1. Algebra och parentesmultiplikation
    \item Förenkla uttrycket: $(2x - 3)(x + 5)$

    % 2. Konjugat- och kvadreringsregler
    \item Utveckla och förenkla: $(y + 4)^2 - (y - 4)^2$

    % 3. Ekvationssystem
    \item Lös ekvationssystemet:
    \begin{align*}
    2x + y &= 11 \\
    x - y &= 1
    \end{align*}

    % 4. Problemlösning med ekvationssystem
    \item Alma köper två glassar och en dricka för 38 kr. En glass kostar 15 kr mer än en dricka. Hur mycket kostar en glass och hur mycket kostar en dricka?

    % 4. Problemlösning med andragradsfunktion
    \item En boll kastas från en höjd av $2$ meter med en utgångshastighet på $16$ m/s. Bollens höjd $h$ (i meter) efter $t$ sekunder ges av $h(t) = 2 + 16t - 5t^2$.
    \begin{enumerate}[label=\alph*)]
        \item När når bollen sin högsta höjd?
        \item Hur högt når bollen?
        \item När träffar bollen marken?
    \end{enumerate}

    % 5. Problemlösning med rektangel
    \item En rektangels area är $30$ cm$^2$ och längden är $x+2$ cm, bredden $x$ cm. Bestäm rektangelns mått.

\end{enumerate}

\newpage
\section*{Facit}
\begin{enumerate}[label=\textbf{\arabic*.}]
    \item $(2x - 3)(x + 5) = 2x^2 + 10x - 3x - 15 = 2x^2 + 7x - 15$
    \item $(y + 4)^2 - (y - 4)^2 = (y^2 + 8y + 16) - (y^2 - 8y + 16) = 16y$
    \item \text{Lös ut $x$ ur andra ekvationen: } x = y + 1.\newline Sätt in i första: $2(y + 1) + y = 11 \Rightarrow 2y + 2 + y = 11 \Rightarrow 3y = 9 \Rightarrow y = 3$.\newline $x = 3 + 1 = 4$\newline Svar: $x=4$, $y=3$
    \item \text{Låt $g$ = glass, $d$ = dricka.}\newline
    $2g + d = 38$\newline
    $g = d + 15$\newline
    Sätt in i första: $2(d+15) + d = 38 \Rightarrow 2d + 30 + d = 38 \Rightarrow 3d = 8 \Rightarrow d = 8$\newline
    $g = 8 + 15 = 23$\newline
    Svar: Glass = 23 kr, dricka = 8 kr
    \item a) Högsta punkten: $t = -\frac{16}{2\cdot(-5)} = 1.6$\newline b) $h(1.6) = 2 + 16 \cdot 1.6 - 5 \cdot (1.6)^2 = 2 + 25.6 - 12.8 = 14.8$\newline c) $h(t) = 0 \Rightarrow 2 + 16t - 5t^2 = 0 \Rightarrow 5t^2 - 16t - 2 = 0$\newline pq-formeln: $t = \frac{16}{10} \pm \sqrt{(1.6)^2 + 0.4} \approx 3.3$\newline Svar: $t \approx 3.3$ sekunder
    \item $x(x+2) = 30 \Rightarrow x^2 + 2x - 30 = 0$\newline pq-formeln: $x = -1 \pm \sqrt{1 + 30} = -1 \pm \sqrt{31}$\newline Svar: $x \approx 4.6$, $x+2 \approx 6.6$ (avrundat)

\end{enumerate}

\end{document}
