\documentclass[a4paper,11pt]{article}
\usepackage[utf8]{inputenc}
\usepackage[T1]{fontenc}
\usepackage[swedish]{babel}
\usepackage{amsmath,amssymb,amsfonts}
\usepackage{graphicx}
\usepackage{enumitem}
\usepackage{geometry}
\geometry{margin=2.5cm}

\title{Repetitionsuppgifter -- Matematik 2b}
\author{Matilda Svanlund}
\date{\today}

\begin{document}

\maketitle

\section{Enkla andragradsekvationer}

\begin{enumerate}[label=\textbf{\arabic*.}]
    \item Lös ekvationen: $x^2 = 16$
    
    \item Lös ekvationen: $x^2 - 9 = 0$
    
    \item Lös ekvationen: $2x^2 = 18$
    
    \item Lös ekvationen: $3x^2 - 12 = 0$
    
    \item Lös ekvationen: $(x - 2)^2 = 9$
\end{enumerate}

\section{Andragradsekvationer med nollproduktsmetoden}

\begin{enumerate}[label=\textbf{\arabic*.}]
    \item Lös ekvationen: $x^2 - 5x = 0$
    
    \item Lös ekvationen: $(x - 3)(x + 2) = 0$
    
    \item Lös ekvationen: $x(x - 7) = 0$
    
    \item Lös ekvationen: $(2x + 1)(x - 4) = 0$
\end{enumerate}

\section{Andragradsekvationer med lösningsformel (pq-formel)}

\begin{enumerate}[label=\textbf{\arabic*.}]
    \item Lös ekvationen med pq-formeln: $x^2 - 6x + 8 = 0$
    
    \item Lös ekvationen med pq-formeln: $x^2 + 2x - 8 = 0$
    
    \item Lös ekvationen med pq-formeln: $x^2 - 4x + 4 = 0$
    
    \item Lös ekvationen med pq-formeln: $2x^2 - 7x + 3 = 0$ (Omvandla först till standardform)
    
    \item Lös ekvationen med pq-formeln: $3x^2 + 6x - 9 = 0$ 
    
    \item Lös ekvationen med pq-formeln: $5x^2 - 10 = 15x$ 
\end{enumerate}

\section{Logaritmer}

\begin{enumerate}[label=\textbf{\arabic*.}]
    \item Beräkna: $\lg(100 \cdot 1000)$
    
    \item Beräkna: $\lg\left(\frac{100}{0,1}\right)$
    
    \item Beräkna: $\lg 5 + \lg 20$
    
    \item Beräkna: $\lg 50 - \lg 2$
    
    \item Lös ekvationen: $\lg x = 2,5$
    
    \item Lös ekvationen: $\lg x + \lg (x-9) = 1$
    
    \item Lös ekvationen: $2^x = 32$
    
    \item Lös ekvationen: $3^{x-1} = 27$
\end{enumerate}

\section{Enkla exponentialekvationer (lös med logaritmer)}

\begin{enumerate}[label=\textbf{\arabic*.}]
    \item Lös ekvationen: $2^x = 7$
    
    \item Lös ekvationen: $5^x = 20$
    
    \item Lös ekvationen: $10^{2x} = 50$
    
    \item Lös ekvationen: $3^{x+1} = 15$
    
    \item Lös ekvationen: $4 \cdot 2^x = 32$
\end{enumerate}

\section{Enkla ekvationssystem}

\begin{enumerate}[label=\textbf{\arabic*.}]  
    \item Lös ekvationssystemet:
    \begin{align*}
    3x + 2y &= 7\\
    x - y &= 4
    \end{align*}
    
    \item Lös ekvationssystemet:
    \begin{align*}
    4x - 3y &= 10\\
    2x + y &= 8
    \end{align*}
    
    \item Lös ekvationssystemet:
    \begin{align*}
    x + 2y &= 5\\
    3x - y &= 4
    \end{align*}
\end{enumerate}

\section{Blandade uppgifter}

\begin{enumerate}[label=\textbf{\arabic*.}]
    \item Lös ekvationen: $x^2 - 25 = 0$
    
    \item Lös ekvationen: $(x - 1)(x + 6) = 0$
    
    \item Lös ekvationen med pq-formeln: $x^2 - 3x - 4 = 0$
    
    \item Lös ekvationssystemet:
    \begin{align*}
    2x + 3y &= 12\\
    x - y &= 2
    \end{align*}
    
    \item Lös ekvationen: $\log (2x) = 2$
    
    \item Lös ekvationen: $3x^2 = 27$
    
    \item Lös ekvationen: $(2x - 1)(x + 3) = 0$
    
    \item Lös ekvationssystemet:
    \begin{align*}
    x + y &= 6\\
    2x - 3y &= -3
    \end{align*}
    
    \item Lös ekvationen: $10^{x-1} = 100$
\end{enumerate}

\end{document}