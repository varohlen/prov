\documentclass[a4paper,11pt]{article}
\usepackage[utf8]{inputenc}
\usepackage[T1]{fontenc}
\usepackage[swedish]{babel}
\usepackage{amsmath,amssymb,amsfonts}
\usepackage{graphicx}
\usepackage{enumitem}
\usepackage{geometry}
\usepackage{tikz}
\geometry{margin=2.5cm}

\title{Repetitionsuppgifter -- Matematik 2b (Facit)}
\author{Fabian Tingstrand}
\date{\today}

\begin{document}

\maketitle

\section{Analys av andragradsfunktioner}

\begin{enumerate}[label=\textbf{\arabic*.}]
    \item För funktionen $f(x) = x^2 - 6x + 5$:
    \begin{enumerate}[label=\alph*)]
        \item Bestäm funktionens nollställen
        \\ \textbf{Facit:} $x^2 - 6x + 5 = 0 \Rightarrow (x-1)(x-5)=0 \Rightarrow x=1$ eller $x=5$
        \item Bestäm symmetrilinjen
        \\ \textbf{Facit:} $x=\frac{6}{2}=3$
        \item Bestäm extrempunkten och avgör om det är ett maximum eller minimum
        \\ \textbf{Facit:} Extrempunkt i $(3, -4)$. Det är ett minimum (a>0).
    \end{enumerate}
    
    \item Nedan visas grafen till en andragradsfunktion $f(x) = ax^2 + bx + c$:
    
    % Tikz-bild här
    
    \begin{enumerate}[label=\alph*)]
        \item Bestäm funktionens nollställen
        \\ \textbf{Facit:} $x=3$ eller $x=-1$
        \item Bestäm symmetrilinjen
        \\ \textbf{Facit:} $x=1$
        \item Bestäm funktionsuttrycket $f(x) = ax^2 + bx + c$
        \\ \textbf{Facit:} $f(x) = -x^2 + 2x + 3$
    \end{enumerate}

    \item För funktionen $f(x) = 3x^2 + 6x - 2$:
    \begin{enumerate}[label=\alph*)]
        \item Bestäm funktionens nollställen
        \\ \textbf{Facit:} $x = \frac{-3 \pm \sqrt{15}}{3}$
        \item Bestäm symmetrilinjen
        \\ \textbf{Facit:} $x=-1$
        \item Bestäm extrempunkten och avgör om det är ett maximum eller minimum
        \\ \textbf{Facit:} $(-1, -5)$, minimum (a>0)
    \end{enumerate}
    
    \item För funktionen $f(x) = -x^2 + 4x + 5$:
    \begin{enumerate}[label=\alph*)]
        \item Bestäm funktionens nollställen
        \\ \textbf{Facit:} $x=5$ eller $x=-1$
        \item Bestäm symmetrilinjen
        \\ \textbf{Facit:} $x=2$
        \item Bestäm extrempunkten och avgör om det är ett maximum eller minimum
        \\ \textbf{Facit:} $(2,9)$, maximum (a<0)
    \end{enumerate}
\end{enumerate}


\section{Problemlösning med andragradsfunktioner}

\begin{enumerate}[label=\textbf{\arabic*.}]
    \item En boll kastas rakt uppåt från marken med en utgångshastighet på $20$ m/s. Bollens höjd $h$ (i meter) efter $t$ sekunder ges av funktionen $h(t) = 20t - 5t^2$. 
    \begin{enumerate}[label=\alph*)]
        \item När når bollen sin högsta höjd?
        \\ \textbf{Facit:} Vid symmetrilinjen $t=2$ sekunder
        \item Hur hög når bollen som högst?
        \\ \textbf{Facit:} $h(2) = 20\cdot2 - 5\cdot4 = 20$ meter
        \item När träffar bollen marken igen?
        \\ \textbf{Facit:} $t=0$ eller $t=4$ sekunder
    \end{enumerate}
    
    \item En rektangel har omkretsen $24$ cm. Låt $x$ vara rektangelns bredd.
    \begin{enumerate}[label=\alph*)]
        \item Uttryck rektangelns längd som en funktion av $x$.
        \\ \textbf{Facit:} $l = 12-x$
        \item Uttryck rektangelns area $A$ som en funktion av $x$.
        \\ \textbf{Facit:} $A = x(12-x)$
        \item Vilka värden kan $x$ anta?
        \\ \textbf{Facit:} $0 < x < 12$
        \item För vilket värde på $x$ blir arean maximal?
        \\ \textbf{Facit:} $x=6$
        \item Vad är den maximala arean?
        \\ \textbf{Facit:} $A=36$ cm$^2$
    \end{enumerate}
\end{enumerate}

\end{document}

