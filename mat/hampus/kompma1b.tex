\documentclass[12pt,a4paper]{article}
\usepackage[utf8]{inputenc}
\usepackage[T1]{fontenc}
\usepackage[swedish]{babel}
\usepackage{amsmath,amssymb,amsthm}
\usepackage{enumitem}
\usepackage{graphicx}
\usepackage{geometry}
\geometry{margin=2.5cm}

\title{Kompletteringsuppgifter Matematik 1b}
\author{Viktor Arohlén}
\date{2025}

\begin{document}

\maketitle

\section*{Instruktioner}
Redovisa dina lösningar på ett separat papper för varje uppgift. Digitala verktyg är inte tillåtna.

\section*{Uppgifter}

\begin{enumerate}[label=\textbf{\arabic*.}, itemsep=1cm]
    \item Vilket uttryck ska stå i den tomma parentesen för att likheten ska gälla? \\ \\
        $2(\underline{\hspace{1.5cm}}) = x (4x + 10)$
    
    \item Faktorisera $27x^3y - 9x^2y^3 + 3xy$ fullständigt.
    
    \item Utveckla och förenkla uttrycket $(x + 2)(x + 3)$ och förenkla så långt som möjligt.
    
    \item Linjen L1 har ekvationen $y = 3x + 19$
        \begin{enumerate}[label=\textbf{\alph*)}, itemsep=0.5cm]
            \item Visa att punkten $(10,49)$ ligger på linjen L1
            \item Linjen L2 är parallell med linjen L1. Punkten $(7,11)$ ligger på linjen L2. \\
            Bestäm ekvationen för linjen L2
        \end{enumerate}

    \item Om 20 år kommer Khaleb att vara lika gammal som Erika är idag. Om 10 år kommer Erika att vara dubbelt så gammal som Khaleb är vid den tidpunkten.

    Hur gamla är de två idag? 

    \item  I ett provrör finns det bakterier. Varje dag fördubblas antalet. Efter 12 dagar finns det
    $2^{18}$ bakterier i provröret.
    Hur många bakterier fanns det i provröret när det hade gått 10 dagar?

\end{enumerate}



\end{document}