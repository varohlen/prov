\documentclass[a4paper,11pt]{article}
\usepackage[utf8]{inputenc}
\usepackage[T1]{fontenc}
\usepackage[swedish]{babel}
\usepackage{amsmath,amssymb,amsfonts}
\usepackage{graphicx}
\usepackage{enumitem}
\usepackage{geometry}
\usepackage{tikz}
\geometry{margin=2.5cm}

\title{Repetitionsuppgifter -- Matematik 1 (Facit)}
\author{}
\date{\today}

\begin{document}

\maketitle

\section*{Grundläggande ekvationslösning och olikheter}

\begin{enumerate}[label=\textbf{\arabic*.}]
    \item Lös ekvationen: $2x + 3 = 7$
    \\ \textbf{Facit:} $x = 2$
    
    \item Lös ekvationen: $3x - 5 = 10$
    \\ \textbf{Facit:} $x = 5$
    
    \item Lös ekvationen: $5x + 2 = 3x - 4$
    \\ \textbf{Facit:} $x = -3$
    
    \item Lös ekvationen: $\frac{x}{3} + 2 = 5$
    \\ \textbf{Facit:} $x = 9$
    
    \item Lös ekvationen: $2(x + 3) = 4x - 6$
    \\ \textbf{Facit:} $x = 6$
    
    \item Lös ekvationen: $\frac{x+1}{2} = \frac{x-3}{4}$
    \\ \textbf{Facit:} $x = -5$
    
    \item Lös ekvationen: $3(x - 1) - 2(x + 3) = 5$
    \\ \textbf{Facit:} $x = 14$
    
    \item Lös ekvationen: $\frac{2x-1}{3} + \frac{x+2}{4} = 2$
    \\ \textbf{Facit:} $x = 2$
    
    \item Lös olikheten: $2x + 3 < 7$
    \\ \textbf{Facit:} $x < 2$
    
    \item Lös olikheten: $3x - 5 \geq 10$
    \\ \textbf{Facit:} $x \geq 5$
    
    \item Lös olikheten: $5x + 2 > 3x - 4$
    \\ \textbf{Facit:} $x > -3$
    
    \item Lös olikheten: $\frac{x}{3} + 2 \leq 5$
    \\ \textbf{Facit:} $x \leq 9$
    
    \item Lös olikheten: $2(x + 3) < 4x - 6$
    \\ \textbf{Facit:} $x > 6$
    
    \item Lös olikheten: $-3 < 2x - 5 < 7$
    \\ \textbf{Facit:} $1 < x < 6$
    
    \item Lös olikheten: $\frac{x-1}{2} > \frac{x+3}{4}$
    \\ \textbf{Facit:} $x > 5$
    
    \item Lös olikheten: $3(x - 1) - 2(x + 3) \leq 5$
    \\ \textbf{Facit:} $x \leq 14$
\end{enumerate}


\newpage
\section*{Räta linjens ekvation}

\begin{enumerate}[label=\textbf{\arabic*.}]
    \item Bestäm räta linjens ekvation som går genom punkten $(2, 5)$ och har lutningen $k = 3$.
    \\ \textbf{Facit:} $y = 3x - 1$
    
    \item Bestäm räta linjens ekvation som går genom punkterna $(1, 3)$ och $(4, 9)$.
    \\ \textbf{Facit:} $k = \frac{9-3}{4-1}=2$, $y = 2x + 1$
    
    \item Bestäm räta linjens ekvation som går genom punkten $(3, -2)$ och har lutningen $k = -2$.
    \\ \textbf{Facit:} $y = -2x + 4$
    
    \item Bestäm räta linjens ekvation som går genom punkterna $(-1, 4)$ och $(2, -5)$.
    \\ \textbf{Facit:} $k = \frac{-5-4}{2-(-1)} = -3$, $y = -3x + 1$
    
    \item En rät linje har ekvationen $y = 2x - 3$.
    \begin{enumerate}[label=\alph*)]
        \item Vad är linjens lutning?
        \\ \textbf{Facit:} $k=2$
        \item Var skär linjen $y$-axeln?
        \\ \textbf{Facit:} $m=-3$
        \item Beräkna $y$-värdet då $x = 4$.
        \\ \textbf{Facit:} $y = 2\cdot4 - 3 = 5$
        \item Beräkna $x$-värdet då $y = 5$.
        \\ \textbf{Facit:} $5 = 2x - 3 \Rightarrow x = 4$
    \end{enumerate}
    
    \item En rät linje går genom punkterna $(0, -3)$ och $(2, 5)$.
    \begin{enumerate}[label=\alph*)]
        \item Bestäm linjens ekvation på formen $y = kx + m$.
        \\ \textbf{Facit:} $k = \frac{5-(-3)}{2-0}=4$, $y=4x-3$
        \item Var skär linjen $x$-axeln?
        \\ \textbf{Facit:} $0 = 4x-3 \Rightarrow x = 0.75$
    \end{enumerate}
\end{enumerate}

\section*{Linj0e4ra funktioner}

\begin{enumerate}[label=\textbf{\arabic*.}]
    \item För funktionen $f(x) = 3x - 2$:
    \begin{enumerate}[label=\alph*)]
        \item Beräkna $f(0)$, $f(1)$ och $f(-1)$.
        \\ \textbf{Facit:} $f(0) = -2$, $f(1) = 1$, $f(-1) = -5$
        \item Bestäm $x$ då $f(x) = 7$.
        \\ \textbf{Facit:} $3x-2=7 \Rightarrow x=3$
    \end{enumerate}
    
    \item För funktionen $g(x) = -2x + 5$:
    \begin{enumerate}[label=\alph*)]
        \item Beräkna $g(0)$, $g(2)$ och $g(-3)$.
        \\ \textbf{Facit:} $g(0)=5$, $g(2)=1$, $g(-3)=11$
        \item Bestäm $x$ då $g(x) = -3$.
        \\ \textbf{Facit:} $-2x+5=-3 \Rightarrow x=4$
    \end{enumerate}
    
    \item Rita grafen till funktionen $h(x) = 2x + 1$ för $-3 \leq x \leq 3$.
    \\ \textbf{Facit:} Rita linjen $h(x)$ med k=2 och m=1.
    
    \item Rita grafen till funktionen $p(x) = -x + 3$ för $-2 \leq x \leq 4$.
    \\ \textbf{Facit:} Rita linjen $p(x)$ med k=-1 och m=3.
    
    \item En linjär funktion $f$ har egenskaperna $f(2) = 5$ och $f(4) = 9$.
    \begin{enumerate}[label=\alph*)]
        \item Bestäm funktionens uttryck på formen $f(x) = kx + m$.
        \\ \textbf{Facit:} $k=2$, $m=1$, $f(x)=2x+1$
        \item Beräkna $f(7)$.
        \\ \textbf{Facit:} $f(7)=15$
    \end{enumerate}
    
    \item En linjär funktion $g$ har egenskaperna $g(0) = -3$ och $g(-2) = 1$.
    \begin{enumerate}[label=\alph*)]
        \item Bestäm funktionens uttryck på formen $g(x) = kx + m$.
        \\ \textbf{Facit:} $k=-2$, $m=-3$, $g(x)=-2x-3$
        \item Bestäm $x$ då $g(x) = 0$.
        \\ \textbf{Facit:} $-2x-3=0 \Rightarrow x=-1.5$
    \end{enumerate}
\end{enumerate}

\newpage
\section*{Exponentiella funktioner}

\begin{enumerate}[label=\textbf{\arabic*.}]
    \item För funktionen $f(x) = 2^x$:
    \begin{enumerate}[label=\alph*)]
        \item Beräkna $f(0)$, $f(1)$, $f(2)$ och $f(3)$.
        \\ \textbf{Facit:} $f(0)=1$, $f(1)=2$, $f(2)=4$, $f(3)=8$
        \item Bestäm $x$ då $f(x) = 8$.
        \\ \textbf{Facit:} $x=3$
    \end{enumerate}
    
    \item För funktionen $g(x) = 3 \cdot 2^x$:
    \begin{enumerate}[label=\alph*)]
        \item Beräkna $g(0)$, $g(1)$ och $g(2)$.
        \\ \textbf{Facit:} $g(0)=3$, $g(1)=6$, $g(2)=12$
        \item Bestäm $x$ då $g(x) = 24$.
        \\ \textbf{Facit:} $x=3$
    \end{enumerate}
    
    \item För funktionen $h(x) = 5 \cdot 3^x$:
    \begin{enumerate}[label=\alph*)]
        \item Beräkna $h(0)$, $h(1)$ och $h(-1)$.
        \\ \textbf{Facit:} $h(0)=5$, $h(1)=15$, $h(-1)=5/3$
        \item Bestäm $x$ då $h(x) = 45$.
        \\ \textbf{Facit:} $x=2$
    \end{enumerate}
    
    \item För funktionen $p(x) = 100 \cdot 0,8^x$:
    \begin{enumerate}[label=\alph*)]
        \item Beräkna $p(0)$, $p(1)$ och $p(2)$.
        \\ \textbf{Facit:} $p(0)=100$, $p(1)=80$, $p(2)=64$
        \item Bestäm $x$ då $p(x) = 50$.
        \\ \textbf{Facit:} $x=\log_{0,8}(0,5) \approx 3.106$
    \end{enumerate}
    
    \item Rita grafen till funktionen $f(x) = 2^x$ för $-2 \leq x \leq 3$.
    \\ \textbf{Facit:} Rita grafen.
    
    \item Rita grafen till funktionen $g(x) = 0,5^x$ för $-2 \leq x \leq 3$.
    \\ \textbf{Facit:} Rita grafen.
\end{enumerate}

\newpage
\section*{Problemlösning med funktioner}

% Här kan du lägga till facit till specifika problemlösningsuppgifter vid behov

\end{document}

