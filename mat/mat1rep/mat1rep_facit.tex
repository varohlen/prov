\documentclass[a4paper,11pt]{article}
\usepackage[utf8]{inputenc}
\usepackage[T1]{fontenc}
\usepackage[swedish]{babel}
\usepackage{amsmath,amssymb,amsfonts}
\usepackage{graphicx}
\usepackage{enumitem}
\usepackage{geometry}
\usepackage{tikz}
\geometry{margin=2.5cm}

\title{Repetitionsuppgifter -- Matematik 1 (Facit)}
\author{}
\date{\today}

\begin{document}

\maketitle

\section*{Grundläggande ekvationslösning och olikheter}

\begin{enumerate}[label=\textbf{\arabic*.}]
    \item Lös ekvationen: $2x + 3 = 7$
    \\ \textbf{Facit:} $x = 2$
    
    \item Lös ekvationen: $3x - 5 = 10$
    \\ \textbf{Facit:} $x = 5$
    
    \item Lös ekvationen: $5x + 2 = 3x - 4$
    \\ \textbf{Facit:} $x = -3$
    
    \item Lös ekvationen: $\frac{x}{3} + 2 = 5$
    \\ \textbf{Facit:} $x = 9$
    
    \item Lös ekvationen: $2(x + 3) = 4x - 6$
    \\ \textbf{Facit:} $x = 6$
    
    \item Lös ekvationen: $\frac{x+1}{2} = \frac{x-3}{4}$
    \\ \textbf{Facit:} $x = -5$
    
    \item Lös ekvationen: $3(x - 1) - 2(x + 3) = 5$
    \\ \textbf{Facit:} $x = 14$
    
    \item Lös ekvationen: $\frac{2x-1}{3} + \frac{x+2}{4} = 2$
    \\ \textbf{Facit:} $x = 2$
    
    \item Lös olikheten: $2x + 3 < 7$
    \\ \textbf{Facit:} $x < 2$
    
    \item Lös olikheten: $3x - 5 \geq 10$
    \\ \textbf{Facit:} $x \geq 5$
    
    \item Lös olikheten: $5x + 2 > 3x - 4$
    \\ \textbf{Facit:} $x > -3$
    
    \item Lös olikheten: $\frac{x}{3} + 2 \leq 5$
    \\ \textbf{Facit:} $x \leq 9$
    
    \item Lös olikheten: $2(x + 3) < 4x - 6$
    \\ \textbf{Facit:} $x > 6$
    
    \item Lös olikheten: $-3 < 2x - 5 < 7$
    \\ \textbf{Facit:} $1 < x < 6$
    
    \item Lös olikheten: $\frac{x-1}{2} > \frac{x+3}{4}$
    \\ \textbf{Facit:} $x > 5$
    
    \item Lös olikheten: $3(x - 1) - 2(x + 3) \leq 5$
    \\ \textbf{Facit:} $x \leq 14$
\end{enumerate}


\newpage
\section*{Räta linjens ekvation}

\begin{enumerate}[label=\textbf{\arabic*.}]
    \item Bestäm räta linjens ekvation som går genom punkten $(2, 5)$ och har lutningen $k = 3$.
    \\ \textbf{Facit:} $y = 3x - 1$
    
    \item Bestäm räta linjens ekvation som går genom punkterna $(1, 3)$ och $(4, 9)$.
    \\ \textbf{Facit:} $k = \frac{9-3}{4-1}=2$, $y = 2x + 1$
    
    \item Bestäm räta linjens ekvation som går genom punkten $(3, -2)$ och har lutningen $k = -2$.
    \\ \textbf{Facit:} $y = -2x + 4$
    
    \item Bestäm räta linjens ekvation som går genom punkterna $(-1, 4)$ och $(2, -5)$.
    \\ \textbf{Facit:} $k = \frac{-5-4}{2-(-1)} = -3$, $y = -3x + 1$
    
    \item En rät linje har ekvationen $y = 2x - 3$.
    \begin{enumerate}[label=\alph*)]
        \item Vad är linjens lutning?
        \\ \textbf{Facit:} $k=2$
        \item Var skär linjen $y$-axeln?
        \\ \textbf{Facit:} $m=-3$
        \item Beräkna $y$-värdet då $x = 4$.
        \\ \textbf{Facit:} $y = 2\cdot4 - 3 = 5$
        \item Beräkna $x$-värdet då $y = 5$.
        \\ \textbf{Facit:} $5 = 2x - 3 \Rightarrow x = 4$
    \end{enumerate}
    
    \item En rät linje går genom punkterna $(0, -3)$ och $(2, 5)$.
    \begin{enumerate}[label=\alph*)]
        \item Bestäm linjens ekvation på formen $y = kx + m$.
        \\ \textbf{Facit:} $k = \frac{5-(-3)}{2-0}=4$, $y=4x-3$
        \item Var skär linjen $x$-axeln?
        \\ \textbf{Facit:} $0 = 4x-3 \Rightarrow x = 0.75$
    \end{enumerate}
\end{enumerate}

\section*{Linjära funktioner}

\begin{enumerate}[label=\textbf{\arabic*.}]
    \item För funktionen $f(x) = 3x - 2$:
    \begin{enumerate}[label=\alph*)]
        \item Beräkna $f(0)$, $f(1)$ och $f(-1)$.
        \\ \textbf{Facit:} $f(0) = -2$, $f(1) = 1$, $f(-1) = -5$
        \item Bestäm $x$ då $f(x) = 7$.
        \\ \textbf{Facit:} $3x-2=7 \Rightarrow x=3$
    \end{enumerate}
    
    \item För funktionen $g(x) = -2x + 5$:
    \begin{enumerate}[label=\alph*)]
        \item Beräkna $g(0)$, $g(2)$ och $g(-3)$.
        \\ \textbf{Facit:} $g(0)=5$, $g(2)=1$, $g(-3)=11$
        \item Bestäm $x$ då $g(x) = -3$.
        \\ \textbf{Facit:} $-2x+5=-3 \Rightarrow x=4$
    \end{enumerate}
    
    \item Rita grafen till funktionen $h(x) = 2x + 1$ för $-3 \leq x \leq 3$.
    \\ \textbf{Facit:} Rita linjen $h(x)$ med k=2 och m=1.
    
    \item Rita grafen till funktionen $p(x) = -x + 3$ för $-2 \leq x \leq 4$.
    \\ \textbf{Facit:} Rita linjen $p(x)$ med k=-1 och m=3.
    
    \item En linjär funktion $f$ har egenskaperna $f(2) = 5$ och $f(4) = 9$.
    \begin{enumerate}[label=\alph*)]
        \item Bestäm funktionens uttryck på formen $f(x) = kx + m$.
        \\ \textbf{Facit:} $k=2$, $m=1$, $f(x)=2x+1$
        \item Beräkna $f(7)$.
        \\ \textbf{Facit:} $f(7)=15$
    \end{enumerate}
    
    \item En linjär funktion $g$ har egenskaperna $g(0) = -3$ och $g(-2) = 1$.
    \begin{enumerate}[label=\alph*)]
        \item Bestäm funktionens uttryck på formen $g(x) = kx + m$.
        \\ \textbf{Facit:} $k=-2$, $m=-3$, $g(x)=-2x-3$
        \item Bestäm $x$ då $g(x) = 0$.
        \\ \textbf{Facit:} $-2x-3=0 \Rightarrow x=-1.5$
    \end{enumerate}
\end{enumerate}

\newpage
\section*{Exponentiella funktioner}

\begin{enumerate}[label=\textbf{\arabic*.}]
    \item För funktionen $f(x) = 2^x$:
    \begin{enumerate}[label=\alph*)]
        \item Beräkna $f(0)$, $f(1)$, $f(2)$ och $f(3)$.
        \\ \textbf{Facit:} $f(0)=1$, $f(1)=2$, $f(2)=4$, $f(3)=8$
        \item Bestäm $x$ då $f(x) = 8$.
        \\ \textbf{Facit:} $2^x=8 \Rightarrow x=3$
    \end{enumerate}
    
    \item För funktionen $g(x) = 3 \cdot 2^x$:
    \begin{enumerate}[label=\alph*)]
        \item Beräkna $g(0)$, $g(1)$ och $g(2)$.
        \\ \textbf{Facit:} $g(0)=3$, $g(1)=6$, $g(2)=12$
        \item Bestäm $x$ då $g(x) = 24$.
        \\ \textbf{Facit:} $3\cdot2^x=24 \Rightarrow 2^x=8 \Rightarrow x=3$
    \end{enumerate}
    
    \item För funktionen $h(x) = 5 \cdot 3^x$:
    \begin{enumerate}[label=\alph*)]
        \item Beräkna $h(0)$, $h(1)$ och $h(-1)$.
        \\ \textbf{Facit:} $h(0)=5$, $h(1)=15$, $h(-1)=5/3$
        \item Bestäm $x$ då $h(x) = 45$.
        \\ \textbf{Facit:} $5\cdot3^x=45 \Rightarrow 3^x=9 \Rightarrow x=2$
    \end{enumerate}
    
    \item För funktionen $p(x) = 100 \cdot 0,8^x$:
    \begin{enumerate}[label=\alph*)]
        \item Beräkna $p(0)$, $p(1)$ och $p(2)$.
        \\ \textbf{Facit:} $p(0)=100$, $p(1)=80$, $p(2)=64$
        \item Bestäm $x$ då $p(x) = 50$.
        \\ \textbf{Facit:} $100\cdot0,8^x=50 \Rightarrow 0,8^x=0,5 \Rightarrow x=\log_{0,8}(0,5) \approx 3,106$
    \end{enumerate}
    
    \item Rita grafen till funktionen $f(x) = 2^x$ för $-2 \leq x \leq 3$.
    \\ \textbf{Facit:} Rita grafen med punkterna $(-2,0.25)$, $(-1,0.5)$, $(0,1)$, $(1,2)$, $(2,4)$, $(3,8)$.
    
    \item Rita grafen till funktionen $g(x) = 0,5^x$ för $-2 \leq x \leq 3$.
    \\ \textbf{Facit:} Rita grafen med punkterna $(-2,4)$, $(-1,2)$, $(0,1)$, $(1,0.5)$, $(2,0.25)$, $(3,0.125)$.
\end{enumerate}

\newpage
\section*{Problemlösning med funktioner}

\begin{enumerate}[label=\textbf{\arabic*.}]
    \item En funktion $f(x) = 2x + 1$. Lös $f(x) = 9$.
    \\ \textbf{Facit:} $2x+1=9 \Rightarrow x=4$
    \item En funktion $g(x) = 3^x$. Bestäm $x$ då $g(x) = 27$.
    \\ \textbf{Facit:} $3^x=27 \Rightarrow x=3$
    \item En funktion $h(x) = 5 - x$. Bestäm $x$ då $h(x) = 2$.
    \\ \textbf{Facit:} $5-x=2 \Rightarrow x=3$
    \item En funktion $p(x) = 100 \cdot 0,8^x$. När är $p(x) = 32$?
    \\ \textbf{Facit:} $100\cdot0,8^x=32 \Rightarrow 0,8^x=0,32 \Rightarrow x=\log_{0,8}(0,32) \approx 6.6439$
\end{enumerate}

\newpage
\section*{Procent och förändringsfaktor}

\begin{enumerate}[label=\textbf{\arabic*.}]
    \item Beräkna:
    \begin{enumerate}[label=\alph*)]
        \item 15\% av 400
        \\ \textbf{Facit:} $0,15 \cdot 400 = 60$
        \item 7,5\% av 80
        \\ \textbf{Facit:} $0,075 \cdot 80 = 6$
        \item 120\% av 50
        \\ \textbf{Facit:} $1,2 \cdot 50 = 60$
    \end{enumerate}
    
    \item Hur många procent är:
    \begin{enumerate}[label=\alph*)]
        \item 30 av 150
        \\ \textbf{Facit:} $\frac{30}{150} = 0,2 = 20\%$
        \item 45 av 180
        \\ \textbf{Facit:} $\frac{45}{180} = 0,25 = 25\%$
        \item 5 av 25
        \\ \textbf{Facit:} $\frac{5}{25} = 0,2 = 20\%$
    \end{enumerate}
    
    \item Ett klädesplagg kostar 800 kr. Under en rea sänks priset med 25\%.
    \begin{enumerate}[label=\alph*)]
        \item Vad blir det nya priset?
        \\ \textbf{Facit:} $800 \cdot 0,75 = 600$ kr
        \item Vilken förändringsfaktor motsvarar prissänkningen?
        \\ \textbf{Facit:} $0,75$
    \end{enumerate}
    
    \item En vara kostar 500 kr. Priset höjs med 12\%.
    \begin{enumerate}[label=\alph*)]
        \item Vad blir det nya priset?
        \\ \textbf{Facit:} $500 \cdot 1,12 = 560$ kr
        \item Vilken förändringsfaktor motsvarar prishöjningen?
        \\ \textbf{Facit:} $1,12$
    \end{enumerate}
    
    \item Priset på en vara höjs från 200 kr till 250 kr.
    \begin{enumerate}[label=\alph*)]
        \item Hur många procent höjs priset?
        \\ \textbf{Facit:} $\frac{250-200}{200} = 0,25 = 25\%$
        \item Vilken förändringsfaktor motsvarar prishöjningen?
        \\ \textbf{Facit:} $\frac{250}{200} = 1,25$
    \end{enumerate}
    
    \item Antalet invånare i en stad minskar från 45\,000 till 40\,500.
    \begin{enumerate}[label=\alph*)]
        \item Hur många procent minskar befolkningen?
        \\ \textbf{Facit:} $\frac{45\,000 - 40\,500}{45\,000} = \frac{4\,500}{45\,000} = 0,1 = 10\%$
        \item Vilken förändringsfaktor motsvarar minskningen?
        \\ \textbf{Facit:} $\frac{40\,500}{45\,000} = 0,9$
    \end{enumerate}
    
    \item En vara kostar 400 kr. Priset höjs först med 20\% och sedan med ytterligare 10\%.
    \begin{enumerate}[label=\alph*)]
        \item Vad blir det slutliga priset?
        \\ \textbf{Facit:} $400 \cdot 1,2 \cdot 1,1 = 528$ kr
        \item Hur många procent har priset totalt höjts med?
        \\ \textbf{Facit:} $\frac{528-400}{400} = 0,32 = 32\%$
        \item Vilken förändringsfaktor motsvarar den totala prishöjningen?
        \\ \textbf{Facit:} $1,2 \cdot 1,1 = 1,32$
    \end{enumerate}
    
    \item En vara kostar 600 kr. Under en rea sänks priset med 30\%. Efter rean höjs priset med 40\%.
    \begin{enumerate}[label=\alph*)]
        \item Vad blir det slutliga priset?
        \\ \textbf{Facit:} $600 \cdot 0,7 \cdot 1,4 = 588$ kr
        \item Hur många procent har priset totalt förändrats med?
        \\ \textbf{Facit:} $\frac{588-600}{600} = -0,02 = -2\%$
        \item Vilken förändringsfaktor motsvarar den totala prisförändringen?
        \\ \textbf{Facit:} $0,7 \cdot 1,4 = 0,98$
    \end{enumerate}
\end{enumerate}

\newpage
\section*{Sannolikhet}

\subsection*{Enkla slumpförsök}

\begin{enumerate}[label=\textbf{\arabic*.}]
    \item En vanlig tärning kastas en gång. Beräkna sannolikheten för att:
    \begin{enumerate}[label=\alph*)]
        \item Få en 6:a
        \\ \textbf{Facit:} $P=\frac{1}{6}$
        \item Få ett jämnt tal
        \\ \textbf{Facit:} $P=\frac{3}{6}=\frac{1}{2}$
        \item Få ett tal som är större än 4
        \\ \textbf{Facit:} $P=\frac{2}{6}=\frac{1}{3}$
    \end{enumerate}
    
    \item En kortlek med 52 kort innehåller 13 kort av varje färg (hjärter, ruter, klöver, spader). Beräkna sannolikheten för att dra:
    \begin{enumerate}[label=\alph*)]
        \item Ett hjärter
        \\ \textbf{Facit:} $P=\frac{13}{52}=\frac{1}{4}$
        \item Ett ess
        \\ \textbf{Facit:} $P=\frac{4}{52}=\frac{1}{13}$
        \item Ett svart kort
        \\ \textbf{Facit:} $P=\frac{26}{52}=\frac{1}{2}$
    \end{enumerate}
    
    \item I en urna finns 5 röda, 3 blå och 2 gröna kulor. En kula dras slumpmässigt. Beräkna sannolikheten för att:
    \begin{enumerate}[label=\alph*)]
        \item Kulan är röd
        \\ \textbf{Facit:} $P=\frac{5}{10}=0,5$
        \item Kulan är blå eller grön
        \\ \textbf{Facit:} $P=\frac{3+2}{10}=0,5$
        \item Kulan är varken röd eller blå
        \\ \textbf{Facit:} $P=\frac{2}{10}=0,2$
    \end{enumerate}
    
    \item I en klass med 30 elever är 18 flickor och 12 pojkar. Av flickorna har 6 glasögon och av pojkarna har 4 glasögon. En elev väljs slumpmässigt. Beräkna sannolikheten för att:
    \begin{enumerate}[label=\alph*)]
        \item Eleven är en flicka
        \\ \textbf{Facit:} $P=\frac{18}{30}=0,6$
        \item Eleven har glasögon
        \\ \textbf{Facit:} $P=\frac{6+4}{30}=\frac{10}{30}=0,333$
        \item Eleven är en pojke med glasögon
        \\ \textbf{Facit:} $P=\frac{4}{30}=0,133$
    \end{enumerate}
\end{enumerate}

\subsection*{Slumpförsök i flera steg}

\begin{enumerate}[label=\textbf{\arabic*.}]
    \item En vanlig tärning kastas två gånger. Beräkna sannolikheten för att:
    \begin{enumerate}[label=\alph*)]
        \item Få två 6:or
        \\ \textbf{Facit:} $P=\frac{1}{6}\cdot\frac{1}{6}=\frac{1}{36}$
        \item Få summan 7
        \\ \textbf{Facit:} $P=\frac{6}{36}=\frac{1}{6}$ (möjliga utfall: (1,6),(2,5),(3,4),(4,3),(5,2),(6,1))
        \item Få minst en 6:a
        \\ \textbf{Facit:} $P=1-(\frac{5}{6}\cdot\frac{5}{6})=1-\frac{25}{36}=\frac{11}{36}$
    \end{enumerate}
    
    \item Två vanliga tärningar kastas samtidigt. Beräkna sannolikheten för att:
    \begin{enumerate}[label=\alph*)]
        \item Få samma tal på båda tärningarna
        \\ \textbf{Facit:} $P=\frac{6}{36}=\frac{1}{6}$
        \item Få summan 8
        \\ \textbf{Facit:} $P=\frac{5}{36}$ (möjliga utfall: (2,6),(3,5),(4,4),(5,3),(6,2))
        \item Få en summa som är högst 4
        \\ \textbf{Facit:} $P=\frac{6}{36}=\frac{1}{6}$ (möjliga utfall: (1,1),(1,2),(2,1),(1,3),(3,1),(2,2))
    \end{enumerate}
    
    \item Från en kortlek med 52 kort dras två kort i följd utan återläggning. Beräkna sannolikheten för att:
    \begin{enumerate}[label=\alph*)]
        \item Båda korten är ess
        \\ \textbf{Facit:} $P=\frac{4}{52}\cdot\frac{3}{51}=\frac{12}{2652}=\frac{1}{221}$
        \item Första kortet är ett ess och andra kortet är en kung
        \\ \textbf{Facit:} $P=\frac{4}{52}\cdot\frac{4}{51}=\frac{16}{2652}=\frac{4}{663}$
        \item Båda korten är röda
        \\ \textbf{Facit:} $P=\frac{26}{52}\cdot\frac{25}{51}=\frac{650}{2652}=\frac{25}{102}$
    \end{enumerate}
    
    \item I en urna finns 4 vita och 6 svarta kulor. Två kulor dras slumpmässigt utan återläggning. Beräkna sannolikheten för att:
    \begin{enumerate}[label=\alph*)]
        \item Båda kulorna är vita
        \\ \textbf{Facit:} $P=\frac{4}{10}\cdot\frac{3}{9}=\frac{12}{90}=\frac{2}{15}$
        \item Båda kulorna är svarta
        \\ \textbf{Facit:} $P=\frac{6}{10}\cdot\frac{5}{9}=\frac{30}{90}=\frac{1}{3}$
        \item En kula är vit och en kula är svart
        \\ \textbf{Facit:} $P=\frac{4}{10}\cdot\frac{6}{9}+\frac{6}{10}\cdot\frac{4}{9}=2\cdot\frac{24}{90}=\frac{48}{90}=\frac{8}{15}$
    \end{enumerate}
    
    \item En påse innehåller 3 röda, 2 blå och 1 grön kula. Du drar slumpmässigt två kulor i följd utan återläggning.
    \begin{enumerate}[label=\alph*)]
        \item Rita ett träddiagram för detta slumpförsök.
        \\ \textbf{Facit:} Träddiagram visar första draget (Röd, Blå, Grön) och för varje gren andra draget (utan återläggning).
        \item Beräkna med hjälp av träddiagrammet sannolikheten att få två kulor med olika färg.
        \\ \textbf{Facit:} $P=1-($sannolikheten att båda har samma färg$) = 1-(\frac{3}{6}\cdot\frac{2}{5}+\frac{2}{6}\cdot\frac{1}{5}+\frac{1}{6}\cdot0)=1-(\frac{6}{30}+\frac{2}{30})=1-\frac{8}{30}=\frac{22}{30}=\frac{11}{15}$
        \item Beräkna sannolikheten att få minst en röd kula.
        \\ \textbf{Facit:} $P=1-($ingen röd$)=1-(\frac{2}{6}\cdot\frac{1}{5})=1-\frac{2}{30}=\frac{28}{30}=\frac{14}{15}$
    \end{enumerate}
    
    \item En skola har två klasser med 20 elever i varje klass. I klass A är 12 elever flickor och i klass B är 8 elever flickor. En elev väljs slumpmässigt från hela skolan genom att först välja en klass och sedan en elev från den valda klassen.
    \begin{enumerate}[label=\alph*)]
        \item Rita ett träddiagram för detta slumpförsök.
        \\ \textbf{Facit:} Träddiagram: Första steg: Klass A (20/40), Klass B (20/40). Andra steg: Flicka/pojke i respektive klass.
        \item Beräkna med hjälp av träddiagrammet sannolikheten att den valda eleven är en flicka.
        \\ \textbf{Facit:} $P=\frac{20}{40}\cdot\frac{12}{20}+\frac{20}{40}\cdot\frac{8}{20}=0,5\cdot0,6+0,5\cdot0,4=0,3+0,2=0,5$
        \item Om den valda eleven visar sig vara en flicka, vad är sannolikheten att hon kommer från klass A?
        \\ \textbf{Facit:} $P=\frac{0,3}{0,5}=0,6$
    \end{enumerate}
\end{enumerate}

\end{document}


